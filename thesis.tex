\documentclass[12pt, a4paper, twoside, openright, english]{book}

\usepackage[english]{babel}
\usepackage[utf8]{inputenc}
\usepackage[T1]{fontenc}
\usepackage[top=2.5cm,
	bottom=3cm,
	right=3.2cm,
	left=3.2cm
]{geometry}

\usepackage{subcaption}
\usepackage{hyperref}
\usepackage{enumitem}
\usepackage{tabularx}
\usepackage{afterpage}
\usepackage{longtable}
\usepackage{multirow}
\usepackage{amsfonts}
\usepackage{amssymb}
\usepackage{listings}
\usepackage{titlesec}
\usepackage{setspace}
\usepackage{fancyhdr}
\usepackage{fancyvrb}
\usepackage[fleqn]{amsmath}
\usepackage{pdfpages}
\usepackage{nccmath}
\usepackage{tocloft}
\usepackage{csquotes}
\usepackage{diagbox}
\usepackage{ifthen}
\usepackage{algorithm}
\usepackage{algpseudocode}
\usepackage[super]{nth}

\usepackage{colortbl}
\usepackage{tabulary}
\usepackage{adjustbox}


\usepackage{nomencl}
\usepackage{makeidx}
\usepackage{expl3}
\usepackage{etoolbox}
%\preto\tabular{\shorthandoff{-}}

\usepackage[style=iso-numeric, backend=biber]{biblatex}
\addbibresource{literature.bib}

% Zoznam skratiek
\makenomenclature
%\renewcommand{\nomname}{Zoznam skratiek a pojmov}

% Algoritmy
\makeatletter
\renewcommand*{\ALG@name}{Algorithms}
%\renewcommand{\listalgorithmname}{Zoznam algoritmov}
\algrenewcommand\algorithmicrequire{\textbf{Input:}}
\algrenewcommand\algorithmicensure{\textbf{Output:}}
\makeatother

% Empty even pages at the end of chapter
\makeatletter
\renewcommand*{\cleardoublepage}{\clearpage\if@twoside \ifodd\c@page\else
\hbox{}%
\thispagestyle{empty}%
\newpage%
\if@twocolumn\hbox{}\newpage\fi\fi\fi}
\makeatother


% Číslo kapitoly na rovnakom riadku ako názov
\titleformat{\chapter}{\normalfont\huge\bf}{\thechapter}{1em}{}

\raggedbottom
\newcommand{\emptypage}{\newpage\thispagestyle{empty}\mbox{}\newpage}
\newcommand{\signaturespace}[2]{
  \begingroup
  \renewcommand{\arraystretch}{0}
  \begin{tabular}[t]{cc}
  \hspace*{0pt}
  \cleaders\hbox{\kern.6pt.\kern.6pt}\hskip#1\relax
  \hspace*{0pt}
  \\[0.5cm]
  #2
  \end{tabular}
  \endgroup
}

\pagestyle{fancy}
\fancyhf{}  % clear all header and footers
\fancyhead[LE]{\leftmark}
\fancyhead[RO]{\rightmark}
\fancyfoot[LE, RO]{\thepage}

\fancypagestyle{plain}{
  \fancyhf{}
  \renewcommand{\headrulewidth}{0pt}
  \fancyhf[lef,rof]{\thepage}
}

\setlength{\headheight}{16pt}

\renewcommand{\ttdefault}{pcr}
\lstdefinestyle{cstyle}{
    language=C,
	basicstyle=\linespread{1.1}\ttfamily\footnotesize,
    numbers=left,
    numberstyle=\tiny,
    frame=single,
    tabsize=4,
    captionpos=b,
    breaklines=true,
    texcl=true,
	numbersep=8pt,
	framexleftmargin=15pt,
	xleftmargin=5ex,
    xrightmargin=3.4pt,
	morekeywords = {uint8_t,uint16_t,int16_t,uint32_t,int32_t,bool}
}
\lstdefinestyle{docs}{
    language=C,
	basicstyle=\linespread{1.1}\ttfamily\small\bfseries,
    tabsize=4,
    breaklines=true,
    belowskip=0pt
}
\renewcommand{\lstlistingname}{Source code}

\setstretch{1.5}
\newcommand{\University}[0] {Slovenská technická univerzita v Bratislave}
\newcommand{\UniversityEN}[0] {Slovak University of Technology in Bratislava}
\newcommand{\Faculty}[0] {Fakulta informatiky a informačných technológií}
\newcommand{\FacultyEN}[0] {Faculty of Informatics and Information Technologies}
\newcommand{\Thesis}[0] {Diplomová práca}
\newcommand{\ThesisEN}[0] {Master's Thesis}
\newcommand{\Title}[0] {Vibrodiagnostika strojov s~priemyselným internetom vecí}
\newcommand{\TitleEN}[0] {Machinery vibrodiagnostics with the~industrial internet of things}
\newcommand{\Author}[0] {Bc. Miroslav Hájek}
\newcommand{\Supervisor}[0] {Ing. Marcel Baláž, PhD.}
\newcommand{\DepartmentalAdvisor}[0] {Ing. Jakub Findura}
\newcommand{\Consultant}[0] {Ing. Lukáš Doubravský}
\newcommand{\SupervisorEN}[0] {Dr. Marcel Baláž}
\newcommand{\RegNo}[0] {FIIT-xxxx-xxxxxx}
\newcommand{\Date}[0] {Máj 2022}
\newcommand{\DateEN}[0] {May 2023}
\newcommand{\StudyProgramme}[0] {Inteligentné softvérové systémy}
\newcommand{\StudyProgrammeEN}[0] {Intelligent Software Systems}
\newcommand{\StudyField}[0] {Informatika}
\newcommand{\StudyFieldEN}[0] {Informatics}
\newcommand{\Institute}[0] {Institute of Computer Engineering and Applied Informatics}
\newcommand{\SignPlace}[0] {Bratislava, }
\newcommand{\SignDateEN}[0] {May 2023}


\begin{document}
\nomenclature{\textbf{Pojem}}{Vysvetlenie}
% Obal -----------------------------------------------------------------------
\thispagestyle{empty}
{\centering
	{\Large \UniversityEN}\par
	{\Large \FacultyEN}\par
	\vspace{\medskipamount}
	\RegNo
	\vfill
	\textbf{\Large \Author}\par
	\vspace{1.5\bigskipamount}
	\textbf{\LARGE \TitleEN}\par
	\vspace{1.5\bigskipamount}
	{\Large \ThesisEN}\par
	\vfill
}
\begin{flushleft}

{\large Thesis Supervisor: \Supervisor \\
\DateEN}
\end{flushleft}
\emptypage

%  Hlavná časť -----------------------------------------------------------------------
\newgeometry{top=2.5cm, bottom=3cm, right=2.5cm, left=3.5cm}

% Titulný list
\pagenumbering{roman}
\thispagestyle{empty}
{\centering
	{\Large \UniversityEN}\par
	{\Large \FacultyEN}\par
	\vspace{\medskipamount}
	Reg. No. \RegNo
	\vfill
	\textbf{\Large \Author}\par
	\vspace{1.5\bigskipamount}
	\textbf{\LARGE \TitleEN}\par
	\vspace{1.5\bigskipamount}
	{\Large \ThesisEN}\par
	\vfill
}
\begin{flushleft}
\begin{longtable}[l]{ll}
Study programme: & \StudyProgrammeEN \\
Study field: & \StudyFieldEN \\
Training workplace: & \Institute\\
Thesis supervisor: & \Supervisor \\
Departmental advisor: & \DepartmentalAdvisor \\
Consultant: & \Consultant \\
\end{longtable}
\indent\DateEN
\end{flushleft}
\emptypage

% Zadanie
\thispagestyle{empty}
\includepdf[pages=-, scale=1]{chapters/assignment}
\emptypage

 % Declation of Honour
\thispagestyle{empty}
\vspace*{\fill}
\section*{Declaration of Honor}

I hereby declare on my honor that I wrote this thesis independently under the supervision of Dr. Marcel Baláž, after consultations and with use of cited literature.

\vspace{3\medskipamount}\noindent
\SignPlace \SignDateEN \hspace*{\fill} \signaturespace{5cm}{\Author} 

\emptypage
% Poďakovanie
\thispagestyle{empty}
\vspace*{\fill}

\begin{center}
\settowidth\longest{\itshape ---Wir müssen wissen.---}
\parbox{\longest}{
  \hrulefill\hspace{0.2cm} \decofourleft\decofourright \hspace{0.2cm} \hrulefill\par
  \raggedright{
  \itshape
  	Wir müssen wissen. \\ Wir werden wissen.\par
  }   
  \raggedleft{--- David Hilbert}\par
  \hrulefill\hspace{0.2cm} \decofourleft\decofourright\hspace{0.2cm} \hrulefill\par
}
\end{center}

\vspace*{\fill}
\section*{Poďakovanie}
{\linespread{1.0}\small Napriek tomu, že tento text je len kvapka v mori záverečných prác, pre mňa bol proces tvorby omnoho viac než zachytávajú slová na stránkach. Srdečná vďaka patrí školiteľovi Ing.~Marcelovi Balážovi,~PhD. a konzultantovi Ing.~Lukášovi Doubravskému z firmy R-DAS,~s.r.o. Obaja boli otvorení mojim všemožným nápadom a podporovali ma v ich realizácii aj cez mnohé ťažkosti. Za praktický expertný pohľad na vibrodiagnostiku vďačím prof. Ing.~Stanislavovi Žiaranovi,~CSc. a Ing.~Ondrejovi Chlebovi,~PhD. zo Strojníckej fakulty STU. Cením si ústretovosť Bratislavskej vodárenskej spoločnosti,~a.s. v sprístupnení čerpadiel na merania a súvisiacich podkladov, a to konkrétne Ing.~Peterovi Csókovi a Ing.~Petrovi Kmeťkovi. Rovnako ďakujem za ochotu firme VNET,~a.s., konkrétne Michalovi Országovi a Mgr.~Vladimírovi Kupčovi za odporúčania možných strojov na merania v dátových centrách a sprístupnenie kompresorov. Prácu by som rád venoval rodine a kolegom-kamarátom, ktorí stáli pri mne na ,,šialenej akademickej dráhe'' a podnetné diskusie s nimi prispeli aj k môjmu pohľadu na odborné problémy. Nuž a v konečnom dôsledku som mal niekedy len viac šťastia ako rozumu.}
\vspace{3cm}
\emptypage
\thispagestyle{empty}
\section*{Annotation}
\UniversityEN \\
\uppercase{\FacultyEN}
\vspace{-8pt}
{\setlength{\mathindent}{0cm}
\begin{align*}
&\text{Degree course:} && \text{\StudyProgrammeEN} \\
&\text{Author:} && \text{\Author} \\
&\text{\ThesisEN:} && \text{\TitleEN} \\
&\text{Supervisor:} && \text{\SupervisorEN} \\
&\text{\DateEN}
\end{align*}}

\emptypage 

\thispagestyle{empty}
\section*{Anotácia}
\University \\
\uppercase{\Faculty}
\vspace{-8pt}
{\setlength{\mathindent}{0cm}
\begin{align*}
&\text{Študijný program:} && \text{\StudyProgramme} \\
&\text{Autor:} && \text{\Author} \\
&\text{\Thesis:} && \text{\Title} \\
&\text{Vedúci diplomovej práce:} && \text{\Supervisor} \\
&\text{\Date}
\end{align*}}

\emptypage

% Obsah
\pagestyle{empty}
\tableofcontents{}
\listoffigures
{\let\clearpage\relax \printnomenclature}
\emptypage

\pagestyle{fancy}
% Kapitoly
\pagenumbering{arabic}

\chapter{Introduction}
Manufacturing is experiencing a shift in the traditional practices of asset operational status evaluation and utilization. The rise of Industry 4.0 means greater automation and robotization of the production halls to achieve optimal usage of available resources. The secondary aspect in the enterprises' endeavor, however not less important, is to keep track of the equipment wear and tear. The corrective action be it repair or replacement should be taken on time in response to the key indicators. 

The goal is to preserve required safety and production efficiency when extending the useful life of machine moving parts. In the factories and logistics where this sort of equipment is vital, there is a rising interest in the ability to monitor in real-time the health of the machines and to proactively diagnose the fault to repair it without adding unnecessary costs. 

Vibrations are the most nonintrusive way with which such faults can be sensed. The experts use it to distinguish faulty states and to identify the malfunction's root cause. In critical circumstances such as in the case of the large turbines in the power plants, the precautions leading to regular machinery check-ups are already in place. To reach wider acceptance and spread, the monitoring solution has to be sufficiently independent, reliable, and as self-sufficient as the model design allows it to be.

The main issue to consider in large-scale machinery monitoring using vibrations are lots of uninformative streams of samples not directly useful for the production line operator. The dashboard must aggregate these flows into trend variables with severity levels categorized based on industrial standards. The majority of signals are viewed once at the maximum therefore to store or even transmit them from the edge device in its entirety would be wasteful. The complex overview of the mechanical equipment status is attainable only when agent devices and sensors are cheap enough with a long lifespan on battery power and preferably remain physically small to reduce the additional clutter.

Attempted machine and deep learning approaches have the crucial impediment that the construction of every single machine is unique to some extent because of tolerances and variable load. The model must be trained specifically for the target environment to achieve the ideal performance. In addition, the failures are relatively rare events occurring usually in the span of multiple months. In these circumstances, it is hard to obtain a large enough sample of fault events quickly. Novelty detection is a technique that can be applied in this case.

The thesis is organized in the following manner. In the first chapter of analysis in section 1 we explore the mechanical maintenance approaches and industry standards on common fault identification. Then section 2 is all about measuring vibrations and transforming them into features meaningful in automatic fault pattern recognition. In section 3 we delve into modes of diagnosis based on reduced relevant indicators. Section 4 deals with evaluation datasets used to determine computational requirements
on IIoT infrastructure. Chapter 2 defines data format and proposes processing steps to diagnose the imminent failure and different fault types. The approach taken is evaluated and validated in Chapter 6. 
  

\chapter{Problem analysis}
In the problem analysis chapter we expore the feature extraction methods and machine learning algorithms for the fault diagnostics.
The basis we build upon is the domain knowledge of the mechnical engineers about vibration signal measurement and their evaluation.

\section{Condition monitoring}
All rotating machinery eventually fails because of the long-term strain on the individual parts or incorrect workmanship, installation, or operational procedures. In the end, these factors cause the equipment not to fulfill its intended functionality. Many instrumentation methods are practiced to reveal evolving faults: vibration and acoustic noise monitoring, electric supply line measurements, thermography, oil and particle analysis, ultrasonic testing, etc. Vibration signals are the preferred tool for rotating machinery monitoring \cite{mohanty_machinery_2015}. 

The defect needs to be either repaired or replaced, preferably without significant production downtime, further damage to the other attached elements, or any endangerment of the responsible personnel. The maintenance strategies are chosen according to the machine's importance as a result of its failure effect evaluation on the system. The guide to set appropriate maintenance procedures is outlined by IEC 60706-2 standard and involves reliability-centered maintenance (RCM) analysis \cite{el-thalji_predictive_2019}.

\subsection{Maintenance strategies}
There are three different approaches to maintenance across the industry: reactive, preventive, and predictive \cite{scheffer_practical_2004}. In general, the more sophisticated methods are beneficial in a high-stakes environment. The unexpected machine shutdown can have a negative economic impact on the enterprise, resulting in decreased product quality and demands spare parts be ready in the supply inventory at all times. However, in certain situations suffice to utilize a simpler maintenance program, but predictive maintenance gains interest in the Industry 4.0 to optimize assets' usage  \cite{cinar_machine_2020}.
\bigbreak

\textbf{Reactive maintenance} allows machinery to run to complete failure. This is the most inappropriate way to maintain the production line, but it is straightforward. It requires a large stock of replacement parts on-site and breakage inflict a `crisis management mode' upon the plant \cite{scheffer_practical_2004}. On-demand repairs are justified for consumer products or in the factory capable of fully and quickly replacing the halted machine with a backup. 
\bigbreak

\textbf{Preventive maintenance} is performed before any issue is detected. Maintenance occurs at regular intervals derived from a predetermined period in the calendar or expected machine running time (e.g. MTTF - Mean Time To Failure). The schedule is crucial but can result in components being replaced in good condition when further utilization is possible or too late after the machine breaks. In this case, conservative planning is usually the norm to keep machine always in perferct state and therefore more frequent intervention. \cite{mohanty_machinery_2015}.  
\bigbreak

\textbf{Predictive maintenance} known as condition-based maintanance (CbM), improves the predictability of reactive maintenance and eliminates the waste in overall resource utilization of cautious prevention. The machine downtime is scheduled after the detection of unhealthy trends in fault monitoring with sensors and troublesome components are identified. 

Measurable decrease in efectivity allows us to order necessary parts in advance and organize repairs of several machines at a convenient time. The misdetection leads to increased costs compared to previous methods and raises the expectation that faults are distinguishable among themselves~\cite{davies_handbook_2012}.

\bigbreak

\begin{figure}[h]
	\centering
	\includegraphics[width=\textwidth]{assets/P-F-Curve.png}
	\caption{P-F curve represets evolution of the asset's health \cite{jennions_integrated_2011}}
	\label{fig:p-f-curve}
\end{figure}

The P-F curve is a widespread representation of equipment degradation over time based on historical records (Fig.~\ref{fig:p-f-curve}). Corrective action should be taken between the event of potential failure (P), when the fault detection is activated, and functional failure point (F) in the P-F interval~\cite{bousdekis_enterprise_2021}.  These division points are not exactly set but have statistical distribution to them.

The Remaining Useful Life (RUL) of the specific running machine in the given instance can be merely estimated analytically, with the survival probabilities of the individual components, and based on the model of the `run-to-failure' histories and usage parameters~\cite{okoh_overview_2014}. Predictive condition monitoring aims to extend the machine lifespan to the maximum by predicting expected RUL.

\begin{figure}[h]
	\centering
	\includegraphics[width=0.8\textwidth]{assets/bath-tub-curve.png}
	\caption{Bath tub curve~\cite{mohanty_machinery_2015}}
	\label{fig:bath-tub-curve}
\end{figure}

High failure rate is present not only at the worn out stage when the parts are fatigued or corroded, but also in the early stages soon after assembly. Causes can be found in the manufacturing or material defects, inadequate installation, or improper start-up procedures. During the stable middle phase the malfunction can occur after machine excessive overload. The time plot to failure rate is known as the bath tub curve (Fig.~\ref{fig:bath-tub-curve}).

\subsection{Vibration fault types}
Mechanical problems during machinery operation bring about vibrations in the vast majority of cases. The cause of vibration comes out of the changing force in its magnitude or direction. The most emerging defects can be encompassed by explaining the deficiencies of the mechanical structure broadly categorized as \textbf{unbalance, misalignment, looseness, excentricity, and influence of the external force}~\cite{davies_handbook_2012}. 

However, rotating machine disorders do exhibit frequency signatures at various ranges in the frequency spectrum. Imbalance, misalignment, and looseness normally appear at frequencies up to 300 Hz. These low-frequency faults are associated with the movement of the shaft and primarily coincide with revolution speed and its harmonics. Bearing and gearbox defects in the late stages of development, show up in the range between 300 Hz to 1 kHz. Higher frequencies measured traditionally to a limit of 10 kHz help notice the flaw of bearings even sooner~\cite{torres_automatic_2022}.


\begin{itemize}
\item predominant frequency
\item synchronous frequency
\item subsynchronous frequency
\item fundamental frequency
\item harmonic frequency
\item subharmonic frequency
\end{itemize}
% p.294 Davies

\cite{scheffer_practical_2004} p.98 - 141

\cite{davies_handbook_2012} p.282
% Automatic Anomaly Detection in Vibration Analysis Based on Machine Learning Algorithms
\cite{torres_automatic_2022}
% Vibration Guide
\cite{noauthor_vibration_2000}
% The experimental application of popular machine learning algorithms on predictive maintenance and the design of IIoT based condition monitoring system
\cite{cakir_experimental_2021}
% Technická diagnostika
\cite{ziaran_technicka_2013}
 
%TODO
Why monitor with vibrations,
- Resonance frequencies of each part - machine must run at speeds not aligned with resonance frequencies - Campbell diagram - task for mechanical engineers
- Faults - reasons and frequency content

% Describe by frequency responses and assign them possibale faults instead

\begin{itemize}
\itemsep0pt
\item Synchrounous response - based on RPM
\item Mass unbalance
\item Misalignment
\item Eccentricity
\item Bent or bow shaft
\item Cracked shaft
\item Rotor rubs - friction
\item Looseness
\item Auxiliery mechanical systems: Gearbox, Bearings, Belt 
\end{itemize}

% Bandsaws
% Vibration of bandsaws
\cite{lengoc_vibration_1990}
% Study on Online Detection and Fault Diagnosis of Band Saw Equipment
\cite{chen_study_2014}

\subsection{Technical standards}
Vibration-based condition monitoring practices adopted in the factory's predictive maintenance management must comply with normative guidelines formalized in ISO international standards. The standards are concerned with each step in the process that originates with transducer placements and data acquisition. They prescribe conventions for setting fault severity levels and provide empirically observed vibration characteristics of common defects. Two relevant standards for IoT diagnostics systems are \emph{ISO 20816} (updated from ISO 10816) and \emph{ISO 13373}.
\bigbreak

\textbf{ISO 20816-1:2016} establishes the approach to vibration measurement and evaluation on non-rotating housing of machinery parts~\cite{noauthor_iso_2016}. The measurement units are agreed upon for kinematic quantities of vibrations. Acceleration is to be measured in meters per second squared ($m/s^2$), velocity in millimeters per second ($mm/s$), and displacement in micrometers ($\mu m$). It is customary to evaluate broad-band vibration velocity in terms of root mean square value (RMS), as it has a relation to its signal energy. No simple direct relationship is expressible among these quantities, except in stationary signals.

The vibration severity is the maximum magnitude value measured in two radial directions (horizontal, and vertical) or supplemented with a third direction along the shaft in the axial axis. Multiple measurement locations, i.e. on several bearings or couplings, should be assessed independently. 

Criteria introduced to judge vibration severity are its absolute vibration magnitude, change in the magnitude vector, and rate of change. In terms of maximal magnitudes the machines of varied sizes are split into four severity zones defined in the chart ~(Tab.~\ref{tab:iso20816-vibration-severity}). The table values serve as guidelines towards realistic requirements between machinemanufacturers and customers.

\begin{table}[h]
\renewcommand{\arraystretch}{1.2}
\begin{adjustbox}{width=\columnwidth,center}
\begin{tabular}{|c|c|c|c|c|}
\hline
\textbf{\begin{tabular}[c]{@{}c@{}}Vibration velocity\\ RMS {[}mm/s{]}\end{tabular}} & \textbf{\begin{tabular}[c]{@{}c@{}}Class I\\ Small machines\end{tabular}} & \textbf{\begin{tabular}[c]{@{}c@{}}Class II\\ Medium machines\end{tabular}} & \textbf{\begin{tabular}[c]{@{}c@{}}Class III\\ Large machines\\ Rigid supports\end{tabular}} & \textbf{\begin{tabular}[c]{@{}c@{}}Class IV\\ Large machines\\ Flexible support\end{tabular}} \\ \hline
0.28                                                                                 & \cellcolor[HTML]{9AFF99}                                                  & \cellcolor[HTML]{9AFF99}                                                    & \cellcolor[HTML]{9AFF99}                                                                     & \cellcolor[HTML]{9AFF99}                                                                      \\ \cline{1-1}
0.45                                                                                 & \cellcolor[HTML]{9AFF99}                                                  & \cellcolor[HTML]{9AFF99}                                                    & \cellcolor[HTML]{9AFF99}                                                                     & \cellcolor[HTML]{9AFF99}                                                                      \\ \cline{1-1}
0.71                                                                                 & \multirow{-3}{*}{\cellcolor[HTML]{9AFF99}\textbf{Good (A)}}               & \cellcolor[HTML]{9AFF99}                                                    & \cellcolor[HTML]{9AFF99}                                                                     & \cellcolor[HTML]{9AFF99}                                                                      \\ \cline{1-2}
1.12                                                                                 & \cellcolor[HTML]{FFFC9E}                                                  & \multirow{-4}{*}{\cellcolor[HTML]{9AFF99}\textbf{Good (A)}}                 & \cellcolor[HTML]{9AFF99}                                                                     & \cellcolor[HTML]{9AFF99}                                                                      \\ \cline{1-1} \cline{3-3}
1.8                                                                                  & \multirow{-2}{*}{\cellcolor[HTML]{FFFC9E}\textbf{Satisfactory (B)}}       & \cellcolor[HTML]{FFFC9E}                                                    & \multirow{-5}{*}{\cellcolor[HTML]{9AFF99}\textbf{Good (A)}}                                  & \cellcolor[HTML]{9AFF99}                                                                      \\ \cline{1-2} \cline{4-4}
2.8                                                                                  & \cellcolor[HTML]{F8A102}                                                  & \multirow{-2}{*}{\cellcolor[HTML]{FFFC9E}\textbf{Satisfactory (B)}}         & \cellcolor[HTML]{FFFC9E}                                                                     & \multirow{-6}{*}{\cellcolor[HTML]{9AFF99}\textbf{Good (A)}}                                   \\ \cline{1-1} \cline{3-3} \cline{5-5} 
4.5                                                                                  & \multirow{-2}{*}{\cellcolor[HTML]{F8A102}\textbf{Unsatisfactory (C)}}     & \cellcolor[HTML]{F8A102}                                                    & \multirow{-2}{*}{\cellcolor[HTML]{FFFC9E}\textbf{Satisfactory (B)}}                          & \cellcolor[HTML]{FFFC9E}                                                                      \\ \cline{1-2} \cline{4-4}
7.1                                                                                  & \cellcolor[HTML]{FD6864}                                                  & \multirow{-2}{*}{\cellcolor[HTML]{F8A102}\textbf{Unsatisfactory (C)}}       & \cellcolor[HTML]{F8A102}                                                                     & \multirow{-2}{*}{\cellcolor[HTML]{FFFC9E}\textbf{Satisfactory (B)}}                           \\ \cline{1-1} \cline{3-3} \cline{5-5} 
11.2                                                                                 & \cellcolor[HTML]{FD6864}                                                  & \cellcolor[HTML]{FD6864}                                                    & \multirow{-2}{*}{\cellcolor[HTML]{F8A102}\textbf{Unsatisfactory (C)}}                        & \cellcolor[HTML]{F8A102}                                                                      \\ \cline{1-1} \cline{4-4}
18                                                                                   & \cellcolor[HTML]{FD6864}                                                  & \cellcolor[HTML]{FD6864}                                                    & \cellcolor[HTML]{FD6864}                                                                     & \multirow{-2}{*}{\cellcolor[HTML]{F8A102}\textbf{Unsatisfactory (C)}}                         \\ \cline{1-1} \cline{5-5} 
28                                                                                   & \cellcolor[HTML]{FD6864}                                                  & \cellcolor[HTML]{FD6864}                                                    & \cellcolor[HTML]{FD6864}                                                                     & \cellcolor[HTML]{FD6864}                                                                      \\ \cline{1-1}
45                                                                                   & \multirow{-5}{*}{\cellcolor[HTML]{FD6864}\textbf{Unacceptable (D)}}       & \multirow{-4}{*}{\cellcolor[HTML]{FD6864}\textbf{Unacceptable (D)}}         & \multirow{-3}{*}{\cellcolor[HTML]{FD6864}\textbf{Unacceptable (D)}}                          & \multirow{-2}{*}{\cellcolor[HTML]{FD6864}\textbf{Unacceptable (D)}}                           \\ \hline
\end{tabular}
\end{adjustbox}
\caption{ISO 20816 vibration serverity chart \cite{noauthor_iso_2016}}
\label{tab:iso20816-vibration-severity}
\end{table}

\emph{Zone A} is reserved for newly commissioned machines. \emph{Zone B} signifies suitability for long-term operation. In \emph{zone C} is the machine deemed in unsatisfactory condition and corrective action should be taken soon. Finally, in zone D vibrations can cause damage to the machine. The span of acceptable values differs on machine class from I through to IV with an output power of 15 kW (class I), 75 kW (class II), 10 MW (class III), or greater.

The operational limits in the form of \emph{alarms} and \emph{trips} are usually established on the zone boundaries or close to them. Alarms are placed between zones B and C and provide a warning about reaching the threshold significant for noticeable change. Trips in between zones C and D urge immediate action or machine shut down.

\textbf{ISO 13373-1:2002} delves into further nuances of vibration monitoring and expands on procedures outlined in the vocabulary of ISO 20186. According to the standard, the data collection operates in continuous or periodic observation modes which follow an event or intervals. Both designs can be permanently mounted, but in continuous, collection `multiplexing rate is sufficiently rapid so there is no significant data or trends lost'~\cite{noauthor_iso_2002}.  When channels are scanned successively with gaps between data points the system is known as `scanning'. 

Measurement of vibrations should be accompanied by a description of the machine and its operating conditions. The machine description includes the machine identifier and its type, power source, rated rotation speed and power, configuration (shaft or belt driven), and machine support. Measurement parameters such as timestamp, transducer type, sensor location and orientation in MIMOSA code, measurement units and units qualifier (p-p, RMS), and other processing options (filters, number of averages, etc.) are to be recorded alongside the measurement value itself~\cite{noauthor_iso_2002}.
  
The transducer of choice for condition monitoring is the accelerometer which can provide the acceleration value of the body and velocity after signal integration. However, standard cautions against double integrating for displacement. The recommended frequency range for an accelerometer is 0.1 Hz to 30 kHz. The choice of transducer mount significantly lowers its resonance frequency which is least influenced by stud mount and stiff cement mount. The resonance is reduced to around 8 kHz with the use of soft epoxy or permanent magnet.

\begin{figure}[h]
	\centering
	\includegraphics[width=0.8\textwidth]{assets/transducer-response.png}
	\caption{The transducer linear response and resonance in tolerance intervals~\cite{noauthor_iso_2002}}
	\label{fig:tranducer-response}
\end{figure}

Broadband measurement requires `frequency ranges of 0.2 times the lowest rotational frequency to the highest frequency of interest' \cite{noauthor_iso_2002}, not exceeding 10 kHz, with RMS velocity 0.1 - 100 mm/s. Bearings and gears diagnosis may push the upper-frequency limit even higher. The tolerances of amplitude and frequency calibrations fall into two types with allowable tolerances of $\pm 5 \%$ or $\pm 10 \%$ (Fig.~\ref{fig:tranducer-response}). 



 
% Different under load 
% Baseline measurement - what are those p.28
% Monitoring programme p.12 (vibrations differ under load)
% Later: Potencial causes for faults (p. 45) - use in vibration fault types

\begin{enumerate}
\itemsep0pt
\item Review machinery history and establish failure modes.
\item When vibration monitoring is not applicable check for other condition monitoring techniques or use preventive maintance routine.
\item Select monitoring points
\item Take preliminary vibration measurements
\item Select vibration monitoring techniques: broadband, frequency analysis or special techniques. Set parameters of measurements units.
\item Take baseline measurements
\item Change levels that would warrant investigation
\item Carry out routine condition monitoring
\item If alarm was exceeded notify approriate personnel, review data and trends, perform diagnostic evaluation, repair as necessary. In case new baseline is needed continue in step of taking baseline measurements.
\item Shut down machine when trip level is exceeded. Than proceed same as after alarm trigger.
\end{enumerate}

\newpage

 

\section{Feature engineering}
Large domain knowledge with compared to other areas of machine learning (mechanics - physics)
\subsection{Preprocessing}
\begin{itemize}
\item Detrending - DC removal filter
\item Time synchronous averaging
\end{itemize}

\subsection{Feature extraction}
% Feature Engineering for Machine Learning
\cite{zheng_feature_2018}
% Feature Engineering and Selection: A Practical Approach for Predictive Models
\cite{johnson_feature_2019}
% A New Statistical Features Based Approach for Bearing Fault Diagnosis Using Vibration Signals
\cite{altaf_new_2022}
% A Novel Online Machine Learning Approach for Real-Time Condition Monitoring of Rotating Machines
\cite{mostafavi_novel_2021}
% Fault Detection of Bearing: An Unsupervised Machine Learning Approach Exploiting Feature Extraction and Dimensionality Reduction
\cite{brito_fault_2021}
% A large set of audio features for sound description
\cite{peeters_large_2004}
% Research on online intelligent monitoring system of band saw blade wear status based on multi‑feature fusion of acoustic emission signals
\cite{zhuo_research_2022}  
% Early Detection of Imbalance in Load and Machine In Front Load Washing Machines by Monitoring Drum Movement
\cite{mohammadi_early_2020}
% A Data Mining based Approach for Electric Motor Anomaly Detection Applied on Vibration Data
\cite{egaji_data_2020}
% Condition Monitoring with Vibration Signals
\cite{nandi_condition_2019}

% Vibration Analysis for IoT Enabled Predictive Maintenance
\cite{jung_vibration_2017}

\paragraph{Statistical measures}
\begin{itemize}
\item Standard Deviation
\item Max. amplitude
\item RMS amplitude
\item Skewness
\item Kurtosis \\
---
\item Spectral centroid
\item RMS frequency
\item Root variance frequency
\item Spectral kurtosis / Fast kurtogram
\item Harmonics (peaks) 
	% Comparative Study of Various CFAR Algorithms for Non-Homogenous Environments
	\cite{hatem_comparative_2018}
	% Multi-Scale Peak and Trough Detection Optimised for Periodic and Quasi-Periodic Neuroscience Data
	\cite{bishop_multi-scale_2018}
	% Non-Parametric Local Maxima and Minima Finder with Filtering Techniques for Bioprocess
	\cite{adikaram_non-parametric_2016}
	% Identification of harmonics and sidebands in a finite set of spectral components
	\cite{gerber_identification_2013}
	% Evaluation of Threshold-Based Algorithms for Detection of Spectral Peaks in Audio
	\cite{nunes_evaluation_2007}
	% Statistical techniques to select detection thresholds for peak signals in ice-core data
	\cite{karlof_statistical_2005}
\item Spectral Envelope
\item Harmonic spectral deviation \\
---
\item Energy
\item Spectral negentropy
	% Spectral negentropy and kurtogram performance comparison for bearing fault diagnosis
	\cite{avoci_spectral_2020}
\item TKEO - Teager-Kaiser energy operator
	%Application of Teager–Kaiser Energy Operator in the Early Fault Diagnosis of Rolling Bearings
	\cite{shi_application_2022}
\end{itemize}

\paragraph{Signal decompositions - sparse approximations}
Matching pursuit algorithm optimalization problem
\cite{song_mfbd_2021}
\begin{itemize}
\item FFT - Short Time Fourier Transform with Hamming window and Welch averaging
	\cite{oulmane_automatic_2015}

\item CWT-SST - Synchrosqueezing Wavelet Transform (vs. Transient-extracting transform) 
	% The fast continuous wavelet transformation (fCWT) for real-time, high-quality, noise-resistant time–frequency analysis
	\cite{arts_fast_2022}
	% A Concentrated Time–Frequency Analysis Tool for Bearing Fault Diagnosis
	\cite{yu_concentrated_2020}
	% Applications of the synchrosqueezing transform in seismic time-frequency analysis
	\cite{herrera_applications_2014}
	
\item WPD - Wavelet Packet Decomposition  - to approximation and detail coef. (Fejer-Korovkin wavelet)  	
	% Wavelet Packet Feature Extraction for Vibration Monitoring		
	\cite{yen_wavelet_2000}
	% A wavelet approach to dimension reduction and classification of hyperspectral data
	\cite{wickmann_wavelet_2007}

\item EWT - Empirical Wavelet Transform - (Meyer wavelet)
	% On the computational complexity of the empirical mode decomposition algorithm
	\cite{wang_computational_2014}
	% Novel self-adaptive vibration signal analysis: Concealed component decomposition and its application in bearing fault diagnosis
	\cite{tiwari_novel_2021}
	% The MFBD: a novel weak features extraction method for rotating machinery
	\cite{song_mfbd_2021}
	% Fault Feature Extraction and Enhancement of Rolling Element Bearings Based on Maximum Correlated Kurtosis Deconvolution and Improved Empirical Wavelet Transform
	\cite{li_fault_2019}
	% An Improved Empirical Wavelet Transform for Noisy and Non-Stationary Signal Processing
	\cite{zhuang_improved_2020}
	% Time and frequency domain scanning fault diagnosis method based on spectral negentropy and its application
	\cite{yonggang_time_2020}
	% An Adaptive Spectrum Segmentation Method to Optimize Empirical Wavelet Transform for Rolling Bearings Fault Diagnosis
	\cite{xu_adaptive_2019}
	% Improved empirical wavelet transform (EWT) and its application in non‑stationary vibration signal of transformer
	\cite{ni_improved_2022}
\end{itemize}


\subsection{Feature transformation}
\begin{itemize}
\item Principal Component Analysis (PCA)
\item Log transformation (Box-Cox Transform) to normal distribution
\item Normalization (min-max, standardize) 
\end{itemize}

\subsection{Feature selection}
Filter method - SelectKBest  in evaluation phase
\begin{itemize}
\item Variance Threshold
\item Pearson correlation
\item ANOVA F-value
\item Mutal information
\item Fisher score
\item Spectral feature selection algorithm (SPEC)
\end{itemize} 

\section{Diagnostics techniques}
Idenification of faulty states in data streams in semi-supervised learning

% Analysis of different RNN autoencoder variants for time series classification and machine prognostics
\cite{yu_analysis_2021}

\subsection{Novelty detection}
\cite{gervasi_anomaly_2020}

\begin{itemize}
\item Local Outlier Factor, Local Correlation Integral (Anomaly score)
	% One-Class Classification with LOF and LOCI: An Empirical Comparison
	\cite{janssens_one-class_2007}
	% Designing a Streaming Algorithm for Outlier Detection in Data Mining - An Incremental Approach
	\cite{yu_designing_2020}

\item DenStream (Density based clustering - DBSCAN)
	% Density-Based Clustering over an Evolving Data Stream with Noise
	\cite{cao_density-based_2006}
	% A Modified Approach of OPTICS Algorithm for Data Streams
	\cite{shukla_modified_2017}
	% Data Clustering - Algorithms and Applications
	\cite{aggarwal_data_2014}
	% State-of-the-art on clustering data streams
	\cite{ghesmoune_state---art_2016}
	% Cluster-Reduce: Compressing Sketches for Distributed Data Streams
	\cite{zhao_cluster-reduce_2021}
	
\item Half-space Trees (Isolation forest)
	% Fast Anomaly Detection for Streaming Data
	\cite{tan_fast_2011}
	% Anomaly Detection for Data Streams Based on Isolation Forest using Scikit-multiflow
	\cite{gervasi_anomaly_2020}
	
\end{itemize}

\subsection{Classification}
\begin{itemize}
\item kNN + Metric Tree (M-Tree for neighbourhood queries) + Euclidian Mahalanobis distance /  RBF similarity

% Review of Artificial Intelligence-based Bearing Vibration Monitoring
\cite{sheng_review_2020}
% Semi-Supervised Learning on Data Streams via Temporal Label Propagation
\cite{wagner_semi-supervised_2018}
% Minimum covariance determinant and extensions
\cite{hubert_minimum_2018}
\end{itemize}

% Feature-based performance of SVM and KNN classifiers for diagnosis of rolling element bearing faults
\cite{jamil_feature-based_2021}
% Classification of washing machines vibration signals using discrete wavelet analysis for feature extraction
\cite{goumas_classification_2002}
% Semi-Supervised Learning
\cite{chapelle_semi-supervised_2006}
\cite{dobilas_semi-supervised_2022}

% A Novel Online Machine Learning Approach for Real-Time Condition Monitoring of Rotating Machines
\cite{maurya_condition-based_2021} 

\section{Evaluation Datasets}
(Pictures of machines)

\paragraph{MAFAULDA}
SpectraQuest's Machinery Fault Simulator (MFS) Alignment-Balance-Vibration (ABVT)
50 kHz, 5 sec. recordings, Imbalance, Horizontal/Vertical misalignment, Bearings (Overhang / Underhang) - Inner, Outer, Cage

\cite{noauthor_mafaulda_nodate}
%TODO - mark the equipmet parts
\begin{figure}[h]
\centering
\includegraphics[width=0.7\textwidth]{assets/mafaulda-simulator.jpg}
\caption{SpectraQuest's Machinery Fault Simulator}
\label{fig:mafaulda-simulator}
\end{figure}

\begin{figure}[h]
\centering
\includegraphics[width=0.7\textwidth]{assets/cwru-test-stand.jpg}
\caption{CWRU apparatus}
\label{fig:mafaulda-simulator}
\end{figure}

\begin{figure}[h]
\centering
\includegraphics[width=0.7\textwidth]{assets/rotating-shaft.jpg}
\caption{Rotating shaft dataset}
\label{fig:rotating-shaft}
\end{figure}


\paragraph{CWRU}
2 HP (1.492 kW) Reliance Electric motor
Bearings - Inner, Outer
12 kHz, 48 kHz
fan and drive end bearings
Fault diameters of 7 mils, 14 mils, 21 mils, 28 mils, and 40 mils (1 mil=0.001 inches) in diameter were introduced separately at the inner raceway, rolling element (i.e. ball) and outer raceway. 
Faulted bearings were reinstalled into the test motor and vibration data was recorded for motor loads of 0 to 3 horsepower (motor speeds of 1797 to 1720 RPM).

% Feature-based performance of SVM and KNN classifiers for diagnosis of rolling element bearing faults
\cite{jamil_feature-based_2021}
% Machine Learning-Based Unbalance Detection of a Rotating Shaft Using Vibration Data
\cite{mey_machine_2020}

\paragraph{Rotating Shaft}
Shaft -  unbalances of different sizes
4 kHz

\section{Sensor and microcontroller}
% Intelligent Sensor Networks: The Integration of Sensor Networks, Signal Processing and Machine Learning
\cite{hu_intelligent_2012} 


%\nocite{*}
{\small \printbibliography[heading=bibintoc, title={Literature}]}

%  Prílohy -----------------------------------------------------------------------
\addtocontents{toc}{\protect\setcounter{tocdepth}{0}}
\addtocontents{toc}{\cftpagenumbersoff{chapter}}
\let\svaddcontentsline\addcontentsline
\renewcommand\addcontentsline[3]{%
  \ifthenelse{\equal{#1}{lof}}{}%
  {\ifthenelse{\equal{#1}{lot}}{}{\svaddcontentsline{#1}{#2}{#3}}}}

\appendix
\titleformat{\chapter}{\normalfont\huge\bf}{Appendix \thechapter:}{1em}{}
\renewcommand{\chaptermark}[1]{\markboth{\MakeUppercase{Appendix \thechapter.\ #1}}{}}

\thispagestyle{empty}
\chapter{Resumé}
\pagenumbering{arabic}
\renewcommand*{\thepage}{A-\arabic{page}}

\section{Úvod}
Vzostup priemyslu 4.0 so sebou prináša väčšiu mieru automatizácie s cieľom dosiahnuť optimálne využitie dostupných zdrojov. Na základe nepretržitého sledovania opotrebenia zariadení v reálnom čase sa majú zabezpečiť nápravné opatrenia na opravu alebo výmenu súčiastok včas, v reakcii na trendové ukazovatele. 

Cieľom je zachovať požadovanú bezpečnosť a efektivitu výroby a zároveň predĺžiť životnosť rotujúcich komponentov. Proaktívna diagnostika porúch je nevyhnutná na začatie opráv bez nadbytočných nákladov. Vibrácie predstavujú nerušivý spôsob, ako zistiť a zaznamenať prípadne fatálne zlyhania hneď v zárodku. Hlavným problémom pri monitorovaní veľa strojov s vibráciami, je to, že vzniká množstvo záznamov, ktoré nie sú priamo užitočné pre operátora výrobnej linky. Väčšina signálov sa zobrazí maximálne raz, preto je zbytočné ich ukladať alebo prenášať vcelku. 

Zároveň na dosiahnutie maximálnej presnosti detekcie musí byť model strojového učenia trénovaný pre cieľové prostredie. Poruchy sú navyše pomerne zriedkavé udalosti, ktoré sa zvyčajne vyskytujú s odstupom niekoľkých mesiacov. Za týchto okolností je ťažké rýchlo získať dostatočne veľkú vzorku poruchových udalostí.

\section{Sledovanie prevádzkového stavu}
Existujú tri rôzne prístupy k údržbe strojov: reaktívny, preventívny a prediktívny.
Pri reaktívnej údržbe beží stroj až do úplného zlyhania a je prijateľná vtedy, keď je možná úplná a rýchla výmena pokazeného stroja za záložný. Preventívna údržba prebieha v pravidelných intervaloch odvodených od vopred určeného rozvrhu v alebo strednej doby medzi poruchami. Prediktívna údržba zlepšuje predvídateľnosť oproti reaktívnej údržbe a eliminuje plytvanie voči príliš obozretnej prevencii. Odstávka stroja je naplánovaná po zistení kritických hodnôt a po odhalení problematických komponentov.

Mechanické problémy počas prevádzky strojov spôsobujú v mnohých prípadoch vibrácie. Vibroakustická diagnostika sa preto považuje za jednu z najdôležitejších metód pri včasnej identifikácii porúch komponentov. Najbežnejšie sa vyskytujúcimi poruchami sú nevyváženosť, nesúososť, vôľa, excentricita, deformácia, trhlina a nadmerné trenie. 

Symptómy porúch rotačných strojov sa prejavujú rôznymi frekvenčnými pásmami, ale väčšina je závislá od rotačnej rýchlosti súčiastky. Nevyváženosť, nesúosovosť a vôľa sa bežne objavujú v frekvenciách do 300 Hz. Poruchy ložísk a prevodovky v neskorých štádiách vývoja sa prejavujú v rozsahu medzi 300 Hz a 1 kHz. Vyššie frekvencie do 10 kHz pomáhajú odhaliť poruchy ložísk v skorších štádiách rozvoja.

Postupy monitorovania stavu založené na vibráciách musia byť v súlade s normatívnymi smernicami ISO 20816 a ISO 13373. Normy sa týkajú umiestnenia meracích zariadení, zberu údajov, konvencií nastavenia úrovní závažnosti porúch. 


\section{Extrakcia a výber atribútov}
Prediktívna údržba má ideálne predpoklady na využitie extrakcie atribútov, pretože signál je zvyčajne stacionárny a trendové premenné v časovej a frekvenčnej oblasti vychádzajú z expertných znalostí v oblasti mechaniky. Výhody dodatočného úsilia v porovnaní so spracovaním pôvodných vzoriek spočívajú v dosahovaní lepšej presnosti klasifikácie, znížení výpočtovej záťaže a znížení potreby úložnej kapacity. Výber atribútov nie je samostatným krokom v procese strojového učenia, ale mal by sa vykonávať iteratívne na zlepšenie výsledného modelu.

Najrozšírenejšími používanými atribútmi sú štatistické miery centrálneho momentu: priemer, rozptyl, štandardná odchýlka, šikmosť a špicatosť. Charakteristiky amplitúdy zahŕňajú kvadratický priemer (rms), vzdialenosť špička-špička a maximum. Ostatné významné atribúty časovej oblasti sú odvodené ako pomery a sú nim: faktor výkyvu, faktor rozpätia, faktor impulzu a faktor tvaru.  

V spektrálnej oblasti môžeme získať obvyklé štatistické vlastnosti distribúcie, ktorými sú spektrálne ťažisko, šikmosť a špicatosť. Okrem toho sa extrahujú roll-on a roll-off, fundamentálna frekvencia, entropia, negentropia, vzájomná korelácia spektier, pomer signálu k šum, energia vo frekvenčných pásmach.

Atribúty neprispievajú k prediktívnej sile modelu s rovnakým podielom. Výber ich optimálnej podmnožiny je NP-ťažký kombinatorický problém. Kroky všeobecného postup pri výbere atribútov metódou filtrovania sú generovanie podmnožín, vyhodnotenie podmnožín, ukončovacie kritérium hľadania, a validácia.

Hodnotenie relevancie atribútov sú založené na skórovaní podobnosti s predikovanou premennou. Často používané spôsoby zoraďovania dôležitosti atribútov sú prah rozptylu, koeficienty korelácie, ANOVA F štatistika, a vzájomná informácia. Viaceré podmnožiny prediktorov produkovaných každou z výberových metrík môžu slúžiť na trénovanie viacerých variantov klasifikačného modelu. Množiny atribútov je možné kombinovať do súboru volebným systémom ako sú väčšinové hlasovanie alebo súčin poradí.

\section{Diagnostické prístupy}
Identifikácia porúch v rotujúcich strojoch je binárny alebo viactriedny klasifikačný problém, ktorý pracuje na princípe učenia čiastočne s učiteľom, pretože označenia pre degradované stavy stroja sú v praxi zriedkavé. Ciele automatizácie monitorovania možno rozdeliť na detekciu anomálií a rozpoznanie typu poruchy.

Detekcia anomálií, novostí alebo odľahlých hodnôt určuje, či sa prevádzkový stav stroja výrazne odchyľuje od normálu. Po upozornení môže zasiahnuť odborník a stroj diagnostikovať. Odľahlé hodnoty sú odvodzované na základe neparametrických štatistických modelov, zhlukovania podľa najbližších susedov a prístupov založených na izolácii anomálnych vzoriek. DenStream je algoritmus zhlukovania založený na hustote prispôsobený z DBSCAN na zhlukovanie prúdových dát do ľubovoľne tvarovaných skupín. Half-space strom predpokladá, že náhodné delenie v každej osi v priestore atribútov izoluje odľahlé hodnoty do samostatných oddielov skôr ako nedeviantné pozorovania. 

Presná viactriedna klasifikácia príčin porúch stroja podľa vopred známych charakteristík je oveľa náročnejšia úloha ako objavenie anomálií. Algoritmus k-najbližších susedov (k-NN) priradí pozorovanie triede, do ktorej patrí väčšina $k$ bodov v blízkom okolí podľa použitej miery vzdialenosti. Nachádza uplatnenie aj v učení čiastočne s učiteľom, pretože dokáže odvodiť označenia len zo znalosti niekoľkých anotácií.

Ďalším prístupom je online alebo postupné učenie, ktoré aktualizuje parametre modelu s každou novou prichádzajúcou udalosťou. Tento prístup je užitočný pri spracovaní veľkých dát, kedy celý súbor údajov nie je k dispozícii vopred alebo ho nemožno spracovať naraz z dôvodu pamäťových obmedzení.

\section{Výskumné otázky}
Cieľom tejto práce je poskytnúť odpovede na štyri výskumné otázky:
\begin{enumerate}
\itemsep0pt
\item Aké atribúty dokážeme extrahovať z vibračných signálov?
\item Akú úsporu dát dosiahneme výberom atribútov?
\item Aké budú presnosti diagnostiky porúch s rôznymi sadami atribútov?
\item Ako môžeme priebežne označovať poruchové stavy?
\end{enumerate}

\section{Návrh spracovania pre MaFaulDa}

\section{Zber vibrácií v priemysle}
Doteraz uplatnená metodika pre súbor údajov zaznamenaných v laboratóriu sa aplikuje na vibračných signáloch z priemyselného prostredia. Pri monitorovaní zužitkujeme mierne prispôsobený postup z noriem. Ten zahŕňa výber strojov určených na monitorovanie, identifikáciu pozícií na meranie podľa technických štandardov, predbežné merania a vývoj senzorovej jednotky.

Na zber údajov boli vyčlenené dva špirálové kompresory ako súčasť klimatizačných jednotiek pre dátové centrum a tri čerpadlá s troma elektrometrami v prečerpávacej stanici na pitnú vodu. Dlhodobejšie merania uskutočníme vlastným vnoreným systémom na báze vývojovej dosky ESP32-PoE-ISO so slotom na SD kartu. Ako senzor vibrácii použijeme MEMS akcelerometer IIS3DWB. Vyznačuje sa vysokou šírkou pásma až 6.3 kHz, nízkym šumom, a vysokou výstupným dátovým tokom 26.7 kHz cez SPI zbernicu.

\section{Vyhodnotenie presnosti diagnostiky}

\section{Rozbor dátovej sady z priemyslu}

\section{Záver}
V diplomovej práci sme sa zamerali na výber trendových ukazovateľov pre riešenie monitorovania prevádzkového stavu a odhaľovanie porúch z vibračných signálov.  Extrahované premenné pochádzajú hlavne z popisných štatistík, z článkov o spracovaní zvukových signálov a technických noriem vibrodiagnostiky.

Dosiahnuté stratové kompresné pomery pre MaFaulDa sú 2381:1 pre všetky atribúty a 25000:1 pre šesť atribútov. Výber atribútov metódou súčinu poradí zabezpečí väčšinou najlepšiu presnosť k-NN modelu oproti metrikám samostatne. Žiadny prístup však nedokázal nájsť trojicu prediktorov s presnosťou blízkou optimálnej, ktorá je až 98\%. Trénovanie k-NN na troch hlavných komponentoch prinieslo lepšiu presnosť ako výber atribútov. 

Model postupného učenia k-NN dosahuje prinajlepšom 90\% presnosť s okamžitou spätnou väzbou, 85\% so značkami oneskorenými o 250 pozorovaní a 82\% s iba 25\% anotovaného súboru údajov. Porovnateľný model trénovaný v dávkach dosahuje presnosť 98\%.  

\clearpage

\thispagestyle{empty}
\chapter{Plan of work}
\pagenumbering{arabic}
\renewcommand*{\thepage}{B-\arabic{page}}

\section{Winter semester}

\begin{table}[h!]
\def\arraystretch{1.25}
\begin{tabular}{|l|p{12cm}|}
\hline
\textbf{Period} & \textbf{Work}                                                                                                                                                                                                                         \\ \hline
\nth{1} week         & Consultation with the supervisor on directions of the future work based on literature review during previous semester.
\\ \hline
\nth{2} week         & Outline the key sections of the analysis part in the thesis.
\\ \hline
\nth{3} week         & Match supporting literature with analysis sections. Further invesigation on the feature engineering methodology in condition monitoring.
 \\ \hline
\nth{4} week         & Summarize notes from condition monitoring articles and videorecordings of tutorials and conferences.
 \\ \hline
\nth{5} week         & Research transformation of vibration signal to feature space using time-frequency, harmonic and energy statistical metrics. Progress report meeting with the supervisor.
 \\ \hline
\nth{6} week         & Find articles and take notes about unsupervised and semi-supervised techniques in streaming data for machinery diagnostics, in order to gather information about suitable features.
 \\ \hline
\nth{7} week         & TBD (Narrow down wide variety applicable methods for signal decomposition)
 \\ \hline
 \nth{8} week         & TBD (Write thesis section on condition monitoring and machinery fault types)
 \\ \hline

\end{tabular}
\end{table}

\clearpage
\newpage


\section{Summer semester}

\clearpage


% Ďalšie prílohy
% \input{chapters/appendix/B-technical-docs}
% Ak nechce vypísať čísla strán na konci prílohy: \cleardoublepage

\input{chapters/appendix/C-digital-medium}

\end{document}
