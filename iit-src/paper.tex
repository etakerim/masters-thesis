\documentclass{llncs}

\usepackage{graphicx}
\usepackage{caption}
%\usepackage[caption=false]{subfig}
\usepackage[english]{babel}

\begin{document}

% 4 pages (5 pages with sources)
\title{Feature selection in fault classification of rotating machinery}

\titlerunning{Features for machinery faults}

\author{
	Miroslav Hájek\thanks{Master study programme in field: Informatics
	Supervisor: Dr. Marcel Baláž, Institute of Computer Engineering and Applied Informatics, Faculty of Informatics and Information Technologies STU in Bratislava }
}

\institute{Faculty of Informatics and Information Technologies STU in Bratislava\\
\email{xhajekm@stuba.sk}}

\maketitle 

\begin{abstract}

what is reasearch about
what was shown in previous studies
new methods
findings

\keywords{keyword  \and keyword \and keyword \and keyword}
\end{abstract}


\section{Introduction}
- predictive maintance, research question

Related work (citations)

existence of standards
domain - time, frequency, time-frequency
feature extraction techniques
feature selection techniques
classification of faults
datasets: mafaulda, cwru, rotating shaft
Raw - KSB Cloud - BVS pumps
!!! CITE do not describe !!!

\section{Methodology}

Mafaulda
1. Signal processing (choose bearing A)
2. Labeling - 6 classes (Class balancing by oversampling)
3. Feature extraction - Time domain, Frequency domain - Min-max scaling before PCA
4. Feature selection - 
	PCA 2,3,4 - accuracy + loading plot (PC2)
	KNN all features- 5-fold cross validation - rôzne domény, rôzne k, rôzne počty features
	KNN exhaustive models - 
		2,3,4 features (check corr, f-stat, mi, rank product scores)
	 -> KNN classification(online and offiline)

Custom dataset
1. Choice of machines (pic measurement places - two images and dots)
2. HW sensor + ESP32 + FW 60 s. recor
3. How measure (plan)
4. Signal processing
4. Labeling  / Labeling (use machine - split by type) 
5. Same as Mafaulda

acc = prec micro = recall micro



\section{Results}


Figures:

Signal analysis
- 1ks obr. Mafaulda ABVT simulator
- 1ks plot (histogram of td signal)?
- 1ks plot (6x subplots) Mafaulda welch from each fault (1s, 2**14 window, hann window)  - largest severity - 2500 rpm
- 1ks plot (6x subplots) Custom dataset - each place in one day spectrum (5s segment)

Feature analysis:
Mafaulda (3) a Custom (4) 
- 1 ks table (how many faults have how many recordings)
- 1 ks plot (2 lines TD, FD) - numer of PC vs. explained variance
- 1 ks plot (2x subplots TD, FD) - loading plot (PC2)
- 1 ks (4 subplots) custom: all machines, pumps, compressors, motors


Classification accuracy (choices of k. and feat. count, 5-fold cross validation)

For mafaulda and custom (which classes - all or just one machine)
- 1 ks All features (2x subplots TD, FD)
	- Each subplot boxplot (k = 3,5,7)

All models (exhausive) - draw rank, corr, f-stat, mi as horizontal line
	- 3 ks plots (2, 3, 4 features)
		- Each plot 2 boxplot subplots (TD, FD) - k-neigh. vs. accuracy of all models


Compare accuracies of best models in each categories for given number of features and k:
- 1 ks plot - bar chart - color rainbow - one x (td), second x (fd)



\section{Summary}

\section{References}


%\begin{figure}[t]
%\centering
%\subfloat{
%	\includegraphics[width=0.45\textwidth]{app.png}
%}
%\subfloat{
%	\includegraphics[width=0.45\textwidth]{mri.png}
%}
%\caption{Patients list and MRI viewer}
%\end{figure}




\begin{thebibliography}{1}

\bibitem{example}
Author, F., Author, S.: Title of a proceedings paper. In: Editor,
F., Editor, S. (eds.) CONFERENCE 2016, LNCS, vol. 9999, pp. 1--13.
Springer, Heidelberg (2016). \doi{10.10007/1234567890}

\end{thebibliography}
\end{document}
