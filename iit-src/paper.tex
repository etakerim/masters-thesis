\documentclass{llncs}

\usepackage{graphicx}
\usepackage{caption}
%\usepackage[caption=false]{subfig}
\usepackage[english]{babel}

\begin{document}

\title{Feature selection in fault detection for predictive maintenance of rotating machinery}
%
\titlerunning{NeurAI}
% If the paper title is too long for the running head, you can set
% an abbreviated paper title here
%
\author{
	Miroslav Hájek\thanks{Master study programme in field: Informatics Supervisor: Dr. Marcel Baláž, Institute of Computer Engineering and Applied Informatics, Faculty of Informatics and Information Technologies STU in Bratislava }
}

\institute{Faculty of Informatics and Information Technologies STU in Bratislava\\
\email{xhajekm@stuba.sk}}

\maketitle 

\begin{abstract}

what is reasearch about
what was shown in previous studies
new methods
findings

\keywords{keyword  \and keyword \and keyword \and keyword}
\end{abstract}


\section{Introduction}


\section{Innovative approachs}


%\begin{figure}[t]
%\centering
%\subfloat{
%	\includegraphics[width=0.45\textwidth]{app.png}
%}
%\subfloat{
%	\includegraphics[width=0.45\textwidth]{mri.png}
%}
%\caption{Patients list and MRI viewer}
%\end{figure}

\section{Summary}


\begin{thebibliography}{1}

\bibitem{example}
Author, F., Author, S.: Title of a proceedings paper. In: Editor,
F., Editor, S. (eds.) CONFERENCE 2016, LNCS, vol. 9999, pp. 1--13.
Springer, Heidelberg (2016). \doi{10.10007/1234567890}

\end{thebibliography}
\end{document}
