\documentclass[12pt, a4paper, twoside, openright, english]{book}

\usepackage[english]{babel}
\usepackage[utf8]{inputenc}
\usepackage[T1]{fontenc}
\usepackage[top=2.5cm,
	bottom=3cm,
	right=3.2cm,
	left=3.2cm
]{geometry}

\usepackage{subcaption}
\usepackage{hyperref}
\usepackage{enumitem}
\usepackage{tabularx}
\usepackage{afterpage}
\usepackage{longtable}
\usepackage{multirow}
\usepackage{amsfonts}
\usepackage{amssymb}
\usepackage{listings}
\usepackage{titlesec}
\usepackage{setspace}
\usepackage{fancyhdr}
\usepackage{fancyvrb}
\usepackage[fleqn]{amsmath}
\usepackage{pdfpages}
\usepackage{nccmath}
\usepackage{csquotes}
\usepackage{diagbox}
\usepackage{ifthen}
\usepackage{algorithm}
\usepackage{algpseudocode}
\usepackage[super]{nth}

\usepackage[titles]{tocloft} % TOC header same as chapter title
%\usepackage{showframe}
\usepackage{titlesec}
\titlespacing*{\chapter}{0pt}{0pt}{40pt}

\usepackage{colortbl}
\usepackage{tabulary}
\usepackage{adjustbox}

\usepackage{lmodern}
\usepackage{fourier-orns}
\newlength\longest


\usepackage{nomencl}
\usepackage{makeidx}
\usepackage{expl3}
\usepackage{etoolbox}
%\preto\tabular{\shorthandoff{-}}

\usepackage[style=iso-numeric, backend=biber, url=false]{biblatex}
\renewcommand*{\bibfont}{\normalfont\small}
\addbibresource{literature.bib}

% Abbreviations
% \makenomenclature
% \renewcommand{\nomname}{List of Terminology and Abbreviations}

% Formulas
\newcommand{\listequationsname}{List of Equations}
\newlistof{myequations}{equ}{\listequationsname}
\newcommand{\myequations}[1]{
\addcontentsline{equ}{myequations}{\protect\numberline{\theequation}\quad #1}}

% Algorithms
\makeatletter
\renewcommand*{\ALG@name}{Algorithms}
%\renewcommand{\listalgorithmname}{Zoznam algoritmov}
\algrenewcommand\algorithmicrequire{\textbf{Input:}}
\algrenewcommand\algorithmicensure{\textbf{Output:}}
\makeatother

% Empty even pages at the end of chapter
\makeatletter
\renewcommand*{\cleardoublepage}{\clearpage\if@twoside \ifodd\c@page\else
\hbox{}%
\thispagestyle{empty}%
\newpage%
\if@twocolumn\hbox{}\newpage\fi\fi\fi}
\makeatother

% roman numerals
\makeatletter
\newcommand*{\rom}[1]{\expandafter\@slowromancap\romannumeral #1@}
\makeatother


% Číslo kapitoly na rovnakom riadku ako názov
\titleformat{\chapter}{\normalfont\huge\bf}{\thechapter}{1em}{}

\raggedbottom
\newcommand{\emptypage}{\newpage\thispagestyle{empty}\mbox{}\newpage}
\newcommand{\signaturespace}[2]{
  \begingroup
  \renewcommand{\arraystretch}{0}
  \begin{tabular}[t]{cc}
  \hspace*{0pt}
  \cleaders\hbox{\kern.6pt.\kern.6pt}\hskip#1\relax
  \hspace*{0pt}
  \\[0.5cm]
  #2
  \end{tabular}
  \endgroup
}

\pagestyle{fancy}
\fancyhf{}  % clear all header and footers
\fancyhead[LE]{\leftmark}
\fancyhead[RO]{\rightmark}
\fancyfoot[LE, RO]{\thepage}

\fancypagestyle{plain}{
  \fancyhf{}
  \renewcommand{\headrulewidth}{0pt}
  \fancyhf[lef,rof]{\thepage}
}

\setlength{\headheight}{16pt}

\renewcommand{\ttdefault}{pcr}
\lstdefinestyle{cstyle}{
    language=C,
	basicstyle=\linespread{1.1}\ttfamily\footnotesize,
    numbers=left,
    numberstyle=\tiny,
    frame=single,
    tabsize=4,
    captionpos=b,
    breaklines=true,
    texcl=true,
	numbersep=8pt,
	framexleftmargin=15pt,
	xleftmargin=5ex,
     xrightmargin=3.4pt,
	 morekeywords = {uint8_t,uint16_t,int16_t,uint32_t,int32_t,bool}
}
\lstdefinestyle{docs}{
    language=C,
	basicstyle=\linespread{1.1}\ttfamily\small\bfseries,
    tabsize=4,
    breaklines=true,
    belowskip=0pt
}
\renewcommand{\lstlistingname}{Source code}

\setstretch{1.5}
\newcommand{\University}[0] {Slovenská technická univerzita v Bratislave}
\newcommand{\UniversityEN}[0] {Slovak University of Technology in Bratislava}
\newcommand{\Faculty}[0] {Fakulta informatiky a informačných technológií}
\newcommand{\FacultyEN}[0] {Faculty of Informatics and Information Technologies}

\newcommand{\ThesisTitle}[0] {Priebežná správa o riešení DP2}
\newcommand{\ThesisTitleEN}[0] {Progress report in Master's thesis project \rom{2}}

\newcommand{\Thesis}[0] {Diplomová práca}
\newcommand{\ThesisEN}[0] {Master's Thesis}
\newcommand{\Title}[0] {Vibrodiagnostika strojov s~priemyselným internetom vecí}
\newcommand{\TitleEN}[0] {Machinery Vibrodiagnostics with~the~Industrial~Internet~of~Things}

\newcommand{\Author}[0] {Bc. Miroslav Hájek}
\newcommand{\Supervisor}[0] {Ing. Marcel Baláž, PhD.}
\newcommand{\DepartmentalAdvisor}[0] {Ing. Jakub Findura}
\newcommand{\Consultant}[0] {Ing. Lukáš Doubravský}

\newcommand{\AuthorEN}[0] {Miroslav Hájek}
\newcommand{\SupervisorEN}[0] {Dr. Marcel Baláž}
\newcommand{\DepartmentalAdvisorEN}[0] {Jakub Findura}
\newcommand{\ConsultantEN}[0] {Lukáš Doubravský}

\newcommand{\RegNo}[0] {FIIT-xxxx-xxxxxx}
\newcommand{\Date}[0] {Január 2024}
\newcommand{\DateEN}[0] {January 2024}
\newcommand{\SignDateEN}[0] {January 2024}
\newcommand{\StudyProgramme}[0] {Inteligentné softvérové systémy}
\newcommand{\StudyProgrammeEN}[0] {Intelligent Software Systems}
\newcommand{\StudyField}[0] {Informatika}
\newcommand{\StudyFieldEN}[0] {Computer Science}
\newcommand{\Institute}[0] {Institute of Computer Engineering and Applied Informatics}
\newcommand{\SignPlace}[0] {Bratislava, }

% Todo list
\newlist{todolist}{itemize}{2}
\setlist[todolist]{label=$\square$}

\begin{document}
\nomenclature{\textbf{Pojem}}{Vysvetlenie}
% Cover -------------------------------------------------------------
\thispagestyle{empty}
{\centering
	{\Large \UniversityEN}\par
	{\Large \FacultyEN}\par
	\vspace{\medskipamount}
	\RegNo
	\vfill
	\textbf{\Large \Author}\par
	\vspace{1.5\bigskipamount}
	\textbf{\LARGE \TitleEN}\par
	\vspace{1.5\bigskipamount}
	{\Large \ThesisEN}\par
	\vfill
}
\begin{flushleft}

{\large Thesis Supervisor: \Supervisor \\
\DateEN}
\end{flushleft}
\emptypage

%  Main part ------------------------------------------------------
\newgeometry{top=2.5cm, bottom=3cm, right=2.5cm, left=3.5cm}

% Title page
\pagenumbering{roman}
\thispagestyle{empty}
{\centering
	{\Large \UniversityEN}\par
	{\Large \FacultyEN}\par
	\vspace{\medskipamount}
	Reg. No. \RegNo
	\vfill
	\textbf{\Large \Author}\par
	\vspace{1.5\bigskipamount}
	\textbf{\LARGE \TitleEN}\par
	\vspace{1.5\bigskipamount}
	{\Large \ThesisEN}\par
	\vfill
}
\begin{flushleft}
\begin{longtable}[l]{ll}
Study programme: & \StudyProgrammeEN \\
Study field: & \StudyFieldEN \\
Training workplace: & \Institute\\
Thesis supervisor: & \Supervisor \\
Departmental advisor: & \DepartmentalAdvisor \\
Consultant: & \Consultant \\
\end{longtable}
\indent\DateEN
\end{flushleft}
\emptypage

% Thesis assignment
\thispagestyle{empty}
\includepdf[pages=-, scale=1]{chapters/assignment}
\emptypage

% Declaration of Honour
 % Declation of Honour
\thispagestyle{empty}
\vspace*{\fill}
\section*{Declaration of Honor}

I hereby declare on my honor that I wrote this thesis independently under the supervision of Dr. Marcel Baláž, after consultations and with use of cited literature.

\vspace{3\medskipamount}\noindent
\SignPlace \SignDateEN \hspace*{\fill} \signaturespace{5cm}{\Author} 

\emptypage

% TODO (DP2): Acknowledgments
% % Poďakovanie
\thispagestyle{empty}
\vspace*{\fill}

\begin{center}
\settowidth\longest{\itshape ---Wir müssen wissen.---}
\parbox{\longest}{
  \hrulefill\hspace{0.2cm} \decofourleft\decofourright \hspace{0.2cm} \hrulefill\par
  \raggedright{
  \itshape
  	Wir müssen wissen. \\ Wir werden wissen.\par
  }   
  \raggedleft{--- David Hilbert}\par
  \hrulefill\hspace{0.2cm} \decofourleft\decofourright\hspace{0.2cm} \hrulefill\par
}
\end{center}

\vspace*{\fill}
\section*{Poďakovanie}
{\linespread{1.0}\small Napriek tomu, že tento text je len kvapka v mori záverečných prác, pre mňa bol proces tvorby omnoho viac než zachytávajú slová na stránkach. Srdečná vďaka patrí školiteľovi Ing.~Marcelovi Balážovi,~PhD. a konzultantovi Ing.~Lukášovi Doubravskému z firmy R-DAS,~s.r.o. Obaja boli otvorení mojim všemožným nápadom a podporovali ma v ich realizácii aj cez mnohé ťažkosti. Za praktický expertný pohľad na vibrodiagnostiku vďačím prof. Ing.~Stanislavovi Žiaranovi,~CSc. a Ing.~Ondrejovi Chlebovi,~PhD. zo Strojníckej fakulty STU. Cením si ústretovosť Bratislavskej vodárenskej spoločnosti,~a.s. v sprístupnení čerpadiel na merania a súvisiacich podkladov, a to konkrétne Ing.~Peterovi Csókovi a Ing.~Petrovi Kmeťkovi. Rovnako ďakujem za ochotu firme VNET,~a.s., konkrétne Michalovi Országovi a Mgr.~Vladimírovi Kupčovi za odporúčania možných strojov na merania v dátových centrách a sprístupnenie kompresorov. Prácu by som rád venoval rodine a kolegom-kamarátom, ktorí stáli pri mne na ,,šialenej akademickej dráhe'' a podnetné diskusie s nimi prispeli aj k môjmu pohľadu na odborné problémy. Nuž a v konečnom dôsledku som mal niekedy len viac šťastia ako rozumu.}
\vspace{3cm}
\emptypage

% Annotations
\thispagestyle{empty}
\section*{Annotation}
\UniversityEN \\
\uppercase{\FacultyEN}
\vspace{-8pt}
{\setlength{\mathindent}{0cm}
\begin{align*}
&\text{Degree course:} && \text{\StudyProgrammeEN} \\
&\text{Author:} && \text{\Author} \\
&\text{\ThesisEN:} && \text{\TitleEN} \\
&\text{Supervisor:} && \text{\SupervisorEN} \\
&\text{\DateEN}
\end{align*}}

\emptypage 

\thispagestyle{empty}
\section*{Anotácia}
\University \\
\uppercase{\Faculty}
\vspace{-8pt}
{\setlength{\mathindent}{0cm}
\begin{align*}
&\text{Študijný program:} && \text{\StudyProgramme} \\
&\text{Autor:} && \text{\Author} \\
&\text{\Thesis:} && \text{\Title} \\
&\text{Vedúci diplomovej práce:} && \text{\Supervisor} \\
&\text{\Date}
\end{align*}}

\emptypage

% Contents
\pagestyle{empty}
\tableofcontents{}
\listoffigures
\listoftables
\listofmyequations
%{\let\clearpage\relax \printnomenclature}
\emptypage

\pagestyle{fancy}
% Chapters
\pagenumbering{arabic}

\chapter{Introduction}
Manufacturing is experiencing a shift in the traditional practices of asset operational status evaluation and utilization. The rise of Industry 4.0 means greater automation and robotization of the production halls to achieve optimal usage of available resources. The secondary aspect in the enterprises' endeavor, however not less important, is to keep track of the equipment wear and tear. The corrective action be it repair or replacement should be taken on time in response to the key indicators. 

The goal is to preserve required safety and production efficiency when extending the useful life of machine moving parts. In the factories and logistics where this sort of equipment is vital, there is a rising interest in the ability to monitor in real-time the health of the machines and to proactively diagnose the fault to repair it without adding unnecessary costs. 

Vibrations are the most nonintrusive way with which such faults can be sensed. The experts use it to distinguish faulty states and to identify the malfunction's root cause. In critical circumstances such as in the case of the large turbines in the power plants, the precautions leading to regular machinery check-ups are already in place. To reach wider acceptance and spread, the monitoring solution has to be sufficiently independent, reliable, and as self-sufficient as the model design allows it to be.

The main issue to consider in large-scale machinery monitoring using vibrations are lots of uninformative streams of samples not directly useful for the production line operator. The dashboard must aggregate these flows into trend variables with severity levels categorized based on industrial standards. The majority of signals are viewed once at the maximum therefore to store or even transmit them from the edge device in its entirety would be wasteful. The complex overview of the mechanical equipment status is attainable only when agent devices and sensors are cheap enough with a long lifespan on battery power and preferably remain physically small to reduce the additional clutter.

Attempted machine and deep learning approaches have the crucial impediment that the construction of every single machine is unique to some extent because of tolerances and variable load. The model must be trained specifically for the target environment to achieve the ideal performance. In addition, the failures are relatively rare events occurring usually in the span of multiple months. In these circumstances, it is hard to obtain a large enough sample of fault events quickly. Novelty detection is a technique that can be applied in this case.

The thesis is organized in the following manner. In the first chapter of analysis in section 1 we explore the mechanical maintenance approaches and industry standards on common fault identification. Then section 2 is all about measuring vibrations and transforming them into features meaningful in automatic fault pattern recognition. In section 3 we delve into modes of diagnosis based on reduced relevant indicators. Section 4 deals with evaluation datasets used to determine computational requirements
on IIoT infrastructure. Chapter 2 defines data format and proposes processing steps to diagnose the imminent failure and different fault types. The approach taken is evaluated and validated in Chapter 6. 
  

\chapter{Problem analysis}
In the problem analysis chapter we expore the feature extraction methods and machine learning algorithms for the fault diagnostics.
The basis we build upon is the domain knowledge of the mechanical engineers in vibration signal measurement and its evaluation.

\section{Condition monitoring} \label{section:condition-monitoring}
All rotating machinery eventually fails because of the long-term strain on the individual parts, inadequate workmanship, installation, or operational procedures. In the end, these factors cause the equipment not to fulfill its intended functionality. Many instrumentation methods are practiced to reveal evolving faults: vibration and acoustic noise monitoring, electric supply line measurements, thermography, oil and particle analysis, ultrasonic testing, etc. Vibration signals are the preferred tool for rotating machinery monitoring \cite{mohanty_machinery_2015}.

The defect needs to be repaired or replaced, preferably without significant production downtime, further damage to the other attached elements, or any endangerment of the responsible personnel. The maintenance strategies are chosen according to the machine's importance as a result of potential failure's impact on the system. The guide to set appropriate maintenance procedures as outlined in the IEC~60706-2 standard, and involves reliability-centered maintenance analysis~\cite{el-thalji_predictive_2019}.

\subsection{Maintenance strategies}
There are three different approaches to maintenance across the industry: \textbf{reactive, preventive, and predictive}~\cite{scheffer_practical_2004}. In general, the more sophisticated methods are beneficial in a high-stakes environment. The unexpected machine shutdown can have a negative economic impact on the enterprise, resulting in decreased product quality and demands for spare parts to be ready in the supply inventory at all times. In certain situations, it suffices to utilize a simpler maintenance program, but predictive maintenance gains attraction in Industry 4.0 to optimize usage of assets~\cite{cinar_machine_2020}.

\textbf{Reactive maintenance} allows machinery to run until a complete failure. This is the most inappropriate way to maintain the production line, but it is straightforward enough. It requires a large stock of replacement parts on-site and breakage inflicts a ``crisis management mode'' upon the plant \cite{scheffer_practical_2004}. On-demand repairs are justified when short downtime is acceptable, full and swift replacement of a broken machine with a backup is possible, or there is a negligible threat to the surrounding environment from failure~\cite{ziaran_technicka_2013}.

\textbf{Preventive maintenance} is performed before any issue is detected. Maintenance occurs at regular intervals derived from a predetermined period in the calendar or expected machine running time (MTTF - Mean Time To Failure). The schedule is crucial but can result in components being replaced in good condition, creating waste. The parts can occasionally stay in operation too long, and the machine breaks as a result. Conservative planning is usually the norm to keep the machines always in a perfect state, and therefore more frequent interventions are required~\cite{mohanty_machinery_2015}.

\textbf{Predictive maintenance} known as condition-based maintenance (CbM), improves the predictability of reactive maintenance and eliminates waste in overall resource utilization of cautious prevention. The machine downtime is scheduled after the detection of unhealthy trends in fault monitoring with sensors and the identification of troublesome components.

A measurable decrease in effectiveness allows us to order necessary parts in advance and organize repairs of several machines at a convenient time. The missed detection leads to increased costs compared to previous methods and raises the expectation that faults are distinguishable among themselves~\cite{davies_handbook_2012}.

\begin{figure}[ht]
	\centering
	\includegraphics[width=\textwidth]{assets/analysis/P-F-Curve.png}
	\caption{P-F curve represents the evolution of the asset's health~\cite{jennions_integrated_2011}}
	\label{fig:p-f-curve}
\end{figure}

The \emph{P-F curve} is a widespread representation of equipment degradation over time based on historical records (Fig.~\ref{fig:p-f-curve}). Corrective action should be taken between the event of potential failure (P), when the fault detection is activated, and functional failure point (F) in the P-F interval~\cite{bousdekis_enterprise_2021}.  These division points are not exactly set but have a statistical distribution to them.

The \emph{Remaining Useful Life} (RUL) of the specific running machine in the given instance can be merely estimated analytically, with the survival probabilities of the individual components, based on the model of the ``run-to-failure'' histories, and usage parameters~\cite{okoh_overview_2014}. Predictive condition monitoring aims to extend lifespan to the maximum by predicting expected RUL.

\begin{figure}[ht]
	\centering
	\includegraphics[width=0.6\textwidth]{assets/analysis/bath-tub-curve.png}
	\caption{Bathtub curve~\cite{mohanty_machinery_2015}}
	\label{fig:bathtub-curve}
\end{figure}

A high failure rate is present not only at the worn-out stage when the parts are fatigued or corroded, but also in the early stages soon after assembly. Manufacturing or material defects, inadequate installation, or improper start-up procedures, are all suspected causes. During the stable middle phase, malfunction can occur after the machine's excessive overload. The time plot to failure rate is known as the bathtub curve~(Fig.~\ref{fig:bathtub-curve}).

\subsection{Vibration fault types}
Mechanical problems during machinery operation bring about vibrations in many instances. Therefore, vibroacoustic diagnostics is considered as one of the most important methods in early component fault identification~\cite{ziaran_technicka_2013}.

The cause of vibration comes out of the changing force in its magnitude or direction. The most emerging defects can be encompassed by explaining the deficiencies of the mechanical structure. These defects are broadly categorized as \textbf{unbalance, misalignment, looseness, eccentricity, deformation, crack, and influence of the external force} (friction)~\cite{davies_handbook_2012}. It is important to stress that our concern is not the underlying deformities in mechanical parts, but the correct fault classification based on the signal waveform.

\begin{figure}[ht]
	\centering
	\includegraphics[width=0.6\textwidth]{assets/analysis/complex-vibrations.png}
	\caption{Complex machinery vibrations~\cite{davies_handbook_2012}}
	\label{fig:machinery-vibrations}
\end{figure}

Rotating machine disorders exhibit frequency signatures at various ranges in the frequency spectrum with supplementary symptoms carried in phase signal. Most of the occurring faults can be tied to the main rotational speed of the component under investigation (Fig.~\ref{fig:machinery-vibrations})~\cite{davies_handbook_2012}. Imbalance, misalignment, and looseness normally appear at frequencies up to 300 Hz. These low-frequency faults are associated with the movement of the shaft and primarily coincide with revolution speed and its harmonics. Bearing and gearbox defects in the late stages of development, show up in the range between 300 Hz and 1 kHz. Higher frequencies, measured traditionally to a limit of 10 kHz, help notice the faults in bearings even sooner~\cite{torres_automatic_2022}.

One of the methods vibration experts utilize in the identification of the damaged part, according to the frequency spectrum, is \textbf{order analysis}. The excessive peaks at harmonic frequencies are of interest. Harmonics are integer multiples of fundamental frequency (1x rpm)~(Tab.~\ref{tab:vibration-causes}):

\begin{table}[h]
\renewcommand{\arraystretch}{1.2}
\begin{adjustbox}{width=\columnwidth,center}
\begin{tabular}{|ll|l|l|}
\hline
\multicolumn{2}{|l|}{\textbf{Frequency content}}                            & \textbf{Likely reason}                                                                                                     & \textbf{Other causes}                                                                                                                          \\ \hline
\multicolumn{1}{|l|}{\multirow{4}{*}{Synchronous}} & 1 x rpm                & Imbalance                                                                                                                & \begin{tabular}[c]{@{}l@{}}Eccentric journals\\ Bent shaft / Misalignment\\    (high axial vibration)\\ Bad belt (if rpm of belt)\end{tabular} \\ \cline{2-4}
\multicolumn{1}{|l|}{}                             & 2 x rpm                & Looseness                                                                                                                & \begin{tabular}[c]{@{}l@{}}Misalignment \\    (high axial vibration)\\ Cracked rotor\\ Bad belt (if rpm of belt)\end{tabular}                  \\ \cline{2-4}
\multicolumn{1}{|l|}{}                             & 3 x rpm                & Misalignment                                                                                                             & and axial looseness                                                                                                                             \\ \cline{2-4}
\multicolumn{1}{|l|}{}                             & Many x rpm             & \begin{tabular}[c]{@{}l@{}}Bad gears\\ Severe looseness\end{tabular}                                                     & \begin{tabular}[c]{@{}l@{}}Gear teeth x rpm\\ Fan blade count x rpm\end{tabular}                                                               \\ \hline
\multicolumn{1}{|l|}{Sub-synchronous}                 & \textless 1 x rpm      & Oil whirl                                                                                                                & \begin{tabular}[c]{@{}l@{}}Bad drive belt\\ Background\\ Resonance\end{tabular}                                                                \\ \hline
\multicolumn{1}{|l|}{Non-synchronous}              & \multicolumn{1}{c|}{-} & \begin{tabular}[c]{@{}l@{}}Electrical problems (x 50 Hz)\\ Reciprocating forces\\ Aerodynamic forces \\ Bad antifriction bearings\end{tabular} & Rubbing \\ \hline
\end{tabular}
\end{adjustbox}
\caption{Expert observed likely vibration causes (based on~\cite{davies_handbook_2012,ziaran_technicka_2013,noauthor_iso_2002})}
\label{tab:vibration-causes}
\end{table}

Because of inherent tolerances in machine manufacturing and assembly, the rotational frequency always manifests itself, even in baseline signature~\cite{davies_handbook_2012, noauthor_iso_2002}. In the most likely scenario, some faults appear as compared to rotational frequency solely in \textbf{synchronous, subsynchronous, or non-synchronous components}. The defects can occur in a predictable combination of the ones mentioned. Other common patterns experts look for are modulation sidebands typical for bearings and gears extractable with cepstrum analysis~\cite{ziaran_technicka_2013}. Procedures relying on elimination narrow down unrelated causes effectively.

%TODO bearing image / equations

\subsection{Technical standards} \label{section:technical-standards}
Vibration-based condition monitoring practices adopted in the factory's predictive maintenance management must comply with normative guidelines formalized in ISO international standards. Standards are concerned with each step in the process that originates with transducer placements and data acquisition. They prescribe conventions for setting fault severity levels and provide empirically observed vibration characteristics of common defects. Relevant standards for IoT diagnostics systems are \emph{ISO 20816} (updated from ISO 10816) and \emph{ISO 13373}.
\bigbreak

\textbf{ISO 20816-1:2016} establishes the approaches of vibration measurement and evaluation on non-rotating housing of machinery parts~\cite{noauthor_iso_2016}. The measurement units are agreed upon for kinematic quantities of vibrations. Acceleration is measured in meters per second squared ($m/s^2$), velocity in millimeters per second ($mm/s$), and displacement in micrometers ($\mu m$). It is customary to evaluate broad-band vibration velocity in terms of root mean square value (rms), as it is related to signal energy. No simple direct relationship is expressible among these physical quantities, except in stationary signals.

The vibration severity is the maximum magnitude measured in two radial directions (horizontal, and vertical) or supplemented with a third direction along the shaft on the axial axis. Multiple measurement locations, i.e. on different bearings or couplings, should be assessed independently.

Criteria introduced to judge vibration severity are absolute vibration magnitude, change in the magnitude vector, and rate of change. In terms of maximal magnitudes, the machines of varying sizes are split into four severity zones defined in the chart ~(Tab.~\ref{tab:iso20816-vibration-severity}). The values in this table serve as guidelines towards realistic requirements between machine manufacturers and their customers.

\begin{table}[h]
\centering
\renewcommand{\arraystretch}{1.2}
\begin{adjustbox}{width=\columnwidth,center}
\begin{tabular}{|c|c|c|c|c|}
\hline
\textbf{\begin{tabular}[c]{@{}c@{}}Vibration velocity\\ rms {[}mm/s{]}\end{tabular}} & \textbf{\begin{tabular}[c]{@{}c@{}}Class I\\ Small machines\end{tabular}} & \textbf{\begin{tabular}[c]{@{}c@{}}Class II\\ Medium machines\end{tabular}} & \textbf{\begin{tabular}[c]{@{}c@{}}Class III\\ Large machines\\ Rigid supports\end{tabular}} & \textbf{\begin{tabular}[c]{@{}c@{}}Class IV\\ Large machines\\ Flexible support\end{tabular}} \\ \hline
0.28                                                                                 & \cellcolor[HTML]{9AFF99}                                                  & \cellcolor[HTML]{9AFF99}                                                    & \cellcolor[HTML]{9AFF99}                                                                     & \cellcolor[HTML]{9AFF99}                                                                      \\ \cline{1-1}
0.45                                                                                 & \cellcolor[HTML]{9AFF99}                                                  & \cellcolor[HTML]{9AFF99}                                                    & \cellcolor[HTML]{9AFF99}                                                                     & \cellcolor[HTML]{9AFF99}                                                                      \\ \cline{1-1}
0.71                                                                                 & \multirow{-3}{*}{\cellcolor[HTML]{9AFF99}\textbf{Good (A)}}               & \cellcolor[HTML]{9AFF99}                                                    & \cellcolor[HTML]{9AFF99}                                                                     & \cellcolor[HTML]{9AFF99}                                                                      \\ \cline{1-2}
1.12                                                                                 & \cellcolor[HTML]{FFFC9E}                                                  & \multirow{-4}{*}{\cellcolor[HTML]{9AFF99}\textbf{Good (A)}}                 & \cellcolor[HTML]{9AFF99}                                                                     & \cellcolor[HTML]{9AFF99}                                                                      \\ \cline{1-1} \cline{3-3}
1.8                                                                                  & \multirow{-2}{*}{\cellcolor[HTML]{FFFC9E}\textbf{Satisfactory (B)}}       & \cellcolor[HTML]{FFFC9E}                                                    & \multirow{-5}{*}{\cellcolor[HTML]{9AFF99}\textbf{Good (A)}}                                  & \cellcolor[HTML]{9AFF99}                                                                      \\ \cline{1-2} \cline{4-4}
2.8                                                                                  & \cellcolor[HTML]{F8A102}                                                  & \multirow{-2}{*}{\cellcolor[HTML]{FFFC9E}\textbf{Satisfactory (B)}}         & \cellcolor[HTML]{FFFC9E}                                                                     & \multirow{-6}{*}{\cellcolor[HTML]{9AFF99}\textbf{Good (A)}}                                   \\ \cline{1-1} \cline{3-3} \cline{5-5}
4.5                                                                                  & \multirow{-2}{*}{\cellcolor[HTML]{F8A102}\textbf{Unsatisfactory (C)}}     & \cellcolor[HTML]{F8A102}                                                    & \multirow{-2}{*}{\cellcolor[HTML]{FFFC9E}\textbf{Satisfactory (B)}}                          & \cellcolor[HTML]{FFFC9E}                                                                      \\ \cline{1-2} \cline{4-4}
7.1                                                                                  & \cellcolor[HTML]{FD6864}                                                  & \multirow{-2}{*}{\cellcolor[HTML]{F8A102}\textbf{Unsatisfactory (C)}}       & \cellcolor[HTML]{F8A102}                                                                     & \multirow{-2}{*}{\cellcolor[HTML]{FFFC9E}\textbf{Satisfactory (B)}}                           \\ \cline{1-1} \cline{3-3} \cline{5-5}
11.2                                                                                 & \cellcolor[HTML]{FD6864}                                                  & \cellcolor[HTML]{FD6864}                                                    & \multirow{-2}{*}{\cellcolor[HTML]{F8A102}\textbf{Unsatisfactory (C)}}                        & \cellcolor[HTML]{F8A102}                                                                      \\ \cline{1-1} \cline{4-4}
18                                                                                   & \cellcolor[HTML]{FD6864}                                                  & \cellcolor[HTML]{FD6864}                                                    & \cellcolor[HTML]{FD6864}                                                                     & \multirow{-2}{*}{\cellcolor[HTML]{F8A102}\textbf{Unsatisfactory (C)}}                         \\ \cline{1-1} \cline{5-5}
28                                                                                   & \cellcolor[HTML]{FD6864}                                                  & \cellcolor[HTML]{FD6864}                                                    & \cellcolor[HTML]{FD6864}                                                                     & \cellcolor[HTML]{FD6864}                                                                      \\ \cline{1-1}
45                                                                                   & \multirow{-5}{*}{\cellcolor[HTML]{FD6864}\textbf{Unacceptable (D)}}       & \multirow{-4}{*}{\cellcolor[HTML]{FD6864}\textbf{Unacceptable (D)}}         & \multirow{-3}{*}{\cellcolor[HTML]{FD6864}\textbf{Unacceptable (D)}}                          & \multirow{-2}{*}{\cellcolor[HTML]{FD6864}\textbf{Unacceptable (D)}}                           \\ \hline
\end{tabular}
\end{adjustbox}
\caption{ISO 20816 vibration severity chart with typical magnitudes \cite{noauthor_iso_2016}}
\label{tab:iso20816-vibration-severity}
\end{table}

Zone A is reserved for newly commissioned machines. Zone B signifies suitability for long-term operation. In zone C the machine is deemed in unsatisfactory condition and corrective action should be taken soon. Finally, in zone D vibrations can cause damage to the machine. The span of acceptable values differs with the machine class from \rom{1} through to \rom{4} and their output power of 15~kW (class~\rom{1}), 75~kW (class~\rom{2}), 10~MW (class~\rom{3}), or greater.

The operational limits in the form of \emph{alarms} and \emph{trips} are usually established on the zone boundaries or close to them. Alarms are placed between zones B and C providing a warning about reaching the threshold significant for noticeable change. Trips in between zones C and D urge immediate action or machine shut down. Both limits should not exceed 1.25 times the upper boundary of the lower zones and initially are set based on previous experience with the machine~\cite{noauthor_iso_2002}.

\textbf{ISO 13373-1:2002} delves into further nuances of vibration monitoring and expands on procedures outlined in terms of the vocabulary in ISO 20186. According to the standard, the data collection operates in continuous or periodic observation modes which follow an event or interval. Both designs can be permanently mounted. In continuous collection, it is recommended to have a ``multiplexing rate sufficiently rapid, so there is no significant data or trends lost''~\cite{noauthor_iso_2002}. When channels are scanned in succession with gaps between data points, the system is known as ``scanning''.

The condition monitoring program is run according to a flowchart adapted from one designed by the standard specifically to best benefit the plant. Those steps can be summarized as follows~\cite{noauthor_iso_2002}:

\begin{enumerate}
\itemsep0pt
\item Review machinery history and establish failure modes.
\item When vibration monitoring is not applicable, check for other condition-monitoring techniques or resort to preventive maintenance.
\item Select monitoring points and take preliminary vibration measurements.
\item Select vibration monitoring techniques: broadband, frequency analysis, or special techniques, and set parameters of measurement units.
\item Take baseline measurements.
\item Change levels that would warrant investigation.
\item Carry out routine condition monitoring.
\item If an alarm was exceeded, notify appropriate personnel to review data and trends, perform diagnostic evaluation, and repair as necessary. In case a new baseline is needed, continue in the step of taking baseline measurements.
\item Shut down the machine when the trip level is exceeded. Then proceed the same as after the alarm is triggered.
\end{enumerate}

Measurement of vibrations should be accompanied by a description of the machine and its operating conditions. The machine description includes its identifier and type, power source, rated rotation speed and power, configuration (shaft or belt driven), and machine support. Measurement parameters are to be recorded alongside the measurement value itself, such as timestamp, transducer type, sensor location and orientation in \textbf{MIMOSA convention} (Machinery Information Management Open System Alliance), measurement units and units qualifier (p-p, rms), and other processing options (filters, number of averages, etc.)~\cite{noauthor_iso_2002}.

The transducer of choice for condition monitoring is the accelerometer, which can provide the acceleration value of the body and velocity after signal integration. However, standard advise against double integrating for displacement. The recommended frequency range for an accelerometer is 0.1~Hz to 30~kHz. The choice of transducer mount significantly lowers its resonance frequency, which is least influenced by the stud mount and stiff cement mount. The resonance is reduced to around 8~kHz with the use of soft epoxy or permanent magnet.

\begin{figure}[h]
	\centering
	\includegraphics[width=0.8\textwidth]{assets/analysis/transducer-response.png}
	\caption{The transducer linear response and resonance in tolerance intervals~\cite{noauthor_iso_2002}}
	\label{fig:tranducer-response}
\end{figure}

Broadband measurement requires ``frequency ranges of 0.2 times the lowest rotational frequency to the highest frequency of interest''~\cite{noauthor_iso_2002}, not exceeding 10 kHz, with rms velocity 0.1 - 100~mm/s. Bearings and gear diagnosis may push the upper-frequency limit even higher. The tolerances of amplitude and frequency calibrations fall into two types with allowable tolerances of $\pm 5 \%$ or $\pm 10 \%$~(Fig.~\ref{fig:tranducer-response}).

Equipment's ``health'' can be mischaracterized when there are significant differences in the machine's normal operating conditions. Baseline measurements in all acceptable conditions are to be acquired to reduce the error in vibration evaluation. According to the bathtub curve~(Fig.~\ref{fig:bathtub-curve}) reference signatures should be obtained after the initial part wear-in period. The reference spectral mask of the baseline condition is designed if maximal acceptance amplitudes are different for each significant frequency band~\cite{ziaran_technicka_2013}.

The vibration baseline is defined by broad-band magnitudes and phases of motion vectors, the waveform in the time and frequency domain, the rotational speed of the machine as well as its frequency response to different speeds during start-up and coast-down captured in the Bode plot and waterfall plot. Changes during the machine's operation are then depicted in value trends. Trends can be shown by overall amplitudes or limited to frequency bands.
\section{Signal preprocessing} \label{section:signal-preprocessing}
The vibration signals in a factory environment are inherently full of disturbances. Nearby equipment operation and handling of heavy objects in the surroundings can all contribute to the unwanted chaotic movement in otherwise mostly pure oscillatory motion. In addition, accelerometers suffer from systematic measurement errors in the form of thermal noise, zero-g offset as a result of slight miscalibration, and bias originating from a constant force of gravity. These unavoidable distortions are suppressible to some extent with digital filters. In the preprocessing stage, we consider detrending, noise reduction with adaptive filters, and time synchronous averaging to eliminate the external interference.

\subsection{Detrending}
The oscillatory motion should be centered around the zero level for further manipulation. The constant offset is eliminated simply by subtracting the overall mean from the signal. Moreover high pass DC blocker infinite impulse response (IIR) filter of 1st order can adjust to shifts of the average value over time (Equation~\ref{equ:iir-dc-blocker}). The transition band depends upon the choice of corner frequency $f_{3dB}$ (Fig.~\ref{fig:dc-blocker}).

\myequations{DC blocker IIR filter of 1st order}
\begin{ceqn}\begin{align} \label{equ:iir-dc-blocker}
y_k = (1 - \frac{\omega}{2}) \cdot (x_k  -  x_{k - 1}) + (1 - \omega) \cdot y_{k - 1}; \quad \omega = 2\pi \cdot \frac{f_{3dB}}{f_s}
\end{align}\end{ceqn}

A steeper 3~dB attenuation band can be achieved by increasing the order of the filter. Then the cutoff frequency should be such that filter coefficients are fractions to counteract rounding errors~\cite{tittelbach-helmrich_digital_2021}.

\begin{figure}[h]
	\centering
	\includegraphics[width=0.7\textwidth]{assets/iir-1-dc-blocker-band.jpg}
	\caption{Transfer function of 1st order DC blocker filters ~\cite{tittelbach-helmrich_digital_2021}}
	\label{fig:dc-blocker}
\end{figure}

The finite impulse response (FIR) filter is not recommended for DC component removal because of the undesirable ripple effect with the small number of taps. Cascaded-integrator-comb (CIC) filters are proposed as an alternative instead~\cite{lyons_understanding_2011}.

\subsection{Adaptive noise cancellation}
Adaptive noise cancellation (ANC) involves an adaptive filter that self-adjusts coefficients through an update algorithm in response to the reference noise signal.  The objective of this filter is to minimize the mean square error (MSE) cost function in the error signal $e_k$ between signal contaminated with Gaussian noise $d_k$ and filter output $y_k$. Additive noise $n_k$ is assumed to be correlated with noise signal $\mathbf{X_k}$~\cite{diniz_adaptive_2020}.

Wiener-Hopf equations solve the optimal gradient of the MSE function and provide us with FIR filter coefficient vector $\mathbf{W_k}$.  The least mean squares (LMS) algorithm recursively approximates this analytical solution with the method of steepest descent (Equation~\ref{equ:lms-adaptive-filter})~\cite{diniz_adaptive_2020}. The multiple parameters are to be considered in the evaluation of filter performance: convergence rate, estimated error, and signal-to-noise ratio (SNR).

\myequations{Adaptive noise filtering LMS update formula}
\begin{ceqn}\begin{align} \label{equ:lms-adaptive-filter}
\mathbf{W}_{k+1} = \mathbf{W}_{k} + 2 \mu \mathbf{X}_{k}  e_k
\end{align}\end{ceqn}

The convergence stability is affected by step size $\mu$ which is bounded from above with the inverse of the maximal eigenvalue of input covariance matrix $\lambda_{max}$. The normalized least mean squares (NLMS) can handle input of varying scales  (Equation~\ref{equ:nlms-adaptive-filter}).

\myequations{Adaptive noise filtering NLMS update formula}
\begin{ceqn}\begin{align} \label{equ:nlms-adaptive-filter}
\mathbf{W}_{k+1} = \mathbf{W}_{k} + \frac{\mu}{\lVert\mathbf{X}_{k}\rVert^2} \mathbf{X}_{k}  e_k
\end{align}\end{ceqn}

\begin{figure}[h]
	\centering
	\includegraphics[width=0.8\textwidth]{assets/adaptive-filter.png}
	\caption{Adaptive noise cancellation filter diagram}
	\label{fig:adaptive-filter}
\end{figure}
\bigbreak

\subsection{Time synchronous averaging}
Time synchronous averaging (TSA) diminishes the impact of vibration sources unrelated to the rotational frequency and its harmonics. TSA averages time-domain waveform over $N$ points and aligns it to a synchronization pulse with period $T$ (Equation~\ref{equ:tsa-average}). This technique has been successfully applied to the gearbox and bearing fault diagnosis~\cite{davies_handbook_2012,nandi_condition_2019}

\myequations{Time synchronous averaging}
\begin{ceqn}\begin{align}
x_{TSA} = \frac{1}{N} \sum_{n = 0}^{N - 1}{x(t + nT)}
\label{equ:tsa-average}
\end{align}\end{ceqn}


\section{Feature engineering} \label{section:feature-engineering}
Raw numerical vectors after preprocessing are merely low-level descriptors of the underlying physical phenomena. At first, these incomprehensible sequences of numbers are reduced to summary attributes called features in the process of feature discovery. Normalization and linear transformations are applied here to discriminate different categories in the feature space. The most informative set of features is obtained with feature selection methods for the diagnostic model. Features can be hand-crafted, learned implicitly within the model representation, or explicitly from an optimization problem solution.  

\subsection{Feature extraction}
Predictive maintenance has ideal prerequisites for the application of feature engineering because the signal is usually pseudo-stationary, and the trend monitoring variables come out of extensive domain expertise in mechanics. The advantages of add-in extraction effort, as opposed to processing samples unmodified, are to gain better classification precision, reduce computational burden and storage capacity downstream with dimensionality reduction~\cite{johnson_feature_2019}. It is important to note that the design of features is not a standalone step in the machine learning pipeline but it should be performed iteratively to improve the target model.  Signal features are computed in the time, frequency, and time-frequency domain~\cite{brito_fault_2021}. 

\subsection{Time domain features}
The most widely found features in the literature are rudimentary statistical measures of the central moment: mean, variance, standard deviation, skewness, and kurtosis~(Tab.~\ref{tab:td-features}). Statistics can be calculated in any domain, but the mean value is not to be used in detrended data. The vibration severity metrics out of technical standards are also highly regarded. The characteristics of amplitude include root mean square, peak-to-peak, and maximum~\cite{mostafavi_novel_2021}.

The other significant time-domain attributes are derived as ratios of previous simpler ones. These ratios are crest factor, margin factor, impulse factor, and shape factor~(Tab.~\ref{tab:td-features})~\cite{nandi_condition_2019}. Many articles have been successful in bearing fault detection out of transients in impulsive signals with kurtosis, crest factor, and margin indicators \cite{brito_fault_2021}. It is also suggested that the shape factor can signify unbalance and misalignment faults~\cite{nandi_condition_2019}.

% TODO: add td and fd features from tsfel
\begin{table}[h]
\centering
\renewcommand{\arraystretch}{2}
\begin{adjustbox}{width=\columnwidth,center}
\begin{tabular}{|l|l|l|l|}
\hline
\textbf{Feature}            & \textbf{Equation}                                                                    & \textbf{Feature}        & \textbf{Equation}                                                                                                  \\ \hline
\textit{Standard deviation} & $ \sigma = \sqrt{\frac{1}{N}\sum_{i = 1}^{N}{\left(x_i - \bar{x}\right)^2}} $        & \textit{Crest factor}   & $ X_{cf} = \frac{max(|x_i|)}{X_{rms}} $                       \\ \hline
\textit{Skewness}           & $ X_{sv} = \frac{1}{N}\sum_{i = 1}^{N}{\left(\frac{x_i - \bar{x}}{\sigma}\right)^3}$ & \textit{Margin factor}  & $ X_{mf} = \frac{max(|x_i|)}{\left( \frac{1}{N} \sum_{i=1}^{N}{\sqrt{|x_i|}} \right)^2} $                          \\ \hline
\textit{Kurtosis}           & $ X_{kv} = \frac{1}{N}\sum_{i = 1}^{N}{\left(\frac{x_i - \bar{x}}{\sigma}\right)^4}$ & \textit{Impulse factor} & $ X_{if} = \frac{max(|x_i|)}{\frac{1}{N} \sum_{i=1}^{N}{|x_i|}} $                                                  \\ \hline
\textit{Root mean square}   & $ X_{rms} = \sqrt{\frac{1}{N}\sum_{i = 1}^{N}{x_i^2}} $             & \textit{Shape factor}   & $ X_{sf} = \frac{X_{rms}}{\frac{1}{N} \sum_{i=1}^{N}{|x_i|}}$ \\ \hline
\textit{Peak-to-peak}       & $ X_{ppv} = \max(x_i) - \min(x_i) $                                                  & \textit{Maximum}        & $ X_{max} = \max(|x_i|) $                                                                                          \\ \hline
\end{tabular}
\end{adjustbox}
\caption{Time domain features}
\label{tab:td-features}
\end{table}

\subsection{Frequency domain features}
The mechanical faults present themselves as oscillatory patterns which are combinations of frequencies with various amplitudes. The Fourier transform is one of the prominent strategies in power spectral density estimation. Experts on vibrodiagnostics utilize it as a primary signal processing technique for data analysis as it is recommended in ISO 13373-2 standard~\cite{noauthor_iso_2016_2}. 

The inherent symmetries in the Fourier matrix made it possible to invent an efficient implementation of the Fast Fourier transform (FFT) algorithm with time complexity $O(n \log n)$. The drawback of the plain spectral analysis is the lack of resolution for events that occurred at distant time instants, and so their spectral components might adversely blend in together. 

In the frequency domain, we can obtain spectral centroid, spectral kurtosis, spectral roll-off, spectral flux, energy in frequency bands, and energy ratio (Tab.~\ref{tab:fd-features})~\cite{peeters_large_2004}. In geometry terms, the spectral centroid represents the barycenter of the frequency magnitude plot. Spectral roll-off gives a notion about the spectral distribution because it searches the frequency $f_c$ below which 95\% of the signal energy is contained. The energy in the roll-off calculation is summed up to the Nyquist limit: $f_s / 2$. According to the definition, spectral flux is normalized cross-correlation between two successive amplitude spectra, value of one means spectra are the most dissimilar.

The harmonic frequencies may originate from higher order oscillation modes of the same shaft or other elastic rod supported on both ends. Several harmonics features worth investigating are fundamental frequencies, noisiness, inharmonicity, and harmonic spectral deviation~(Tab.~\ref{tab:fd-features}). Noisiness as a harmonic feature is a ratio of noise (non-harmonic components) energy to the total contained energy. Inharmonicity measures the divergence of spectral components from a purely harmonic to inharmonic signal on scale from 0 to 1. Harmonic spectral deviation adds up differences of amplitudes at harmonic peaks $a(h)$ from the spectral envelope $\mathrm{SE}(h)$~\cite{peeters_large_2004}. 

If a single principal frequency exists, it can be determined with maximum likelihood estimation. Such frequency would explain the signal spectrum the best~\cite{peeters_large_2004}.  The frequency spectrum is a discrete set of amplitudes where peaks have to be reliably identified to create representative attributes.
The essential peak-finding approaches are based either on magnitude or gradient. All found extrema are commonly filtered with the magnitude of prominences and the widths at half prominence. In the magnitude-based method, middle point $x_i$ is compared to neighboring two points and the peak is then: $x_{i-1} < x_i > x_{i+1}$. The gradient-based method evaluates the first derivative at the point which is equal to zero in case the point is a local maximum, local minimum, or inflection point~\cite{adikaram_non-parametric_2016}.

A substantial improvement is a robust non-parametric peak identification named \emph{MMS} based on the sum of terms in an arithmetic progression based on maximum, minimum, and sum. MMS max-min finder in the elementary form processes points in the window of length 3, it advances one point and deems its middle point as a local extremum if it satisfies equalities below. Equation~\ref{equ:mms-maxima} is for the hill and Equation~\ref{equ:mms-minima} is for the valley. The filtration techniques are incorporated in the adaptations of the MMS algorithm: MMS-WBF, MMS-SG, MMS-LH~\cite{adikaram_non-parametric_2016}.

\myequations{MMS algorithm: Local maxima identification equality}
\begin{ceqn}\begin{align}
\frac{a_{max} - a_{min}}{S_3 - a_{min} \cdot 3} = \frac{a_{mid} - a_{min}}{S_3 - a_{min} \cdot 3}
\label{equ:mms-maxima}
\end{align}\end{ceqn}

\myequations{MMS algorithm: Local minima identification equality}
\begin{ceqn}\begin{align}
\frac{a_{max} - a_{min}}{a_{max} \cdot 3 - S_3} = \frac{a_{max} - a_{mid}}{a_{max} \cdot 3 - S_3}
\label{equ:mms-minima}
 \end{align}\end{ceqn}

Multiple harmonic series and the sidebands can be separated into a discrete set of frequency components, each with central frequency, uncertainty, and amplitude $C_i(v_i, \Delta v_i, A_i)$ by an exhaustive search algorithm. Harmonic family identification is a non-trivial problem because of the spectrum estimation errors. The criterion is proposed to select harmonic at the minimal distance from the true fundamental frequency multiple~(Equation~\ref{equ:harmonic-search}). Two series with the same fundamental frequency are merged and thought of as a modulation series~\cite{gerber_identification_2013}.

\myequations{Harmonic series search criterion}
\begin{ceqn}\begin{align}
v_i^{(r)} = \frac{v_j}{\min{|v_j - r \cdot v_i|}}
\label{equ:harmonic-search}
\end{align}\end{ceqn}
 
\begin{table}[ht]
\renewcommand{\arraystretch}{2}
\centering
%\begin{adjustbox}{width=\columnwidth,center}
\begin{tabular}{|l|l|}
\hline
\textbf{Feature}           & \textbf{Equation}                                                                                                  \\ \hline
\textit{Spectral centroid} & $ X_{fc} = \frac{\sum_{i = 0}^{N - 1}{f_i \cdot a(f_i)}}{\sum_{i = 0}^{N - 1}{a(f_i)}}$                   \\ \hline 
\textit{Spectral roll-off} & $ E(f_c) = 0.95 \cdot E(f_s / 2) $                                                   \\ \hline
\textit{Spectral flux}     & $X_{\mathrm{flux}} = 1 - \mathrm{corr}(A_{t-1}, A_t)$ \\ \hline 
\textit{Harmonic spectral deviation} & $ X_{\mathrm{HDEV}} = \frac{1}{H}\sum_h(a(h) - \mathrm{SE}(h))$                                          \\ \hline
\textit{Noisiness}                   & $X_{noise} = \frac{\sum_{k \notin h}^{N}E_k}{\sum_{i = 1}^{N}E_i} = \mathrm{SNR}(X) = \mu \;/\; \sigma $                             \\ \hline
\textit{Inharmonicity}               & $X_{inharmo} = \frac{2}{f_0} \cdot \frac{\sum_h | f(h) - h \cdot f_0| \cdot a^2(h)}{\sum_h a^2(h)} $ \\ \hline
\textit{Energy}            & $ E(N) = \sum_{i = 1}^{N} a^2(t) $                                                                    \\ \hline \textit{Energy ratio}                & $E_r = E_i \;/\; \sum_{i = 1}^{N}{E_i} $                                                        \\ \hline
\textit{Shannon entropy}      & $H(X) = - \sum_{i} P(X = x_i) \cdot \ln(P(X = x_i)) $                                                  \\ \hline
\end{tabular}
%\end{adjustbox}
\caption{Frequency domain features}
\label{tab:fd-features}
\end{table}

\subsection{Time-frequency domain features}
The \textbf{Short-time Fourier transform} (STFT) splits the time-domain signal to an equal-length intervals. Individual chunks have 50\% overlap and are multiplied with weights of window function to balance scalloping loss and spectral leakage due to Fourier transform periodicity assumption. Traditionally, the Hann window is commonly used instead of rectangular window~\cite{ziaran_technicka_2013,noauthor_iso_2016_2}.

In the time-frequency domain, the same features can be derived as in the frequency domain, but in addition, the attributes are time-localized in this way. The STFT has a considerable flaw for implementation in a self-adaptable system and that is fixed resolution. The optimal window size has to be set beforehand or chosen after performing multiple transformations on chunks out of the range of lengths. Welch's methods averages multiple consecutive blocks to better estimate the spectrum. We have already researched the suitability of STFT for online detection of constant frequencies~\cite{hajek_iot_2022}.

The bands for lower frequencies should be longer in duration than for higher frequencies. The \textbf{Wavelet transform} (WT) possesses such multi-scale discrimination property effectively increasing resolution in time-frequency plain. Wavelet basis functions are constructed for that purpose (Equation~\ref{equ:mother-wavelet}). There are several wavelet families, for example, Haar, Daubechies, Coiflets, Symlets, Morlet, and Meyer~\cite{nandi_condition_2019}.


\myequations{Mother wavelet function}
\begin{ceqn}\begin{align}
\psi_{s, \tau} = \frac{1}{\sqrt{s}}\psi\left(\frac{t - \tau}{s}\right)
\label{equ:mother-wavelet}
\end{align}\end{ceqn}

\textbf{Continuous Wavelet transform} (CWT)~(Equation~\ref{equ:cwt-transform}) is performed by scaling and translating the mother wavelet $\psi$ picked out of the appropriate family~\cite{nandi_condition_2019}. The scale factor is denoted with~$s$ and time position with~$\tau$. The choice of wavelet type is data-driven because distinct wavelet shapes have an impact on the response and ultimately contribute to filter length. The decision lies between recognition abilities for impulse-like signals or the inclusion of wider surrounding space.

\myequations{Continuous Wavelet transform}
\begin{ceqn}\begin{align}
W_{x(t)}(s, \tau) = \frac{1}{\sqrt{s}}\int x(t) \cdot \psi^*\left(\frac{t - \tau}{s}\right)\;\mathrm{dt}
\label{equ:cwt-transform}
\end{align}\end{ceqn}

The CWT is computationally intensive when a highly detailed scale resolution is required because each wavelet scale convolves with the entire signal. The fCWT algorithm allows 100 times higher spectral resolution then previous implementation at the same speed. It increases performance 122 times compared to Wavelib and 34 times in comparison with PyWavelets~\cite{arts_fast_2022}. 

The fast CWT algorithm reaches compelling improvement by applying Parseval's theorem to the wavelet transform formula that removes the dependence on the time offset parameter~\cite{arts_fast_2022}. The convolution takes place with the mother wavelet in the Fourier base. Then, inverse FFT produces the coefficients for individual scales.

\textbf{Synchrosqueezing Wavelet transform} (SST) is a modification of CWT attempting to sharpen the representation of frequency components by coefficient reassignments from around the central frequencies towards the middle of the bands. The justification for these reallocations is rooted in the signal approximation as amplitude-modulated oscillating modes with additive noise $\eta(t)$ (Equation~\ref{equ:sst-transform})~\cite{herrera_applications_2014}.

\myequations{Representation of components for Synchrosqueezing Wavelet transform}
\begin{ceqn}\begin{align}
s(t) = \sum_{k=1}^{K} A_k(t)\cos(\theta_k(t)) + \eta(t)
\label{equ:sst-transform}
\end{align}\end{ceqn}

The components are defined by their instantaneous amplitudes $A_k(t)$ and instantaneous phases $\theta_k(t)$. The energy spread to adjacent bins can be effectively squeezed only in regions with constant phase and large enough component separation. Despite the promising properties of this transform, white noise causes severe interference in the resulting time-frequency map.

The alternative to isolating weak impacts with high time resolution is a \textbf{Teager-Kaiser energy operator} (TKEO) (Equation~ \ref{equ:tkeo-operator}). It is a tool for envelope analysis to demodulate characteristic AM-FM signal present during bearing faults. Energy operator output can be utilized as a standalone feature attribute.

\myequations{Teager–Kaiser energy operator}
\begin{ceqn}\begin{align}
\psi[x(n)] = [x(n)]^2 - x(n - 1)x(n + 1)
\label{equ:tkeo-operator}
\end{align}\end{ceqn}

Improved TKEO is necessary to prevent analysis from suffering from the noisy source. The key idea is to perform TKEO after signal decomposition into narrow-band components with different center frequencies. The extracted modes are reconstructed with weights assigned based on their correlations to the original signal~\cite{shi_application_2022}.

Time-frequency spectrum modification preserving localization of abrupt wide-band spikes at time $t_0$ and simultaneously reducing energy smear over the larger region is a goal of \textbf{Transient-extracting transform} (TET). Post-processing of STFT window $G(t,\omega)$ involves multiplication of the spectrum with the Transient extracting operator (TEO). This operator is expressed in the form of a Dirac delta function $\delta(t)$ (Equation~\ref{equ:tet-transform}).

\myequations{Transient-extracting transform}
\begin{ceqn}\begin{align}
\mathrm{Te}(t,\omega) = G(t,\omega) \cdot  \delta(t - t_0)
\label{equ:tet-transform}
\end{align}\end{ceqn}

TET representation retains non-zero coefficients where the absolute value of the ratio between two STFTs $G^{tg}[n, k]\;/\;G[n, k]$ is less than half the sampling interval $T$. These two transforms use distinct windows $g[n]$ and $n \cdot g[n]$. Decomposition of signal with TET is proved to produce significantly larger kurtosis (around 38 in TET, 4 in other methods) and hence better discriminate the transient fault~\cite{yu_concentrated_2020}.

The spectrograms to illustrate the difference in the ability of Fourier transform, continuous Wavelet transform, and their modifications to pick up underlying patterns in bearing faults are shown in the Fig.~\ref{fig:transforms}.

\begin{figure}[ht]
    \centering
    \begin{subfigure}[b]{0.49\textwidth}
        \includegraphics[width=\textwidth]{assets/stft-spectrogram-sample.png}
        \caption{STFT + Gaussian window}
    \end{subfigure}
    \hfill
    \begin{subfigure}[b]{0.49\textwidth}
        \includegraphics[width=\textwidth]{assets/tet-spectrogram-sample.png}
        \caption{TET}
    \end{subfigure}
    \hfill
    \begin{subfigure}[b]{0.49\textwidth}
        \includegraphics[width=\textwidth]{assets/wt-spectrogram-sample.png}
        \caption{CWT + Morlet wavelet}
    \end{subfigure}
    \hfill
    \begin{subfigure}[b]{0.49\textwidth}
        \includegraphics[width=\textwidth]{assets/sst-spectrogram-sample.png}
        \caption{SST}
    \end{subfigure}
    \caption{Comparison of time-frequency transform spectrograms~\cite{yu_concentrated_2020}}
    \label{fig:transforms}
\end{figure}

The dyadic filter bank is another signal decomposition technique that generates subbands on multiple granularity levels. The practical realization of the multi-scale description is \textbf{Discrete Wavelet Transform} (DWT). The DWT behaves as a quadrature mirror filter and splits waveform using a wavelet filter to detail coefficients (D1) and approximation coefficients (A1) (Fig.~\ref{fig:dwt-filter-bank})~\cite{nandi_condition_2019}. Low-pass filter $h(k)$ creates approximation coefficients further decomposed in the successive levels. Detail coefficients represent the result of the high-pass filter $g(k) = (-1)^k h(1 - k)$ after decimation by the factor of 2. 

The maximum depth of the decomposition tree is $\log_2{n}$ where $n$ is the number of input samples. Energy, energy ratio, and entropy are prevalent features that succinctly encode the wavelet coefficients. Otherwise, the additional extracted levels raise the total number of data dimensions.

In washing machine status classification, the discrete wavelet transform with Daubechies wavelet (db4) and fifth-level decomposition provided features combined from approximation (cA5) and detail coefficients (cD1, \dots, cD5). Washing machines belonged to three categories: no fault, electric motor clamping screws problem, and a loose or broken counterweight. Extracted measures were sample mean and sample variances over autocorrelation functions of coefficients (AcD$n$) and smoothed coefficients cD1, cD2 by moving average filter~\cite{goumas_classification_2002}.

\textbf{Wavelet Packet Decomposition} (WPD) applies filters to split detail coefficients identically as approximation ones (Fig.~\ref{fig:wpd-filter-bank}) thus increasing the resolution in the high-frequency bands and providing uniform spectrum partitioning. 

\myequations{Wavelet packet coefficient}
\begin{ceqn}\begin{align}
 w_{j,n,k} = \langle f, W_{j,k}^n\rangle = \langle f, 2^{j/2} W^n (2^jt-k) \rangle
\label{equ:wavelet-packet-coefficient}
\end{align}\end{ceqn}

Each wavelet packet coefficient $w_{j,n,k}$ captures subband frequency content around time instant $2^j k$ (Equation~\ref{equ:wavelet-packet-coefficient})~\cite{yen_wavelet_2000}. This measure is an inner product of the source and scaled wavelet packet function. The aforementioned feature extraction established in DWT can be applied, for example calculation of the wavelet packet node energy.

\begin{figure}[ht]
    \centering
    \begin{subfigure}[b]{0.49\textwidth}
        \includegraphics[width=\textwidth]{assets/DWT.png}
        \caption{Discrete wavelet transform}
        \label{fig:dwt-filter-bank}
    \end{subfigure}
    \hfill
    \begin{subfigure}[b]{0.49\textwidth}
        \includegraphics[width=\textwidth]{assets/WPD.png}
        \caption{Wavelet packet decomposition}
        \label{fig:wpd-filter-bank}
    \end{subfigure}
    \caption{Dyadic filter banks for discrete wavelet transform~\cite{nandi_condition_2019}}
\end{figure}

The wavelet packet energy ratio for weak feature extraction has been incorporated into the method of multiple frequency bands demodulation (MFBD). The highest $n$ energy coefficients are selected out of 8 narrow frequency bands to subsequently affect the principal components. Demodulation principle for the frequency bandwidth prescribes $n$ to satisfy condition: $1\;/\;2^n > f_{\mathrm{modulation}}\;/\;f_s$. The first few eigenvectors explaining together more than 80\% of energy are retained to reconstruct the signal with Fourier transform and retain the weak fault components~\cite{song_mfbd_2021}.

Tool wear diagnosis based on acoustic emission (AE) signal considers \emph{wavelet packet energy} in bands $E_{8}$, $E_{10}$, $E_{12}$, \emph{energy ratios} $P_{8}$, $P_{13}$, and \emph{energy entropy} as having high correlation ($|r| > 0.8$) with the band saw flank face width. The acoustic signal is decomposed into three layers using Daubechies db3 wavelet. The bottom layer contains bands numbered 7 through 14, each with a bandwidth of 62.5 kHz, because of the 1 MHz sampling frequency. The feature vector constructed in the article includes other statistical metrics out of power spectral density that has reached a notable correlation with evolving wear. The statistics are \emph{skewness}, \emph{kurtosis}, \emph{shape factor}, and \emph{centroid frequency}~\cite{zhuo_research_2022}.

Discrete wavelet transform and wavelet packets partition the spectrum into predefined frequency bands that do not always adequately capture individual elementary oscillations. Adaptive spectral segmentation is needed to extract separate intrinsic mode functions (IMF).

\textbf{Empirical Wavelet Transform} (EWT) constructs adaptive bandpass filters with Meyer wavelet. The inner product of the signal with the scaling function $\hat{\phi_n}$ obtains the approximation coefficients and the inner product with the wavelet function results in detail coefficients.

The normalized frequency axis in range $\omega \in [0, \pi]$ is divided by split points $\omega_0, \dots, \omega_N$ where $\omega_n = f_n \cdot 2 \pi\;/\;f_s$ (Fig.~\ref{fig:ewt-spectrum-segmentation}). Each segment is bounded between $[\omega_{n-1}, \omega_n]$ with transition phase of width $2\tau_n$ and polynomial transition function $\beta(x)$. A tight frame set of empirical wavelets is built by setting the transition phase proportional to the band boundary: $\tau_n = \lambda \omega_n$, and proportional constant must obey constrain: $\lambda = \min\left(\frac{\omega_{n+1} - \omega_n}{\omega_{n+1} + \omega_n}\right)$. The $N$ boundaries defining different portions of the Fourier spectrum are placed at the center between two consecutive local maxima~\cite{gilles_empirical_2013}.

\begin{figure}[ht]
    \centering
    \includegraphics[width=0.9\textwidth]{assets/EWT.png}
    \caption{Empirical wavelet segmentation of Fourier spectrum~\cite{gilles_empirical_2013}}
    \label{fig:ewt-spectrum-segmentation}
\end{figure}

The drawback of EWT is improper segmentation in noisy and non-stationary signals producing too many uninformative partitions of the spectrum. The solutions we discuss fall into two groups that apply improvements before or after the wavelet transformation step. 

A combination of \textbf{Maximum Correlated Kurtosis Deconvolution} and improved EWT (MKCD-EWT) favors periodic impacts by dynamically constructing an optimal FIR filter that maximizes the correlation kurtosis of the signal. The envelope curve smooths of the amplitude spectrum by linear interpolation (Fig.~\ref{fig:mkcd-ewt-segmentation}). The threshold accounting for desired SNR modifies the envelope when more than $n$ components are discovered ($A_h$ is maximal amplitude and $A_l$ is minimal amplitude): $\lambda = A_l + \frac{C}{\mathrm{SNR}}(A_h - A_l)$. \emph{Squared envelope spectrum} and \emph{Teager energy operator spectrum} are computed as fault features for the IMF with highest kurtosis~\cite{li_fault_2019}. 

The segmentation boundaries detection can be accomplished also on the spectral envelope consisting of \textbf{Piecewise Cubic Hermite Interpolating Polynomial} (PCHIP-EWT) instead of scanning the Fourier spectrum directly (Fig.~\ref{fig:pchip-ewt-segmentation}). Divisions of the upper cut-off frequency are kept if local power indicates a subband with useful information ($p_i(f) \geq \lambda$)  as in the case of abrupt level change. Local power is the ratio of $n$th local envelope maximum and difference of indexes of its adjacent local minima:  $p_i(f) = K_{max(i)(f)} \;/\;\left(f_{k_{min(i+1)}} - f_{k_{min(i)}}\right)$. The best $\lambda$ is set experimentally~\cite{zhuang_improved_2020}.

\begin{figure}[ht]
    \centering
    \begin{subfigure}[b]{0.49\textwidth}
        \includegraphics[width=\textwidth]{assets/MKCD-EWT.png}
        \caption{MKCD-EWT~\cite{li_fault_2019}}
        \label{fig:mkcd-ewt-segmentation}
    \end{subfigure}
    \hfill
    \begin{subfigure}[b]{0.49\textwidth}
        \includegraphics[width=\textwidth]{assets/PCHIP-EWT.png}
        \caption{PCHIP-EWT~\cite{zhuang_improved_2020}}
        \label{fig:pchip-ewt-segmentation}
    \end{subfigure}
    \caption{Illustration of improved EWT spectral segmentations}
\end{figure}

Alternatively, the central frequencies of bands containing the most unevenness measured by negentropy are picked after the EWT processing passed. The filter bank is then reconstructed accordingly. `Negentropy measures the inclination of a system to increase its level of organization'~\cite{avoci_spectral_2020}. The larger \emph{spectral negentropy} (Equation~\ref{equ:spectral-negentropy}) suggests more fault-induced impulses. $E(k, f, \Delta f)$ denotes the squared envelope spectrum (SES) which is an envelope in the Fourier domain: $\mathcal{F}\{ |y(k, f, \Delta f)|^2 \}$.

\myequations{Spectral negentropy}
\begin{ceqn}\begin{align}
\Delta I_E(f, \Delta f) = \sum_{k = 0}^{N - 1}{\frac{E(k, f, \Delta f)^2}{\frac{1}{N} \sum_{k=0}^{N-1} E(k, f, \Delta f)^2}} \cdot \ln\left(\frac{E(k, f, \Delta f)^2}{\frac{1}{N} \sum_{k=0}^{N-1} E(k, f, \Delta f)^2}\right)
\label{equ:spectral-negentropy}
\end{align}\end{ceqn}

In \textbf{time-frequency domain scanning empirical spectral negentropy} method (T-FSESNE) the scanning EWT filter establishes the empirical mode components $X_i(t)$. The central frequencies $f_{cj}$ are retained in resonance bands where the frequency-domain spectrum negentropy (FSNE) is greater than mean $\overline{\Delta I_E}$. The optimal bandwidth for each component is searched among the set of incrementally expanding regions fixed at $f_{cj}$. The interval with the largest time-domain spectrum negentropy (TSNE) is selected to restrict the bounds of the mode $(f_{cj}, B_{wk})$~\cite{yonggang_time_2020}.

The following scheme termed \textbf{adaptive and fast empirical wavelet transform} (AFEWT) calculates spectral negentropy of key functions $K'_i(f)$. The spectral section with the highest entropy is assumed to contribute the most to the rolling bearing fault. First, the Fourier transform of the spectrum establishes the key function $K(f)$. This function is analogous to the cepstrum. The leftmost ridge in the $K(f)$ plot is transformed using inverse FFT to attain trend spectrum $T(f)$. Finally, the array of local minima points in the trend spectrum bound frequency bands whose Fourier transform produces key functions $K'_i(f)$~\cite{xu_adaptive_2019}. 

The inspiration for filtering with adaptive basis comes from \textbf{Empirical mode decomposition} (EMD). This procedure locates local maxima and minima in the time waveform that are interpolated with cubic splines to form signal envelopes. The average of the upper and lower envelope is repeatedly subtracted from the residuals recovering the higher order and lower frequency IMFs~\cite{wang_computational_2014}. 

The EMD is not recommended for practical applications because it suffers from mode mixing problems and lacks rigid theoretical foundations. Many improvements based on the mode sifting process, such as Local Mean Decomposition (LMD), Ensemble Empirical Mode Decomposition (EEMD), Concealed Component Decomposition (CCD)~\cite{tiwari_novel_2021}, and Empirical Wavelet Transform, have been devised as being capable of estimating more reasonable modes.

The time-frequency representation offers plenty of features depending on the choice of spectral transformation. We have reviewed Short-time Fourier transform, Continuous Wavelet transform, Synchrosqueezing transform, Discrete Wavelet transform, Wavelet packet decomposition, Empirical Wavelet transform, and operators of negentropy, TKEO, and TEO.

% TODO: we did not utilize wavelets as of yet
% 
% 
%

\subsection{Feature transformation}
Numeric features from the feature extraction phase have non-normal distributions and span the range of scales. Broad differences among features skew the spread in the particular axis. Inevitably it can degrade the discernment of fault diagnostics models that map input onto smooth function as regression does~\cite{zheng_feature_2018}. The feature scaling, power transform, and principal component analysis modify attribute values to gain more meaningful predictors, but one must be cautious in model interpretation.

\myequations{Min-max feature scaling}
\begin{ceqn}\begin{align}
\widetilde{x} = \frac{x - \min(x)}{\max(x) - \min(x)}
\label{equ:min-max-scaler}
\end{align}\end{ceqn}

\myequations{Feature standardization}
\begin{ceqn}\begin{align}
\widetilde{x} = \frac{x - \bar{x}}{\sigma_x}
\label{equ:standardization-scaler}
\end{align}\end{ceqn}

Feature normalization most often takes two forms. \textbf{Min-max scaling} changes original range of values into interval $[0, 1]$ (Equation~\ref{equ:min-max-scaler}). \textbf{Standardization} (Equation~\ref{equ:standardization-scaler}) constrains the mean of the variable to 0 with a variance of 1~\cite{zheng_feature_2018}.

Highly correlated features generated from a relatively small original column space are redundant as they do not provide any additional information for diagnostics. \textbf{Principal Component Analysis} (PCA) solves this problem by projecting potentially linearly dependant features into a new feature space where the incoming information is preserved in a smaller number of features~\cite{zheng_feature_2018}. The threshold of how many principal components are picked depends on the amount of explained variance and desired quantity of data reduction. 

\myequations{Singular Value Decomposition}
\begin{ceqn}\begin{align}
\mathbf{C} = \mathbf{U \Sigma V^T}
\label{equ:svd}
\end{align}\end{ceqn}

PCA consists of taking the Singular Value Decomposition (SVD) (Equation~\ref{equ:svd}) of the mean-centered input matrix. The disadvantage of this method is the loss of explainability in transformed space though it generally outperforms the model working with hand-crafted features~\cite{brito_fault_2021}. The signal samples can be processed directly by PCA without going through an intermediate step of calculating statistical measures. 

\subsection{Feature selection}
Not all features contribute to the model's discriminatory power with an even share. A certain subset can reach better results than other substitute options. The \emph{filtering methods} of feature selection rank the predictors in order of their importance for the problem at hand, or they separate the group of predictors that achieve the top modeling accuracies in unison~\cite{johnson_feature_2019}.

Still, the features can be chosen by \emph{embedded methods} intrinsically as a part of a model or else in \emph{wrapper methods} by machine learning search algorithm at a serious computational expense. However, we will focus next only on filtering methods.

The general steps in selecting the adequate predictors is outlined next~\cite{nandi_condition_2019}:
\begin{enumerate}
    \itemsep0pt
    \item \textbf{Subset generation} - sets of features are generated in different search directions and with various strategies. Attributes are either appended to an empty set or pruned away from a universal set, sequentially or randomly.
    \item \textbf{Subset evaluation} - comparison of subset quality is assessed with relevance measure some of which are discussed below.
    \item \textbf{Stopping criteria} - search is exhausted when the specified number of features has been found, subset metrics cannot be improved further, or satisfactory model performance is achieved. Subset generation and evaluation can be performed multiple times until the stopping criteria are met.
    \item \textbf{Validation} - resulting subset is tested for the specific model on synthetic and real-world datasets against well-known results. 
\end{enumerate}

Filter-based feature selection is preprocessing step independent of model choice with small computational requirements. Measures of information, correlation, similarity, and interdependence output the relevancy rating. Predictors are rated individually or in interacting congregations. 

Most of the scores are based on supervised learning, so they expect true class labels to apportion the measurements respectively. After the scores are assigned to the first $n$ features, those below a threshold are removed. 

The frequently used scores upon which the feature relevance is ordered are~\cite{nandi_condition_2019}: 
\begin{itemize}
\item \textbf{Variance threshold} - removes low-variance features below set threshold.
\item \textbf{Pearson correlation} - expresses linear relationship between class label $c$ and features $f_i$. Attributes are ranked in descending order according to the absolute value of their correlation coefficient from zero upwards, i. e. we seek the highest correlations to the class label.
\myequations{Pearson correlation coefficient}
\begin{ceqn}\begin{align}
r(i) = \frac{\mathrm{cov}(f_i, c)}{\sqrt{\mathrm{var}(f_i) \cdot \mathrm{var}(c)}}
\end{align}\end{ceqn}

\item \textbf{Fisher score} - measures the difference among the means of the classes. It is interchangeable with ANOVA F-value, but it is evaluated for each feature $X^j$ separately. Ideally, the features in the subset have large distances between samples of various classes in $C$ and distances within a class are the smallest possible. In the formula (\ref{equ:fisher-score}), $n_j$ is the sample size of $j$th feature, $\mu^j$ is its sample mean, and $\mu$ is the overall mean.

\myequations{Fisher score}
\begin{ceqn}\begin{align}
\mathrm{FS}(X^j) = \frac{\sum_{i=1}^{C} n_i(\mu_i^j - \mu^i)^2}{\sum_{i=1}^{C} n_i (\sigma_i^j)^2}
\label{equ:fisher-score}
\end{align}\end{ceqn}

\item \textbf{Mutal information} - quantifies the dependence among features, or between features and class labels. It is almost identical to Information Gain. The probability distribution proximity of variables derives from relative entropy known as the Kullback-Leibler distance. Probabilities $P(x)$, $P(y)$, $P(x, y)$ are estimated in the contingency table from event occurrence count to all sample population $|x|\;/\;N$. Joint probability $P(x, y)$ represents samples of feature $x$ simultaneously in class $y$.

\myequations{Mutal information}
\begin{ceqn}\begin{align}
\mathrm{MI}(X, Y) = \sum_{y \in Y} \sum_{x \in X} P(x, y) \log\left(\frac{P(x, y)}{P(x)P(y)}\right)
\end{align}\end{ceqn}
\end{itemize}

Multiple subsets of predictors produced by each evaluation metric can train several variants of a classification model. Sets of attributes can be likewise combined into an ensemble by majority voting, taking the best features out of every group.
\section{Diagnostics techniques} \label{section:diagnostics-techniques}
Fault identification in the rotating machinery is one-class or multi-class classification problem acting in a semi-supervised manner because labels for degraded conditions are scarce in practice. The automation goals in monitoring can be broadly categorized as anomaly detection and recognizing the momentary fault type.

The guiding principles for algorithm selection are simplicity in terms of their straightforward visual explanation for the production managers, and the ability to progressively improve the model on the streaming data to address peculiarities in individual machine constructions.

\subsection{Novelty detection}
Anomaly, novelty, or outlier detection determines whether a health status deviates considerably from the baseline profile. The expert can then step in and diagnose the machine after the notice. Anomaly is a rare observation different from the others raising suspicion that it was created with an unrelated behavior~\cite{aggarwal_outlier_2016}. The observations get assigned anomaly scores, and those over the threshold are novelties.

The measurements coming in the steaming fashion have to be processed in a single pass. The detection model must deal with the minimal admissible assumptions about the nature of the input events. The outliers are based on non-parametric statistical models, nearest-neighbor clustering, and isolation-based approaches~\cite{gervasi_anomaly_2020}.
\bigbreak

\textbf{DenStream} is a density-based algorithm adapted from DBSCAN to cluster streaming data of arbitrarily shaped groups. Samples it includes in the first step into coherent clusters are core data points in each other's neighborhoods. Core points have at least \emph{MinPts} ($\mu$) points in their neighborhood of radius \emph{Eps} ($\varepsilon$) units. Then non-core points in the proximity area of the core point are attached to the cluster containing it~\cite{aggarwal_data_2014}.

\begin{figure}[ht]
    \centering
    \includegraphics[width=0.8\textwidth]{assets/DenStream.png}
    \caption{DenStream~\cite{amini_density_2012}}
    \label{fig:denstream}
\end{figure}

In the online maintenance phase, DenStream summarizes the nearby observations into core \emph{micro-clusters} that can be potential or outlier \emph{micro-clusters} (Fig.~\ref{fig:denstream})~\cite{ghesmoune_state---art_2016}. The (outlier) \emph{o-micro-clusters} can grow into (potential) \emph{p-micro-clusters} when they encompass $\beta \mu$ points. The outliers are discounted after some time in accordance to the decay function: $f(t) = 2^{-\lambda t}$ or below lower weight limit $\xi$. The on-demand offline stage runs DBSCAN over the approximate representation in micro-clusters to deliver final apportionment~\cite{cao_density-based_2006}.
\bigbreak

\textbf{Half-Space Tree} (HS-Tree) stands upon the concept of Isolation forest. It assumes that random splitting of each axis in the feature space will isolate outliers to their separate divisions sooner than non-deviant observations~\cite{gervasi_anomaly_2020,torres_automatic_2022}. This ensemble of trees is better suited for batch setting. HS-Tree stands out in adapting to changing streams because it is trained solely on normal data, requires constant memory, and is faster than density-based methods~\cite{tan_fast_2011}.

\begin{figure}[ht]
    \centering
    \includegraphics[width=0.8\textwidth]{assets/HS-Tree.png}
    \caption{Half-space tree~\cite{tan_fast_2011}}
\end{figure}

A full binary tree is built before the novelty detection begins by splitting tree nodes along the divisions in the randomly chosen perpendicular planes. The node stores its depth, value limits of the axis bisection (half-space), count of contained data points (mass) in two consecutive windows, and link to both child nodes~\cite{tan_fast_2011}. 

The anomaly profile in the latest window is always compared to the predecessor reference window. After the latest window is filled up it replaces the reference window. It suffices to use a window size of 250 and 25 tree ensemble~\cite{tan_fast_2011}.

\subsection{Classification}
Accurate multi-class classification of machine fault causes out of the characteristics of known ones is a much more difficult task than novelty detection. Universal enough fault baseline has to be recorded and transformed into feature space. Interactions among fault manifestations have to be accounted for. We are aware of rapid advances in knowledge transfer for deep neural networks~\cite{maurya_condition-based_2021}. So far, solutions seem not production ready. Therefore, we opt to use a more rudimentary model.
\bigbreak

\textbf{K-nearest neighbors} (kNN) assigns the data point to the class where the majority of $k$ closest instances belong (Fig.~\ref{fig:knn}). This means it can work in the semi-supervised environment because it can infer labels just from knowing a few annotations.

\begin{figure}[ht]
    \centering
    \begin{subfigure}[b]{0.49\textwidth}
        \includegraphics[width=\textwidth]{assets/kNN.png}
        \caption{kNN with k = 5}
        \label{fig:knn}
    \end{subfigure}
    \hfill
    \begin{subfigure}[b]{0.49\textwidth}
        \includegraphics[width=\textwidth]{assets/M-tree.png}
        \caption{M-tree data structure}
        \label{fig:m-tree}
    \end{subfigure}
    \caption{Nearest neighbors classification algorithm~\cite{chen_skyline_2009}}
\end{figure}

The sense of distance between feature vectors $\mathbf{x}$, $\mathbf{y}$ have to be defined, so several metrics are available like \emph{Euclidian distance}, \emph{Mahalanobis distance}, or \emph{RBF kernel} (Tab.~\ref{tab:knn-distance})~\cite{sheng_review_2020}. The demanding neighborhood queries are sped up by a Metric search tree data structure such as \emph{M-Tree} (Fig.~\ref{fig:m-tree}). The optimal $k$ parameter is set in supervised learning according to the breaking point in the elbow curve that plots choices of $k$ against the error rate.

\begin{table}[ht]
\centering
\renewcommand{\arraystretch}{2}
\begin{tabular}{|l|l|}
\hline
\textbf{Distance}     & \textbf{$d(\mathbf{x}, \mathbf{y})$}                                   \\ \hline
Euclidian distance    & $ \sqrt{\sum_{i = 1}^{n}(x_i - y_i)^2} $                               \\ \hline
Mahalanobis distance  & $ (\mathbf{x} - \mathbf{y})^T C^{-1} (\mathbf{x} - \mathbf{y}) $       \\ \hline
Radial basis function & $ \exp\left(-\frac{\lVert \mathbf{x} - \mathbf{y} \rVert^2}{t}\right) $ \\ \hline
\end{tabular}
\caption{Distance metrics for kNN}
\label{tab:knn-distance}
\end{table}

Nearest-neighbour classifier has been successfully applied in machinery fault diagnostics. On the CWRU bearing dataset, the kNN with the accuracy of 96.2\% slightly outperformed SVM (95\%) on the combination of time and frequency-domain features, time-domain features - kNN 91.2\%, SVM 88.8\%, and frequency domain features - kNN (98.8\%), SVM (96.2\%)~\cite{jamil_feature-based_2021}. 

Comparison of kNN and KLDA on a feature set consisting of average, kurtosis, skewness, and standard deviation vectors in each domain has been conducted, achieving a data reduction rate of 95\%. Best accuracies were reached for PSD features with 99.13\% with KLDA and 96.64\% with KNN classifiers and Mahalanobis metric. The sampling frequency was set at 40 kHz~\cite{altaf_new_2022}. Despite kNN lagging in accuracy, we have to keep in mind annotations for faults were complete and machine learning was not tested in a streaming context.

\section{Evaluation datasets}



To validate the industrial solution beforehand on comprehensive dataset we use following as a benchmark.  Faults in the wild are rare so the preparation of balanced datasets is done create deficient configuartions on purpose. Datsets MaFaulDa and CWRU bearing dataset are being used in related work and are publically available in the Comma-Separated Values format. There is also less comprehensive dataset conserning unbalance, but demostrates behaviour during revolution speed up.  

\subsection{Machinery Fault Database}
(MaFaulDa)
50 kHz, 5 sec. recordings, 
Rotation speeds 10~Hz - 60~Hz (737 rpm - 3686 rpm with steps of approximately 60 rpm)

% Sensors: 
	% - Accelerometers IMI 601A01/604B31,  Range ±50 g (±490 m/s2)
	% Magnetic tachometer MT-190
	% Shure SM81 microphone with frequency range of 20-20.000 Hz
	% National Instruments NI 9234 4 channel analog acquisition modules, with sample rate of 51.2 kHz 

% Normal
% Imbalance  - load values within the range from 6 g to 35 g (step of 5g)
% Misalignment (Horizontal/Vertical)
% Bearings (Overhang / Underhang) - Inner, Outer, Cage

% total of 1951 different fault scenarios for 4 different operational conditions. 
% 49 of which from the normal class, 333 from unbalance class, 498 from misalignment class and 1071 from the bearing fault.

% 2 sets of accelerometers each one associated to one bearing (inner and outer) and measuring in 3 directions

% Column 1 - tachometer signal that allows to estimate rotation frequency
% Columns 2 to 4 - underhang bearing accelerometer (axial, radiale tangential direction)
% Columns 5 to 7 - overhang bearing accelerometer (axial, radiale tangential direction)
% Column 8 - microphone


RotorKit Alignment Balance Vibration Trainer (ABVT)
\cite{pestana-viana_influence_2016}
\cite{ribeiro_rotating_2017}

 \footnote{\url{https://www02.smt.ufrj.br/~offshore/mfs/page_01.html}}

\begin{figure}[h]
\centering
\begin{subfigure}[b]{0.48\textwidth}
	\includegraphics[width=\textwidth]{assets/mafaulda-simulator.png}
	\caption{Schematic diagram \cite{pestana-viana_influence_2016}}
\end{subfigure}
\hfill
\begin{subfigure}[b]{0.48\textwidth}
	\includegraphics[width=\textwidth]{assets/machinery-fault-simulator.jpg}
	\caption{Mechanical construction \cite{noauthor_spectraquest_nodate}}
\end{subfigure}
\caption{Machinery Fault Simulator for MaFaulDa}
\label{fig:mafaulda-simulator}
\end{figure}



\subsection{Case Western Reserve University bearing database}
(CWRU) 
2 HP (1.492 kW) Reliance Electric motor \footnote{\url{https://engineering.case.edu/bearingdatacenter/download-data-file}}
Bearings - Inner, Outer
12 kHz, 48 kHz
fan and drive end bearings
Fault diameters of 7 mils, 14 mils, 21 mils, 28 mils, and 40 mils (1 mil=0.001 inches) in diameter were introduced separately at the inner raceway, rolling element (i.e. ball) and outer raceway. 

% Faults ranging from 0.007 inches in diameter to 0.040 inches in diameter were introduced separately 
% at the inner raceway, rolling element (i.e. ball) and outer raceway. 
%Faulted bearings were reinstalled into the test motor and vibration data
%was recorded for motor loads of 0 to 3 horsepower (motor speeds of 1797 to 1720 RPM).

% Feature-based performance of SVM and KNN classifiers for diagnosis of rolling element bearing faults
\cite{jamil_feature-based_2021}

\begin{figure}[h]
\centering
\begin{subfigure}[b]{0.48\textwidth}
	\includegraphics[width=\textwidth]{assets/cwru-test-stand-2.png}
	\caption{Schematic diagram \cite{song_bearing_2022}}
\end{subfigure}
\hfill
\begin{subfigure}[b]{0.48\textwidth}
	\includegraphics[width=\textwidth]{assets/cwru-test-stand.png}
	\caption{Mechanical construction \cite{yuhong_new_2021}}
\end{subfigure}
\caption{CWRU apparatus}
\label{fig:cwru-simulator}
\end{figure}



\subsection{Unbalance on rotating shaft}
Shaft -  unbalances of different sizes \footnote{\url{https://www.kaggle.com/datasets/jishnukoliyadan/vibration-analysis-on-rotating-shaft}}

% Unbalances of different sizes was recorded. Sampling rate =  4096 Hz
% 4 different unbalance strengths were recorded as well as one dataset with the unbalance holder without additional weight (i.e. without unbalance). 
%The rotation speed was varied between approx. 630 and 2330 RPM in the development datasets 
% and between approx. 1060 and 1900 RPM 

%Columns:
% V_in =  The input voltage to the motor controller V_in (in V)
% Measured_RPM  = The rotation speed of the motor (in RPM; computed from speed measurements using the DT9837)
% Vibration_[1 - 3]      : The signal from the 1.,2.,3. vibration sensor

% (“0” = no unbalance, “4” = strong unbalance), (“D” = development or training, “E” = evaluation)
% Radius in mm = 0, 14, 18.5, 23, 23
% Mass 3.3 g (0 ... 3), 6.6 g in 4

\begin{figure}[h]
\centering
\includegraphics[width=0.7\textwidth]{assets/rotating-shaft.jpg}
\caption{Rotating shaft dataset \cite{mey_machine_2020}}
\label{fig:rotating-shaft}
\end{figure}

% Machine Learning-Based Unbalance Detection of a Rotating Shaft Using Vibration Data
\cite{mey_machine_2020}


\chapter{Design}

\section{Research questions}
\begin{enumerate}
\item \emph{Which time-frequency features can be extracted from vibrational signals to provide an accurate record of machinery faults?}
\item \emph{What are the savings in transmission bandwidth when chosen signal features are used in comparison to raw sampled measurement or lossless compression techniques?}
\item \emph{How can the machinery faults be continuously identified based on collected events?}
\end{enumerate}

\section{Infrastructure}
 \begin{itemize}
\item \textbf{Input:} Samples from acceleration in 3-axis, RPM, Noise background
\item \textbf{Output is either:} machine overall status, type of fault, remaining useful life
\item \textbf{Output for domain expert}: Annotation interface, Control chart of trend features, Power frequency spectrum, Waterfall plot
 \end{itemize}

\begin{enumerate}
\item MEMS accelerometers are placed on at least two distinict measurement points in two perpendicular axis and one sensor in base for denoising. Rotational speed is captured at the same time too.
\item Sensors are triggered in regular intervals (every 15 minutes) to collect sample recording from the band saw. Configurable parameters set based on experiments: sampling frequency, dynamic range, window type (Hamming) and size (based on resolution), PSD estimation method (Welch)
\item \textbf{Features} are computed and compared to recent measurements. If there is an statistically significant change the whole summary is send, otherwise keepalive notification is send.
\item Possible local communication between sensors to pre-compute clustering information
\item Database stores history of measurements
\item \textbf{Diagnosis panel runs clustering} with introduction of annotations to notify the operator about observed fault and imminent failure of the machine.
\end{enumerate}

Nároky na hardvér ako výstup analýzy - koprocesor (výpočet) - inštruckčná sada, real time odozva, ramka pri oknách



%\chapter{Implementation} \label{chapter:implementation}
The tools implemented include exploration of datasets using statistical overviews and visualizations, experiments of feature selection which involve machine learning pipeline, and firmware for data logger's hardware. 

\section{Data analysis}
The data processing and data mining on the MaFaulDa and Pump dataset takes place in several \emph{JupyterLab} notebooks written in \emph{Python} language. The purpose of individual notebooks is described in detail in Appendix~\ref{appendix:technical-docs}. Utility functions are located in separate package \emph{vibrodiagnostics} so that the experiments can be realized under multiple conditions.

The tabular data is handled using \emph{Pandas}dataframes that are training batch models from \emph{scikit-learn} library and \emph{imbalanced-learn}, or online models from \emph{RiverML}. The wide variety of graphs and other visualizations are stylized with \emph{Matplotlib} and \emph{Seaborn}. Feature calculation is crafted according to mathematical formulas atop libraries \emph{Numpy}, \emph{SciPy}, and \emph{Time Series Feature Extraction Library}~(TSFEL).

\begin{lstlisting}[style=pythonstyle,caption=Rank product of feature matrix X to label column Y,label={lst:rank-product},morekeywords={DataFrame,rank,set_index,sort_values,gmean}]
METRICS = {
    "corr": selection.corr_classif, 
    "f_stat": sklearn.feature_selection.f_classif, 
    "mi": sklearn.feature_selection.mutual_info_classif }
r = pandas.DataFrame()
for name, metric in METRICS.items():
    # Order the features to leaderboard
    r[name] = leaderboard  
ranks = r.rank(axis="rows", method="first", ascending=False)
return ranks.apply(scipy.stats.gmean, axis=1).sort_values()
\end{lstlisting}

\begin{lstlisting}[style=pythonstyle,caption=Leaderboard of feature importance metric scores,label={lst:feature-leaderboard},morekeywords={DataFrame,rank,set_index,sort_values,metric,gmean}]
pandas.DataFrame(zip(X.columns, metric(X, Y)), 
             columns=["feature", "score"])
      .set_index("feature")
      .sort_values(by="score", ascending=False))
\end{lstlisting}

Since the central focus of this work targets feature selection, the source code listing~\ref{lst:rank-product} and \ref{lst:rank-product} demonstrates the ranking scores calculated for features in data frame X. Leaderboards are united by geometric mean to produce rank product ordering.

\section{Firmware}
The drivers for low-level interfaces of ESP32 microcontroller and FAT32 filesystem are already available in Espressif ESP-IDF SDK. The accelerometer driver is made available by the vendor\footnote{IIS3DWB driver: \url{https://github.com/STMicroelectronics/iis3dwb-pid}}. The entry routine of the firmware mounts the SD card, sets up GPIO pins for LED and button, and executes three \emph{FreeRTOS} tasks. Procedures in firmware are described in Appendix~\ref{appendix:technical-docs}. The hardware (Fig.~\ref{fig:hw-data-logger}) was built by the thesis consultant based on the complete specification supplied by the author.

\begin{figure}[h]
    \centering
    \includegraphics[width=0.7\textwidth]{assets/design/sensor/data-logger.jpg}
    \caption{Accelerometer Data Logger}
    \label{fig:hw-data-logger}
\end{figure}

The tasks run on an event-driven basis through notifications and a queue. They have the following purposes:
\begin{itemize}
\itemsep0pt
\item \textbf{Trigger task} - reacts to notification from button's interrupt handler. Depending on whether the recording is in progress, it either starts peripherals and creates a file or stops them and closes the file. Button debouncing has the form of a two-second delay when interrupts are ignored. The task is pinned to core 1 with priority 2 (higher numbers have more priority).
\item \textbf{Read task} - waits for notification from 9 ms periodic timer to read around half of the accelerometer's FIFO via half-duplex SPI bus at 8 MHz. It sends read-out samples to the queue. In case of any buffer overrun, it turns off the LED prematurely. The task is pinned to core 0 with priority 1.
\item \textbf{Write task} - reads the samples from the queue, and after locking the mutex for the opened file, it writes them to the card. The file is also manipulated within the trigger task, so the lock prevents race conditions. The task is pinned to core 1 with priority 1.
\end{itemize}

The testing process of the firmware revealed two issues that were resolved subsequently. The STEVAL evaluation board is plagued with broken hardware interrupt lines. The FIFO watermark interrupt on the INT1 pin of the accelerometer stopped firing after a few seconds and stayed at a high logic level. We noted the problem first on the digital multimeter, then confirmed it on an oscilloscope and found the same issue in the vendor's forum thread\footnote{IIS3DWB Interrupt stops triggering while sampling for long periods: \url{https://community.st.com/t5/mems-sensors/iis3dwb-interrupt-stops-triggering-while-sampling-for-long/td-p/203630}}.

\begin{figure}[h]
    \centering
    \begin{subfigure}[b]{0.49\textwidth}
        \includegraphics[width=\textwidth]{assets/design/oscilloscope/bin-fomat.jpg}
        \caption{Binary format}
        \label{fig:implementation:binary-format}
    \end{subfigure}
    \hfill
    \begin{subfigure}[b]{0.49\textwidth}
        \includegraphics[width=\textwidth]{assets/design/oscilloscope/tsv-format.jpg}
        \caption{TSV format}
        \label{fig:implementation:tsv-format}
    \end{subfigure}
    \caption{Timing issue of writing samples to SD card captured on oscilloscope}
\end{figure}

Other impedement presented the slow string formatting with \emph{printf} family of functions during the sampling of the accelerometer. Direct write of columns to TSV is not possible because the real time deadline could not be met due to high sampling rate. This is true even if the queue is utilized and the processing takes place in an another task. The hardware FIFO of 512 sample vectors fills up in 19.05 ms.


%TODO
% After GPIO pin was set to change output level on one buffer write completion the TSV rows took around 70 ms to format and write out as evidenced by Figure~ref{fig:implementation:tsv-format}. This overextended the time 3.7 times.

% Timing issue
% $y = 3.7x$ is monotonically increasing function (y is buffers left to process and x is buffers already processed). At some point it surpasses the buffer of any size. real-time deadline must satisfy condition $t < 2 \cdot n_{FIFO} \cdot \frac{1}{f_s}$ which is 38.1 ms beyond which the samples get dropped.

% tool for convert bin2csv


%TODO
% Sensor attached to machines with connector up and pressed gently towards the surface.

%TODO - advise you have to store 5 seconds in memory or write binary then process on second pass in constrained system

\begin{figure}[h]
    \centering
    \begin{subfigure}[b]{0.49\textwidth}
        \includegraphics[width=\textwidth]{assets/design/sensor/sensor.jpg}
        \caption{Sensor on water pump}
    \end{subfigure}
    \hfill
    \begin{subfigure}[b]{0.49\textwidth}
        \includegraphics[width=\textwidth]{assets/design/sensor/sensor-compressor.jpg}
        \caption{Sensor on scroll compressor}
    \end{subfigure}
    \caption{Data logger on the machines in measurement positions}
\end{figure}

\chapter{Conclusion} \label{section:conclusion}  
In the thesis we focused on trend indicator selection for an inexpensive industrial condition monitoring solution from vibration signals. The goal is to enable timely fault detection of machinery parts with as little input data as possible. For that purpose, we answer four research questions.

Attributes extracted to describe machine behavior come mainly from descriptive statistics, audio signal processing studies, and vibrodiagnostics technical standards (ISO 20186 and ISO 13343). These formulas compute 10 features summarizing the waveform in the temporal domain and 11 features characterizing spectral density estimation in 3 spatial directions \textbf{(RQ1)}.

In order to achieve more pronounced data savings, we choose the subset of 3 features in each domain by keeping the ones with the most similarity to the target variable. Feature selection metrics of the correlation coefficient, F statistic, mutual information, and their rank product are applicable in supervised learning. The features are squished from multiple dimensions using the Euclidian norm. Lossy compression ratios attained are 2381:1 for all features and 25000:1 for 6 features in the MaFaulDa dataset. We managed to discard more than 99.995\% of irrelevant data \textbf{(RQ2)}.

Feature subsets are subjected after normalization to a k-nearest neighbor classifier that ascertains their relative fault detection power. Because of model overtraining with small k, the feature triplets equal or slightly outperformed the whole set of features on the validation set with accuracy up to 10\%. The spectral features reach higher accuracies than temporal because of their smaller interdependency \textbf{(RQ3)}. 

The ensemble of feature selection with rank product produces the best model performance out of three combined in the majority of situations. No approach could find a triplet of predictors with an accuracy close to optimal one discovered exhaustively. Training on three principal components produced better accuracy than filtering feature selection, but PC mapping onto original trend indicators is unclear even in loading plots \textbf{(RQ3)}.

The considerable obstacle in an autonomous fault detection system deployment is the availability of labels for target variables. Annotations can be assigned belatedly or even never. Incremental learning kNN model on an unbalanced dataset on the whole feature set achieves at best 90\% accuracy with immediate feedback, 85\% with labels coming in 250 long tumbling windows, and 82\% with just 25\% of observations associated with the label. The comparable model trained in batch reaches an accuracy of 98\% \textbf{(RQ4)}.

Conclusions so far have been made on the MaFaulDa dataset imitating the realistic conditions. Therefore, in proper validation of the proposed solution in practice, we will compare it with the custom-made dataset. 

The vibration signals will be gathered on compressors in air conditioning units and water pumps in municipal pumping stations. We will develop firmware for sensor units capable of sampling accelerometers and saving measurements onto SD cards. The challenge awaits us in labeling samples themselves as it requires substantial expert knowledge. 

Feature selection methods should be incorporated into incremental learning. Hyperparameters shall be adjusted, and the balancing method used to increase model performance. The outlined tasks awaiting completion are planned for the DP~\rom{3} phase.

%\nocite{*}
\printbibliography[heading=bibintoc]

%  Appendix ---------------------------------------------------------
\addtocontents{toc}{\protect\setcounter{tocdepth}{0}}
\addtocontents{toc}{\cftpagenumbersoff{chapter}}
\let\svaddcontentsline\addcontentsline
\renewcommand\addcontentsline[3]{%
  \ifthenelse{\equal{#1}{lof}}{}%
  {\ifthenelse{\equal{#1}{lot}}{}{\svaddcontentsline{#1}{#2}{#3}}}}

\appendix
\titleformat{\chapter}{\normalfont\huge\bf}{Appendix \thechapter:}{1em}{}
\renewcommand{\chaptermark}[1]{\markboth{\MakeUppercase{Appendix \thechapter.\ #1}}{}}

% Resume
\thispagestyle{empty}
\chapter{Resumé}
\pagenumbering{arabic}
\renewcommand*{\thepage}{A-\arabic{page}}

\section{Úvod}
Vzostup priemyslu 4.0 so sebou prináša väčšiu mieru automatizácie s cieľom dosiahnuť optimálne využitie dostupných zdrojov. Na základe nepretržitého sledovania opotrebenia zariadení v reálnom čase sa majú zabezpečiť nápravné opatrenia na opravu alebo výmenu súčiastok včas, v reakcii na trendové ukazovatele. 

Cieľom je zachovať požadovanú bezpečnosť a efektivitu výroby a zároveň predĺžiť životnosť rotujúcich komponentov. Proaktívna diagnostika porúch je nevyhnutná na začatie opráv bez nadbytočných nákladov. Vibrácie predstavujú nerušivý spôsob, ako zistiť a zaznamenať prípadne fatálne zlyhania hneď v zárodku. Hlavným problémom pri monitorovaní veľa strojov s vibráciami, je to, že vzniká množstvo záznamov, ktoré nie sú priamo užitočné pre operátora výrobnej linky. Väčšina signálov sa zobrazí maximálne raz, preto je zbytočné ich ukladať alebo prenášať vcelku. 

Zároveň na dosiahnutie maximálnej presnosti detekcie musí byť model strojového učenia trénovaný pre cieľové prostredie. Poruchy sú navyše pomerne zriedkavé udalosti, ktoré sa zvyčajne vyskytujú s odstupom niekoľkých mesiacov. Za týchto okolností je ťažké rýchlo získať dostatočne veľkú vzorku poruchových udalostí.

\section{Sledovanie prevádzkového stavu}
Existujú tri rôzne prístupy k údržbe strojov: reaktívny, preventívny a prediktívny.
Pri reaktívnej údržbe beží stroj až do úplného zlyhania a je prijateľná vtedy, keď je možná úplná a rýchla výmena pokazeného stroja za záložný. Preventívna údržba prebieha v pravidelných intervaloch odvodených od vopred určeného rozvrhu v alebo strednej doby medzi poruchami. Prediktívna údržba zlepšuje predvídateľnosť oproti reaktívnej údržbe a eliminuje plytvanie voči príliš obozretnej prevencii. Odstávka stroja je naplánovaná po zistení kritických hodnôt a po odhalení problematických komponentov.

Mechanické problémy počas prevádzky strojov spôsobujú v mnohých prípadoch vibrácie. Vibroakustická diagnostika sa preto považuje za jednu z najdôležitejších metód pri včasnej identifikácii porúch komponentov. Najbežnejšie sa vyskytujúcimi poruchami sú nevyváženosť, nesúososť, vôľa, excentricita, deformácia, trhlina a nadmerné trenie. 

Symptómy porúch rotačných strojov sa prejavujú rôznymi frekvenčnými pásmami, ale väčšina je závislá od rotačnej rýchlosti súčiastky. Nevyváženosť, nesúosovosť a vôľa sa bežne objavujú v frekvenciách do 300 Hz. Poruchy ložísk a prevodovky v neskorých štádiách vývoja sa prejavujú v rozsahu medzi 300 Hz a 1 kHz. Vyššie frekvencie do 10 kHz pomáhajú odhaliť poruchy ložísk v skorších štádiách rozvoja.

Postupy monitorovania stavu založené na vibráciách musia byť v súlade s normatívnymi smernicami ISO 20816 a ISO 13373. Normy sa týkajú umiestnenia meracích zariadení, zberu údajov, konvencií nastavenia úrovní závažnosti porúch. 


\section{Extrakcia a výber atribútov}
Prediktívna údržba má ideálne predpoklady na využitie extrakcie atribútov, pretože signál je zvyčajne stacionárny a trendové premenné v časovej a frekvenčnej oblasti vychádzajú z expertných znalostí v oblasti mechaniky. Výhody dodatočného úsilia v porovnaní so spracovaním pôvodných vzoriek spočívajú v dosahovaní lepšej presnosti klasifikácie, znížení výpočtovej záťaže a znížení potreby úložnej kapacity. Výber atribútov nie je samostatným krokom v procese strojového učenia, ale mal by sa vykonávať iteratívne na zlepšenie výsledného modelu.

Najrozšírenejšími používanými atribútmi sú štatistické miery centrálneho momentu: priemer, rozptyl, štandardná odchýlka, šikmosť a špicatosť. Charakteristiky amplitúdy zahŕňajú kvadratický priemer (rms), vzdialenosť špička-špička a maximum. Ostatné významné atribúty časovej oblasti sú odvodené ako pomery a sú nim: faktor výkyvu, faktor rozpätia, faktor impulzu a faktor tvaru.  

V spektrálnej oblasti môžeme získať obvyklé štatistické vlastnosti distribúcie, ktorými sú spektrálne ťažisko, šikmosť a špicatosť. Okrem toho sa extrahujú roll-on a roll-off, fundamentálna frekvencia, entropia, negentropia, vzájomná korelácia spektier, pomer signálu k šum, energia vo frekvenčných pásmach.

Atribúty neprispievajú k prediktívnej sile modelu s rovnakým podielom. Výber ich optimálnej podmnožiny je NP-ťažký kombinatorický problém. Kroky všeobecného postup pri výbere atribútov metódou filtrovania sú generovanie podmnožín, vyhodnotenie podmnožín, ukončovacie kritérium hľadania, a validácia.

Hodnotenie relevancie atribútov sú založené na skórovaní podobnosti s predikovanou premennou. Často používané spôsoby zoraďovania dôležitosti atribútov sú prah rozptylu, koeficienty korelácie, ANOVA F štatistika, a vzájomná informácia. Viaceré podmnožiny prediktorov produkovaných každou z výberových metrík môžu slúžiť na trénovanie viacerých variantov klasifikačného modelu. Množiny atribútov je možné kombinovať do súboru volebným systémom ako sú väčšinové hlasovanie alebo súčin poradí.

\section{Diagnostické prístupy}
Identifikácia porúch v rotujúcich strojoch je binárny alebo viactriedny klasifikačný problém, ktorý pracuje na princípe učenia čiastočne s učiteľom, pretože označenia pre degradované stavy stroja sú v praxi zriedkavé. Ciele automatizácie monitorovania možno rozdeliť na detekciu anomálií a rozpoznanie typu poruchy.

Detekcia anomálií, novostí alebo odľahlých hodnôt určuje, či sa prevádzkový stav stroja výrazne odchyľuje od normálu. Po upozornení môže zasiahnuť odborník a stroj diagnostikovať. Odľahlé hodnoty sú odvodzované na základe neparametrických štatistických modelov, zhlukovania podľa najbližších susedov a prístupov založených na izolácii anomálnych vzoriek. DenStream je algoritmus zhlukovania založený na hustote prispôsobený z DBSCAN na zhlukovanie prúdových dát do ľubovoľne tvarovaných skupín. Half-space strom predpokladá, že náhodné delenie v každej osi v priestore atribútov izoluje odľahlé hodnoty do samostatných oddielov skôr ako nedeviantné pozorovania. 

Presná viactriedna klasifikácia príčin porúch stroja podľa vopred známych charakteristík je oveľa náročnejšia úloha ako objavenie anomálií. Algoritmus k-najbližších susedov (k-NN) priradí pozorovanie triede, do ktorej patrí väčšina $k$ bodov v blízkom okolí podľa použitej miery vzdialenosti. Nachádza uplatnenie aj v učení čiastočne s učiteľom, pretože dokáže odvodiť označenia len zo znalosti niekoľkých anotácií.

Ďalším prístupom je online alebo postupné učenie, ktoré aktualizuje parametre modelu s každou novou prichádzajúcou udalosťou. Tento prístup je užitočný pri spracovaní veľkých dát, kedy celý súbor údajov nie je k dispozícii vopred alebo ho nemožno spracovať naraz z dôvodu pamäťových obmedzení.

\section{Výskumné otázky}
Cieľom tejto práce je poskytnúť odpovede na štyri výskumné otázky:
\begin{enumerate}
\itemsep0pt
\item Aké atribúty dokážeme extrahovať z vibračných signálov?
\item Akú úsporu dát dosiahneme výberom atribútov?
\item Aké budú presnosti diagnostiky porúch s rôznymi sadami atribútov?
\item Ako môžeme priebežne označovať poruchové stavy?
\end{enumerate}

\section{Návrh spracovania pre MaFaulDa}

\section{Zber vibrácií v priemysle}
Doteraz uplatnená metodika pre súbor údajov zaznamenaných v laboratóriu sa aplikuje na vibračných signáloch z priemyselného prostredia. Pri monitorovaní zužitkujeme mierne prispôsobený postup z noriem. Ten zahŕňa výber strojov určených na monitorovanie, identifikáciu pozícií na meranie podľa technických štandardov, predbežné merania a vývoj senzorovej jednotky.

Na zber údajov boli vyčlenené dva špirálové kompresory ako súčasť klimatizačných jednotiek pre dátové centrum a tri čerpadlá s troma elektrometrami v prečerpávacej stanici na pitnú vodu. Dlhodobejšie merania uskutočníme vlastným vnoreným systémom na báze vývojovej dosky ESP32-PoE-ISO so slotom na SD kartu. Ako senzor vibrácii použijeme MEMS akcelerometer IIS3DWB. Vyznačuje sa vysokou šírkou pásma až 6.3 kHz, nízkym šumom, a vysokou výstupným dátovým tokom 26.7 kHz cez SPI zbernicu.

\section{Vyhodnotenie presnosti diagnostiky}

\section{Rozbor dátovej sady z priemyslu}

\section{Záver}
V diplomovej práci sme sa zamerali na výber trendových ukazovateľov pre riešenie monitorovania prevádzkového stavu a odhaľovanie porúch z vibračných signálov.  Extrahované premenné pochádzajú hlavne z popisných štatistík, z článkov o spracovaní zvukových signálov a technických noriem vibrodiagnostiky.

Dosiahnuté stratové kompresné pomery pre MaFaulDa sú 2381:1 pre všetky atribúty a 25000:1 pre šesť atribútov. Výber atribútov metódou súčinu poradí zabezpečí väčšinou najlepšiu presnosť k-NN modelu oproti metrikám samostatne. Žiadny prístup však nedokázal nájsť trojicu prediktorov s presnosťou blízkou optimálnej, ktorá je až 98\%. Trénovanie k-NN na troch hlavných komponentoch prinieslo lepšiu presnosť ako výber atribútov. 

Model postupného učenia k-NN dosahuje prinajlepšom 90\% presnosť s okamžitou spätnou väzbou, 85\% so značkami oneskorenými o 250 pozorovaní a 82\% s iba 25\% anotovaného súboru údajov. Porovnateľný model trénovaný v dávkach dosahuje presnosť 98\%.  

\clearpage

\thispagestyle{empty}
\chapter{Plan of work}
\pagenumbering{arabic}
\renewcommand*{\thepage}{B-\arabic{page}}

\section{Winter semester}

\begin{table}[h!]
\def\arraystretch{1.25}
\begin{tabular}{|l|p{12cm}|}
\hline
\textbf{Period} & \textbf{Work}                                                                                                                                                                                                                         \\ \hline
\nth{1} week         & Consultation with the supervisor on directions of the future work based on literature review during previous semester.
\\ \hline
\nth{2} week         & Outline the key sections of the analysis part in the thesis.
\\ \hline
\nth{3} week         & Match supporting literature with analysis sections. Further invesigation on the feature engineering methodology in condition monitoring.
 \\ \hline
\nth{4} week         & Summarize notes from condition monitoring articles and videorecordings of tutorials and conferences.
 \\ \hline
\nth{5} week         & Research transformation of vibration signal to feature space using time-frequency, harmonic and energy statistical metrics. Progress report meeting with the supervisor.
 \\ \hline
\nth{6} week         & Find articles and take notes about unsupervised and semi-supervised techniques in streaming data for machinery diagnostics, in order to gather information about suitable features.
 \\ \hline
\nth{7} week         & TBD (Narrow down wide variety applicable methods for signal decomposition)
 \\ \hline
 \nth{8} week         & TBD (Write thesis section on condition monitoring and machinery fault types)
 \\ \hline

\end{tabular}
\end{table}

\clearpage
\newpage


\section{Summer semester}

\clearpage


% Ďalšie prílohy
% \input{chapters/appendix/B-technical-docs}
% Ak nechce vypísať čísla strán na konci prílohy: \cleardoublepage
	
%TODO: Digital medium
% \input{chapters/appendix/C-digital-medium}

\end{document}
