\thispagestyle{empty}
\chapter{Resumé}
\pagenumbering{arabic}
\renewcommand*{\thepage}{A-\arabic{page}}

\section{Úvod}
Vzostup priemyslu 4.0 so sebou prináša väčšiu mieru automatizácie s cieľom dosiahnuť optimálne využitie dostupných zdrojov. Na základe nepretržitého sledovania opotrebenia zariadení v reálnom čase sa majú zabezpečiť nápravné opatrenia na opravu alebo výmenu súčiastok včas, v reakcii na trendové ukazovatele. 

Cieľom je zachovať požadovanú bezpečnosť a efektivitu výroby a zároveň predĺžiť životnosť rotujúcich komponentov. Proaktívna diagnostika porúch je nevyhnutná na začatie opráv bez nadbytočných nákladov. Vibrácie predstavujú nerušivý spôsob, ako zistiť a zaznamenať prípadne fatálne zlyhania hneď v zárodku. Hlavným problémom pri monitorovaní veľa strojov s vibráciami, je to, že vzniká množstvo záznamov, ktoré nie sú priamo užitočné pre operátora výrobnej linky. Väčšina signálov sa zobrazí maximálne raz, preto je zbytočné ich ukladať alebo prenášať vcelku. 

Zároveň na dosiahnutie maximálnej presnosti detekcie musí byť model strojového učenia trénovaný pre cieľové prostredie. Poruchy sú navyše pomerne zriedkavé udalosti, ktoré sa zvyčajne vyskytujú s odstupom niekoľkých mesiacov. Za týchto okolností je ťažké rýchlo získať dostatočne veľkú vzorku poruchových udalostí.

\section{Sledovanie prevádzkového stavu}
Existujú tri rôzne prístupy k údržbe strojov: reaktívny, preventívny a prediktívny.
Pri reaktívnej údržbe beží stroj až do úplného zlyhania a je prijateľná vtedy, keď je možná úplná a rýchla výmena pokazeného stroja za záložný. Preventívna údržba prebieha v pravidelných intervaloch odvodených od vopred určeného rozvrhu v alebo strednej doby medzi poruchami. Prediktívna údržba zlepšuje predvídateľnosť oproti reaktívnej údržbe a eliminuje plytvanie voči príliš obozretnej prevencii. Odstávka stroja je naplánovaná po zistení kritických hodnôt a po odhalení problematických komponentov.

Mechanické problémy počas prevádzky strojov spôsobujú v mnohých prípadoch vibrácie. Vibroakustická diagnostika sa preto považuje za jednu z najdôležitejších metód pri včasnej identifikácii porúch komponentov. Najbežnejšie sa vyskytujúcimi poruchami sú nevyváženosť, nesúososť, vôľa, excentricita, deformácia, trhlina a nadmerné trenie. 

Symptómy porúch rotačných strojov sa prejavujú rôznymi frekvenčnými pásmami, ale väčšina je závislá od rotačnej rýchlosti súčiastky. Nevyváženosť, nesúosovosť a vôľa sa bežne objavujú v frekvenciách do 300 Hz. Poruchy ložísk a prevodovky v neskorých štádiách vývoja sa prejavujú v rozsahu medzi 300 Hz a 1 kHz. Vyššie frekvencie do 10 kHz pomáhajú odhaliť poruchy ložísk v skorších štádiách rozvoja.

Postupy monitorovania stavu založené na vibráciách musia byť v súlade s normatívnymi smernicami ISO 20816 a ISO 13373. Normy sa týkajú umiestnenia meracích zariadení, zberu údajov, konvencií nastavenia úrovní závažnosti porúch. 


\section{Extrakcia a výber atribútov}
Prediktívna údržba má ideálne predpoklady na využitie extrakcie atribútov, pretože signál je zvyčajne stacionárny a trendové premenné v časovej a frekvenčnej oblasti vychádzajú z expertných znalostí v oblasti mechaniky. Výhody dodatočného úsilia v porovnaní so spracovaním pôvodných vzoriek spočívajú v dosahovaní lepšej presnosti klasifikácie, znížení výpočtovej záťaže a znížení potreby úložnej kapacity. Výber atribútov nie je samostatným krokom v procese strojového učenia, ale mal by sa vykonávať iteratívne na zlepšenie výsledného modelu.

Najrozšírenejšími používanými atribútmi sú štatistické miery centrálneho momentu: priemer, rozptyl, štandardná odchýlka, šikmosť a špicatosť. Charakteristiky amplitúdy zahŕňajú kvadratický priemer (rms), vzdialenosť špička-špička a maximum. Ostatné významné atribúty časovej oblasti sú odvodené ako pomery a sú nim: faktor výkyvu, faktor rozpätia, faktor impulzu a faktor tvaru.  

V spektrálnej oblasti môžeme získať obvyklé štatistické vlastnosti distribúcie, ktorými sú spektrálne ťažisko, šikmosť a špicatosť. Okrem toho sa extrahujú roll-on a roll-off, fundamentálna frekvencia, entropia, negentropia, vzájomná korelácia spektier, pomer signálu k šum, energia vo frekvenčných pásmach.

Atribúty neprispievajú k prediktívnej sile modelu s rovnakým podielom. Výber ich optimálnej podmnožiny je NP-ťažký kombinatorický problém. Kroky všeobecného postup pri výbere atribútov metódou filtrovania sú generovanie podmnožín, vyhodnotenie podmnožín, ukončovacie kritérium hľadania, a validácia.

Hodnotenie relevancie atribútov sú založené na skórovaní podobnosti s predikovanou premennou. Často používané spôsoby zoraďovania dôležitosti atribútov sú prah rozptylu, koeficienty korelácie, ANOVA F štatistika, a vzájomná informácia. Viaceré podmnožiny prediktorov produkovaných každou z výberových metrík môžu slúžiť na trénovanie viacerých variantov klasifikačného modelu. Množiny atribútov je možné kombinovať do súboru volebným systémom ako sú väčšinové hlasovanie alebo súčin poradí.

\section{Diagnostické prístupy}
Identifikácia porúch v rotujúcich strojoch je binárny alebo viactriedny klasifikačný problém, ktorý pracuje na princípe učenia čiastočne s učiteľom, pretože označenia pre degradované stavy stroja sú v praxi zriedkavé. Ciele automatizácie monitorovania možno rozdeliť na detekciu anomálií a rozpoznanie typu poruchy.

Detekcia anomálií, novostí alebo odľahlých hodnôt určuje, či sa prevádzkový stav stroja výrazne odchyľuje od normálu. Po upozornení môže zasiahnuť odborník a stroj diagnostikovať. Odľahlé hodnoty sú odvodzované na základe neparametrických štatistických modelov, zhlukovania podľa najbližších susedov a prístupov založených na izolácii anomálnych vzoriek. DenStream je algoritmus zhlukovania založený na hustote prispôsobený z DBSCAN na zhlukovanie prúdových dát do ľubovoľne tvarovaných skupín. Half-space strom predpokladá, že náhodné delenie v každej osi v priestore atribútov izoluje odľahlé hodnoty do samostatných oddielov skôr ako nedeviantné pozorovania. 

Presná viactriedna klasifikácia príčin porúch stroja podľa vopred známych charakteristík je oveľa náročnejšia úloha ako objavenie anomálií. Algoritmus k-najbližších susedov (k-NN) priradí pozorovanie triede, do ktorej patrí väčšina $k$ bodov v blízkom okolí podľa použitej miery vzdialenosti. Nachádza uplatnenie aj v učení čiastočne s učiteľom, pretože dokáže odvodiť označenia len zo znalosti niekoľkých anotácií.

Ďalším prístupom je online alebo postupné učenie, ktoré aktualizuje parametre modelu s každou novou prichádzajúcou udalosťou. Tento prístup je užitočný pri spracovaní veľkých dát, kedy celý súbor údajov nie je k dispozícii vopred alebo ho nemožno spracovať naraz z dôvodu pamäťových obmedzení.

\section{Výskumné otázky}
Cieľom tejto práce je poskytnúť odpovede na štyri výskumné otázky:
\begin{enumerate}
\itemsep0pt
\item Aké atribúty dokážeme extrahovať z vibračných signálov?
\item Akú úsporu dát dosiahneme výberom atribútov?
\item Aké budú presnosti diagnostiky porúch s rôznymi sadami atribútov?
\item Ako môžeme priebežne označovať poruchové stavy?
\end{enumerate}

\section{Návrh metód s MaFaulDa}
Súbor údajov MaFaulDa používame ako smerodajný pri určovaní metód schopných nasadenia na senzorovú jednotku. MaFaulda obsahuje 1951 záznamov so vzorkovaciu frekvenciu 50 kHz a označenými simulovanými poruchami rôznej závažnosti. Nahrávky obsahujú časové rady z dvoch trojosích piezoelektrických akcelerometrov. Po úprave ponechávame 6 typov značiek: referenčný bezporuchový stav, dve poruchy hriadeľa (nevyváženosť, nesúososť) a tri poruchy ložísk (poruchy klietky, guľôčok, vonkajšieho krúžku). 

Zo signálov rozdeleních na päť jednosekundových častí sa odstraňuje jednosmerná zložka odčítaním priemernej akcelerácie, po ktorej nasleduje dolnopriepustný 10 kHz filter. Z častí signálu je potom vytvorených 10 časových a 11 spektrálnych premenných. Euklidovská norma trojrozmerných atribútov eliminuje závislosť na smere merania. Pri výbere atribútov nie je do množiny pridaná taká dvojica, ak ich absolútna hodnota korelácie presahuje 0.95.

Predpokladáme, že dátové body rozprestreté v každej dimenzii priestoru môžu dobre rozlíšiť skupiny. Ukazuje sa, že premenné sú navzájom viac korelované v časovej doméne ako ilustruje analýza hlavných komponentov. Pre 95\% vysvetleného rozptylu PCA sú v časovej doméne potrebné 3 zložky, zatiaľ čo vo frekvenčnej doméne 4 zložky. PCA efektívne vyjadruje atribúty v menej rozmernom priestore, ale ich výslednou lineárnou kombináciou sa ťažko odôvodňujú rozhodnutia modelu.

Pri posudzovaní všeobecne najdôležitejších atribútov vychádzame z trojíc nekorelovaných atribútov vytvorených v 24 scenároch. Podmienky scenárov vznikli kombináciou štyroch kritérií: dávkové alebo inkrementálne učenie, pozícia ložiska, predikovaná premenná, limit na rotačnú rýchlosť. Na základe schvaľovacieho hlasovania sú najčastejšie sa vyskytujúcimi atribútmi v časovej doméne: špička-špička, faktor tvaru, faktor výkyvu. Vo frekvenčnej doméne sú to spektrálne ťažisko, roll-on a roll-off.

Tri typy experimentov s modelom k-NN prebiehajú s validáciou metódou hold-out, Najprv sa model naučí všetky extrahované prvky, takže nedochádza k výberu prvkov. Metódou hrubej sily sa hľadá kombinácia troch atribútov s najvyššou trénovaciou presnosťou. Následne sa porovnáva presnosť modelu pre tri atribúty zvolené technikami výberu atribútov. Dávkový model k-NN slúži ako referenčný, podľa ktorého sa posudzuje k-NN v postupnom učení.

Online učenie napodobňuje sťažené podmienky diagnostiky strojov, ktoré sa objavujú v praxi. Oneskorené dodanie alebo vynechanie skutočných značiek nepochybne znižuje spoľahlivosť klasifikácie. Modely k-NN v experimentoch s postupným učením sú trénované na rovnakom základnom súbore údajov pre ložisko A nad všetkými extrahovanými atribútmi. Týmto spôsobom môžeme porovnať trénovacie presnosti tréningu pre dávkové a postupné učenie. k-NN sme nastavili na 5 susedov a euklidovskú vzdialenosť. Metriky online učenia sa vyhodnocujú v metódou progresívnom vyhodnocovanie na ešte nevyváženom súbore údajov.

\section{Zber vibrácií v priemysle}
Doteraz uplatnená metodika pre súbor údajov zaznamenaných v laboratóriu sa aplikuje na vibračných signáloch z priemyselného prostredia. Pri monitorovaní zužitkujeme mierne prispôsobený postup z noriem. Ten zahŕňa výber strojov určených na monitorovanie, identifikáciu pozícií na meranie podľa technických štandardov, predbežné merania a vývoj senzorovej jednotky. Zber nového súboru údajov sprevádza vopred dohodnutý harmonogram.

Na zber údajov boli vyčlenené dva špirálové kompresory ako súčasť klimatizačných jednotiek pre dátové centrum a tri čerpadlá s troma elektrometrami v prečerpávacej stanici na pitnú vodu. Dlhodobejšie merania uskutočníme vlastným vnoreným systémom na báze vývojovej dosky ESP32-PoE-ISO so slotom na SD kartu. Ako senzor vibrácii použijeme MEMS akcelerometer IIS3DWB. Vyznačuje sa vysokou šírkou pásma až 6.3 kHz, nízkym šumom, a vysokou výstupným dátovým tokom 26.7 kHz cez SPI zbernicu.

\section{Záver}
V diplomovej práci sme sa zamerali na výber trendových ukazovateľov pre riešenie monitorovania prevádzkového stavu a odhaľovanie porúch z vibračných signálov.  Extrahované premenné pochádzajú hlavne z popisných štatistík, z článkov o spracovaní zvukových signálov a technických noriem vibrodiagnostiky.

Dosiahnuté stratové kompresné pomery pre MaFaulDa sú 2381:1 pre všetky atribúty a 25000:1 pre šesť atribútov. Výber atribútov metódou súčinu poradí zabezpečí väčšinou najlepšiu presnosť k-NN modelu oproti metrikám samostatne. Žiadny prístup však nedokázal nájsť trojicu prediktorov s presnosťou blízkou optimálnej, ktorá je až 98\%. Trénovanie k-NN na troch hlavných komponentoch prinieslo lepšiu presnosť ako výber atribútov. 

Model postupného učenia k-NN dosahuje prinajlepšom 90\% presnosť s okamžitou spätnou väzbou, 85\% so značkami oneskorenými o 250 pozorovaní a 82\% s iba 25\% anotovaného súboru údajov. Porovnateľný model trénovaný v dávkach dosahuje presnosť 98\%.  Doteraz boli urobené závery podľa súboru údajov MaFaulDa, ktoré plánujeme overiť na súbore údajov získaných v priemysle počas DP3.

\clearpage