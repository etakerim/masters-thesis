\thispagestyle{empty}
\chapter{Resumé}
\pagenumbering{arabic}
\renewcommand*{\thepage}{A-\arabic{page}}

\section{Úvod}
Vzostup priemyslu 4.0 prináša so sebou väčšiu mieru automatizácie s cieľom dosiahnuť optimálne využitie dostupných zdrojov. Na základe nepretržitého sledovania opotrebenia zariadení v reálnom čase sa majú zabezpečiť nápravné opatrenia na opravu alebo výmenu súčiastok včas, v reakcii na trendové ukazovatele. 

Cieľom je zachovať požadovanú bezpečnosť a efektivitu výroby a zároveň predĺžiť životnosť rotujúcich komponentov. Proaktívna diagnostika porúch je nevyhnutná na začatie opráv bez nadbytočných nákladov. Vibrácie predstavujú nerušivý spôsob, ako zistiť a zaznamenať prípadne fatálne zlyhania hneď v zárodku. Hlavným problémom pri monitorovaní veľa strojov s vibráciami, je to, že vzniká množstvo záznamov, ktoré nie sú priamo užitočné pre operátora výrobnej linky. Väčšina signálov sa zobrazí maximálne raz, preto je zbytočné ich ukladať alebo prenášať vcelku. 

Zároveň na dosiahnutie maximálnej presnosti detekcie musí byť model strojového učenia trénovaný pre cieľové prostredie. Poruchy sú navyše pomerne zriedkavé udalosti, ktoré sa zvyčajne vyskytujú s odstupom niekoľkých mesiacov. Za týchto okolností je ťažké rýchlo získať dostatočne veľkú vzorku poruchových udalostí.

\section{Sledovanie prevádzkového stavu}
Existujú tri rôzne prístupy k údržbe strojov: reaktívny, preventívny a prediktívny.
Pri reaktívnej údržbe beží stroj až do úplného zlyhania a je prijateľná vtedy, keď sú dovolené krátke prestoje, je možná úplná a rýchla výmena pokazeného stroja za záložný, alebo ohrozenie zapríčinené poruchou je zanedbateľné. Preventívna údržba prebieha v pravidelných intervaloch odvodených od vopred určeného rozvrhu v alebo strednej doby medzi poruchami. Prediktívna údržba zlepšuje predvídateľnosť oproti reaktívnej údržbe a eliminuje plytvanie voči prílš obozretnej prevencii. Odstávka stroja je naplánovaná po zistení kritických hodnôt a po odhalení problematických komponentov.

Mechanické problémy počas prevádzky strojov spôsobujú v mnohých prípadoch vibrácie. Vibroakustická diagnostika sa preto považuje za jednu z najdôležitejších metód pri včasnej identifikácii porúch komponentov. Najbežnejšie sa vyskytujúcimi poruchami sú nevyváženosť, nesúososť, vôla, excentricita, deformácia, trhlina a nadmerné trenie. 

Symptómy porúch rotačných strojov sa prejavujú rôznymi frekvenčnými pásmami, ale väčšina je závislá od rotačnej rýchlosti súčiastky. Nevyváženosť, nevyváženosť a vôla sa bežne objavujú v frekvenciách do 300 Hz. Poruchy ložísk a prevodovky v neskorých štádiách vývoja sa prejavujú v rozsahu medzi 300 Hz a 1 kHz. Vyššie frekvencie do 10 kHz pomáhajú odhaliť poruchy ložísk v skorších štádiach rozvoja.

Postupy monitorovania stavu založené na vibráciách musia byť v súlade s normatívnymi smernicami ISO 20816 a ISO 13373. Normy sa týkajú umiestnenia meracích zariadení, zberu údajov, konvencií nastavenia úrovní závažnosti porúch. 


\section{Extrakcia a výber atribútov}
%TODO Page 14

\section{Diagnostické prístupy}


\section{Výskumné otázky}


\section{Návrh metód s MaFaulDa}


\section{Zber vibrácií v priemysle}


\section{Záver}


\clearpage
