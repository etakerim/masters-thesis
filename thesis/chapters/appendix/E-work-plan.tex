\thispagestyle{empty}
\chapter{Work plan}
\pagenumbering{arabic}
\renewcommand*{\thepage}{E-\arabic{page}}

\section{Summer semester - DP1}

\begin{table}[h!]
\def\arraystretch{1.25}
\begin{tabular}{|l|p{12cm}|}
\hline
\textbf{Period} & \textbf{Work}                                                                                                                                                                                                                         \\ \hline
\nth{1} week         & Consultation with the supervisor on directions of the future work based on literature review during the previous semester.
\\ \hline
\nth{2} week         & Outline the key sections of the analysis part in the thesis.
\\ \hline
\nth{3} week         & Match supporting literature with analysis sections. Further investigation on the feature engineering methodology in CbM.
 \\ \hline
\nth{4} week         & Summarize notes from condition monitoring articles and video recordings of tutorials and conferences.
 \\ \hline
\nth{5} week         & Research transformation of a vibration signal to feature space using time-frequency, harmonic, and energy statistical metrics. Progress report meeting with the supervisor.
 \\ \hline
\nth{6} week         & Find articles and take notes about unsupervised and semi-supervised techniques in streaming data for machinery diagnostics.
 \\ \hline
\nth{7} week         & Narrow down a wide variety of applicable methods for signal decomposition.
 \\ \hline
 \nth{8} week         & Exploratory analysis on evaluation datasets. Progress report meeting with the supervisor on the topic of related work.
 \\ \hline
 \nth{9} week         & Organize detailed outline out of notes gathered during literature research. 
 \\ \hline
  \nth{10} week         & Write up the problem analysis about condition monitoring and evaluation datasets.
 \\ \hline
  \nth{11} week         & Write up the analysis section about feature engineering.
 \\ \hline
  \nth{12} week         & Write up the analysis section about machine learning diagnostics and consult the final choice of methods in the analysis section.
 \\ \hline
\end{tabular}
\end{table}

\clearpage
\newpage
\section{Winter semester - DP2}

\begin{table}[h!]
\def\arraystretch{1.25}
\begin{tabular}{|l|p{12cm}|}
\hline
\textbf{Period} & \textbf{Work}                                                                                                                                                                                                                         \\ \hline
\nth{1} week         & Semester kickoff meeting to set goals, experiments and discuss the status of collaborations with partners.
\\ \hline
\nth{2} week         &  Arrange collaboration with an alternative industry partner. Prepare a checklist for the technical inspection of machinery. Construct a device for exploratory measurements.
\\ \hline
\nth{3} week         & Technical inspection of air conditioning units in the data center. Feature extraction step to calculate features from MaFaulDa.
 \\ \hline
\nth{4} week         & Consultation about plan for machine to measure in the data center and better device for measurement.
 \\ \hline
\nth{5} week         &  In feature selection using various metrics to determine sets of best features in the MaFaulDa.
 \\ \hline
\nth{6} week         & Feature selection used in k-NN multiclass and binary classificator.
 \\ \hline
\nth{7} week         & Explore incremental learning k-NN algorithm with MaFaulDa dataset. Consultation and status report.
 \\ \hline
 \nth{8} week         & Shorten analysis chapter about wavelets. Look at clustering detection incremental learning in MaFaulDa.
 \\ \hline
 \nth{9} week         &  Refactor separate trials in feature selection to integrate them into the pipeline for k-NN validation. Consultation to discuss progress.
 \\ \hline
  \nth{10} week         & Include incremental learning in analysis. Experiment with gradual feature selection in incremental learning.
 \\ \hline
  \nth{11} week         & Consultation in preparation for machine inspections. Preliminary measurements of fan, compressors, and pump. Desihn the firmware for new datalogger.
 \\ \hline
  \nth{12} - \nth{15} week         &  Write up design chapter: research questions and visualization export. Finish writing chapter on design and implementation.
 \\ \hline
\end{tabular}
\end{table}

The semester for DP2 was split into 3 periods. Feature engineering and ML model evaluation was planned from \nth{1} to \nth{4} week, and technical inspections and plan of measurement in weeks \nth{4} to \nth{8}.

In reality, these tasks switched order because between \nth{1} to \nth{5} week, we needed to acquire alternative partner and check their machiners for viability in our study. Only then from \nth{4} to \nth{11} week the main focus was on experiments with the MaFaulDa. 

Lastly, preliminary measurements were conducted, as was the plan. Requirements were presented for the new sensor device and its firmware.
\clearpage
\newpage

\section{Summer semester - DP3}
\begin{table}[h!]
\def\arraystretch{1.25}
\begin{tabular}{|l|p{12cm}|}
\hline
\textbf{Period} & \textbf{Work}                                                                                                                                                                                                                         \\ \hline
\nth{1} week  & Build and debug hardware of vibration logger with the consultant. 
\\ \hline
\nth{2} week & Implement and test firmware on standing fan for logging signal to SD card.
\\ \hline
\nth{3} week & First compressor (\nth{20} February) and water pump (\nth{27} and \nth{28} February) vibration measurements and their fault and frequency analysis. 
 \\ \hline
\nth{4} week & Replace two time domain features and evaluate on MaFaulDa and own dataset. Repeated measurement of compressors (\nth{5} March).
 \\ \hline
\nth{5} week & Perform PCA analysis of whole feature sets. Consultation about methods for own dataset evaluation.
 \\ \hline
\nth{6} week  & Consultation with experts in SjF STU. Formulate research goals for scientific paper and create graphs of the model results. Experiment with k-value and number of features to reach better accuracy. Repeated measurement of compressors (\nth{19} March).
 \\ \hline
\nth{7} week & Repeated measurement of water pumps (\nth{26} and \nth{27} March). Request and download data about pumps from KSB Cloud.
 \\ \hline
 \nth{8} week & Write up complete paper and submit to IIT.SRC student conference after feedback from supervisor.
 \\ \hline
 \nth{9} week & Refactor jupyter notebooks nad apply different labeling strategy applied in paper to incremental learning.
 \\ \hline
  \nth{10} week & Write up implementation chapter of the thesis.
 \\ \hline
  \nth{11} week & 
 \\ \hline
  \nth{12} - \nth{15} week &  
 \\ \hline
\end{tabular}
\end{table}

The aim of final semestre was to gather own dataset from compressors and water pumps with custom data logger. The k-NN model was validated with diffrent hyperparameters.


\clearpage
