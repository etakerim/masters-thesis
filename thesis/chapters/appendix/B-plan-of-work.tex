\thispagestyle{empty}
\chapter{Work plan}
\pagenumbering{arabic}
\renewcommand*{\thepage}{B-\arabic{page}}

\section{Summer semester - DP1}

\begin{table}[h!]
\def\arraystretch{1.25}
\begin{tabular}{|l|p{12cm}|}
\hline
\textbf{Period} & \textbf{Work}                                                                                                                                                                                                                         \\ \hline
\nth{1} week         & Consultation with the supervisor on directions of the future work based on literature review during the previous semester.
\\ \hline
\nth{2} week         & Outline the key sections of the analysis part in the thesis.
\\ \hline
\nth{3} week         & Match supporting literature with analysis sections. Further investigation on the feature engineering methodology in CbM.
 \\ \hline
\nth{4} week         & Summarize notes from condition monitoring articles and video recordings of tutorials and conferences.
 \\ \hline
\nth{5} week         & Research transformation of a vibration signal to feature space using time-frequency, harmonic, and energy statistical metrics. Progress report meeting with the supervisor.
 \\ \hline
\nth{6} week         & Find articles and take notes about unsupervised and semi-supervised techniques in streaming data for machinery diagnostics.
 \\ \hline
\nth{7} week         & Narrow down a wide variety of applicable methods for signal decomposition.
 \\ \hline
 \nth{8} week         & Exploratory analysis on evaluation datasets. Progress report meeting with the supervisor on the topic of related work.
 \\ \hline
 \nth{9} week         & Organize detailed outline out of notes gathered during literature research. 
 \\ \hline
  \nth{10} week         & Write up the problem analysis about condition monitoring and evaluation datasets.
 \\ \hline
  \nth{11} week         & Write up the analysis section about feature engineering.
 \\ \hline
  \nth{12} week         & Write up the analysis section about machine learning diagnostics and consult the final choice of methods in the analysis section.
 \\ \hline
\end{tabular}
\end{table}

\clearpage
\newpage
\section{Winter semester - DP2}

\begin{table}[h!]
\def\arraystretch{1.25}
\begin{tabular}{|l|p{12cm}|}
\hline
\textbf{Period} & \textbf{Work}                                                                                                                                                                                                                         \\ \hline
\nth{1} week         & Semester kickoff meeting to set goals, and experiments and discuss the status of collaborations with partners.
\\ \hline
\nth{2} week         &  Arrange collaboration with an alternative industry partner. Prepare a checklist for the technical inspection of machinery. Construct a device for exploratory measurements.
\\ \hline
\nth{3} week         & Technical inspection of air conditioning units in the data center. Feature extraction step to calculate features from MaFaulDa.
 \\ \hline
\nth{4} week         & Consultation about plan for machine to measure in the data center and better device for measurement.
 \\ \hline
\nth{5} week         &  In feature selection using various metrics to determine sets of best features in the MaFaulDa.
 \\ \hline
\nth{6} week         & Feature selection used in kNN multiclass and binary classificator.
 \\ \hline
\nth{7} week         & Explore incremental learning kNN algorithm with MaFaulDa dataset. Consultation and status report.
 \\ \hline
 \nth{8} week         & Shorten analysis chapter about wavelets. Look at clustering detection incremental learning in MaFaulDa.
 \\ \hline
 \nth{9} week         &  Refactor separate trials in feature selection to integrate them into the pipeline for kNN validation. Consultation to discuss progress.
 \\ \hline
  \nth{10} week         & Include incremental learning in analysis. Experiment with gradual feature selection in incremental learning. Fix the sampling rate issue in the provisional sensor unit.
 \\ \hline
  \nth{11} week         & Consultation in preparation for machine inspections. Preliminary measurements of fan, compressors, and pump.
 \\ \hline
  \nth{12} - \nth{13} week         &  Write up design chapter: research questions, visualization export. Finish writing design chapter.
 \\ \hline
\end{tabular}
\end{table}

The semester for DP2 was split into 3 periods. Feature engineering and ML model evaluation was planned from \nth{1} to \nth{4} week, and technical inspections and plan of measurement in weeks \nth{4} to \nth{8}.

In reality, these tasks switched order because, between \nth{1} to \nth{5} week, we needed to acquire alternative partner and check their machiners for viability in our study. Only then from \nth{4} to \nth{11} week, main focus was on experiments with the MaFaulDa. 

Lastly, preliminary measurements were conducted, as was the plan. Requirements were presented for the new sensor device, and a chapter about design was written.
\clearpage
\newpage

\section{Summer semester - DP3}
\begin{table}[h!]
\def\arraystretch{1.25}
\begin{tabular}{|l|p{12cm}|}
\hline
\textbf{Period} & \textbf{Work}                                                                                                                                                                                                                         \\ \hline
\nth{1}  - \nth{2} week         & 
\begin{enumerate}
\itemsep0pt
\item Incorporate suggestions from DP2 defense. 
\item Develop and test firmware for designed sensor unit.
\end{enumerate}
\\ \hline
\nth{2}  - \nth{10} week         & 
\begin{enumerate}
\itemsep0pt
\item Long-term manual regular gathering of vibration data from compressors and pumps. 
\item Find optimal number of features. 
\item Try to increase kNN prediction metrics.
\item Put together incremental learning with feature selection. 
\item Explore incremental clustering in search of method that can help with labeling.
\end{enumerate}
\\ \hline
\nth{10} - \nth{12} week         & Conclude results from measurements in real environment.
 \\ \hline
\end{tabular}
\end{table}

\clearpage
