\thispagestyle{empty}
\setcounter{figure}{0}
\chapter{Digital medium}
\pagenumbering{arabic}
\renewcommand*{\thepage}{E-\arabic{page}}
\par Registration number of thesis in AiS: \RegNo
\par Contents of the digital medium:
\par Name of submitted archive directory: DP\_MiroslavHajek.zip
\par Notice: Digital medium has more than 1 GiB and is available to the thesis supervisor.
\par The MaFaulDa dataset is publically available at: \url{https://www02.smt.ufrj.br/~offshore/mfs/database/mafaulda/full.zip}

\begin{itemize}[noitemsep]
\item[\textbf{>}] \textbf{datasets} - raw vibration data from machinery, and reports that took long to calculate.
	\begin{itemize}[noitemsep]
	\item[\textbf{>}] \textbf{FluidPump.zip} - custom recorded dataset of vibrations from air compressors and water pumps
	\item[\textbf{>}] \emph{MAFAULDA.zip} - placeholder where the downloaded MaFaulDa dataset should be placed
	\item[\textbf{>}] \textbf{best-subset} - the subset of features chosen to three-member subsets most frequently
	\item[\textbf{>}] \textbf{features} - features extracted from datasets in multiple domains
	\item[\textbf{>}] \textbf{ksb-cloud} - data exported from KSB cloud monitoring of vibration rms velocity and frequency waveform for chosen dates.
	\item[\textbf{>}] \textbf{misc-fluid-pump} - other non-standard measurements from pumps such as during speed up, slow down, or noise.
	\item[\textbf{>}] \textbf{knn-accuracy-distribution} - 
	\item[\textbf{>}] \textbf{knn-incremental-accuracy} - The incremental k-NN model accuracy after each sample for various lengths of tumbling window and gaps between labels 
	\item[\textbf{>}] \textbf{standing-fan} - audio for an estimate of fan rotational speed and measurements done in firmware verification on the back, side, and front of the fan.
	\end{itemize}
\item[\textbf{>}] \textbf{docs} - the documentation generated from source code comments  using automated tool
	\begin{itemize}[noitemsep]
	\item[\textbf{>}] \textbf{html-notebooks} - export of Jupyter notebooks for all basic configurations in HTML format
	\item[\textbf{>}] \textbf{doxygen} - documentation of \emph{firmware} 
	\item[\textbf{>}] \textbf{sphinx} - documentation of \emph{vibrodiagnostics} package
	\end{itemize}
\item[\textbf{>}] \textbf{firmware} - source code in \emph{main} directory for accelerometer data logger firmware written in C language with ESP-IDF SDK. The script \emph{bin2tsv.py} converts files saved to SD card in binary format to a tsv file.
\item[\textbf{>}] \textbf{notebooks} - Jupyter notebooks for data exploration, data analysis, and machine learning on provided datasets
\item[\textbf{>}] \textbf{vibrodiagnostics} - Python package of commonly used function in Jupyter notebooks for data processing
\end{itemize}