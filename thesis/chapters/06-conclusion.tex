\chapter{Conclusion} \label{section:conclusion}  
The focus of this master's thesis is on the optimal choice of trend indicators from vibration signals in an inexpensive industrial condition monitoring solution. The goal is to enable timely fault detection of machinery parts with as little input data as possible. We answer five proposed \textbf{research questions} (RQ) for those purposes.

The variables to describe machine behaviour come from mechanical properties of bearings, descriptive statistics, audio signal processing studies, and vibrodiagnostics technical standards: ISO 20816 and ISO 13373. The formulas compute a complete set of features that are ten features summarizing the waveform in the time domain (\emph{TD}) and eleven features describing spectral density estimation (\emph{FD}). The features are obtained in the axis of the jitter or are aggregated from all three spatial directions (\textbf{RQ1}).

The groups of three features are chosen in each domain to achieve even more pronounced data savings. We strive to keep the predictors that together have the most resemblance to the target variable. Feature selection metrics of the correlation coefficient, F statistic, mutual information, and their rank product are computed in supervised learning scenarios. The features are squished from multiple dimensions using the Euclidean norm. Lossy compression ratios attained are 2381:1 for all features and 25000:1 for six features from the MaFaulDa dataset. We managed to discard more than 99.995\% of the irrelevant data (\textbf{RQ2}).

The designed preprocessing pipeline creates the attributes from time series. The k-nearest neighbour classifier ascertains their relative fault detection power by five-fold cross-validation. The experiments for k-NN model evaluation include complete feature sets with one for each domain, the exhaustively enumerated feature subset combinations, and feature subsets chosen by filtering feature selection methods. 

We found that the chosen time-domain features have reached higher accuracies than the frequency-domain set because of fewer intercorrelations. An increase in the number of neighbours for k-NN used on complete sets lessens the model accuracy because class boundaries overlap and are relatively noisy. Above the k-value of thirteen, the trend for complete feature sets of accuracy decrease flattens out gradually. Predictors aggregated from tri-axis recording performed better in classification than those from one axis. 

The k-NN accuracy with different feature subset combinations pulled out of the complete sets has a standard deviation of 7\%, and it is 10\% within IQR. The feature selection methods can pick, in most cases, a group with accuracy above the upper quartile of accuracy distribution of every attribute combination with the same cardinality. Training k-NN on three principal components produced worse accuracy than filtering feature selection.

The ensemble feature selection methods as the rank product are necessary to balance the adverse effect of individual selection metrics. In the slight majority of 44\% of 216 dataset alterations, the rank product found the best-performing attributes closely followed by mutual information. The rank product also reaches the 97.50\% percentile in the accuracy distribution of the same-sized sets. Mutual information has a better distribution percentile by 2.84\% than the rank product \textbf{(RQ3)}. 

The number of features has a direct proportionality effect accuracy but at most 6\% increase is observed while moving from two to three features. The absolute best predictor triplets for MaFaulDa are in TD (accuracy of 85.47\%): zero-crossing rate, peak-to-peak, skewness, and in FD (accuracy of 87.52\%): spectral centroid, roll-off frequency, and entropy \textbf{(RQ3)}.

The vibrations were also collected for over a month with a custom assembled accelerometer data logger. The designated machines were air conditioning compressors and water pumps with motors in municipal pumping station. The data logger firmware reads samples from the accelerometer at 26.8 kHz and stores them onto a MicroSD card.

Collected dataset analysis confirmed the stationarity of the signal spectrum and the five-second long time series as satisfactory for feature extraction. The dataset serves as a practical demonstration of labeling potentially future faults from bearing defect frequency amplitudes. It also proved the immense difficulty of obtaining observations of faulty machinery parts in the field without purposefully damaging bearings and offsetting the shaft \textbf{(RQ4)}.

The considerable obstacle in an autonomous fault detection system deployment is the availability of labels for target variables. Annotations can be assigned belatedly or even never. Incremental learning k-NN algorithm on an unbalanced dataset of three member feature sets achieves at best 77\% accuracy with labels coming in 100 samples long tumbling windows. The accuracy of 67\% is reached for the online model when just 10\% of observations stay associated with the label. The comparable model trained in batch reaches an accuracy of 88\% \textbf{(RQ5)}.

\textbf{Future work} can extend various aspects and answer additional questions arising from the experiments in the thesis. The natural extension is to confirm that the results hold in other publicly available datasets of machinery faults, e.g. in CWRU. The other alteration is to use more and different starting base feature sets. Harmonic frequencies or wavelet coefficients can generate predictors that could be further reduced to the smallest possible subsets. 

Ensemble strategies inspired by classical election systems may be utilized in the task of feature selection. Different types of classifiers for online learning could be compared to k-nearest neighbours, e.g. Hoeffding tree. A data logger could be combined with the classifier model and a mechanism to annotate observation either automatically depending on the overall vibration level or manually during maintenance or after an unexpected failure.

The strategic aim would be to connect sensors on a large scale on multiple test rigs to train individual models on each bearing and combine them in, for example, a federative learning approach. Of course, vibration monitoring can be employed in other diagnostic tasks without direct expert knowledge, such as train wheel or rail degradation due to wear and tear.