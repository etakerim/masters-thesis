\chapter{Conclusion} \label{section:conclusion}  
In the thesis we focused on trend indicator selection for an inexpensive industrial condition monitoring solution from vibration signals. The goal is to enable timely fault detection of machinery parts with as little input data as possible. For that purpose, we answer four research questions.

Attributes extracted to describe machine behavior come mainly from descriptive statistics, audio signal processing studies, and vibrodiagnostics technical standards (ISO 20186 and ISO 13343). These formulas compute ten features summarizing the waveform in the temporal domain and eleven features characterizing spectral density estimation in axis of motion or all three spatial directions \textbf{(RQ1)}.

In order to achieve more pronounced data savings, we choose the subset of 3 features in each domain by keeping the ones with the most similarity to the target variable. Feature selection metrics of the correlation coefficient, F statistic, mutual information, and their rank product are applicable in supervised learning. The features are squished from multiple dimensions using the Euclidian norm. Lossy compression ratios attained are 2381:1 for all features and 25000:1 for 6 features in the MaFaulDa dataset. We managed to discard more than 99.995\% of irrelevant data \textbf{(RQ2)}.

Feature subsets are subjected after normalization to a k-nearest neighbor classifier that ascertains their relative fault detection power. Because of model overtraining with small k, the feature triplets equal or slightly outperformed the whole set of features on the validation set with accuracy up to 10\%. The time-domain set of features reach higher accuracies than frequency-domain set because of their smaller interdependency \textbf{(RQ3)}. 

The ensemble of feature selection with rank product produces the best model performance out of three combined in the majority of situations. No approach could find a triplet of predictors with an accuracy close to optimal one discovered exhaustively. 
%TODO wtf
Training on three principal components produced better accuracy than filtering feature selection, but PC mapping onto original trend indicators is unclear even in loading plots \textbf{(RQ3)}.

The considerable obstacle in an autonomous fault detection system deployment is the availability of labels for target variables. Annotations can be assigned belatedly or even never. Incremental learning k-NN model on an unbalanced dataset on the whole feature set achieves at best 90\% accuracy with immediate feedback, 85\% with labels coming in 250 long tumbling windows, and 82\% with just 25\% of observations associated with the label. The comparable model trained in batch reaches an accuracy of 98\% \textbf{(RQ4)}.

The vibration signals will be gathered on compressors in air conditioning units and water pumps in municipal pumping stations. We will develop firmware for sensor units capable of sampling accelerometers and saving measurements onto SD cards. The challenge awaits us in labeling samples themselves as it requires substantial expert knowledge. 

%Abstract: 
% Analysis toolkit (library) for vibrodiagnostics in Python

% Interdisciplinary methodological overview of implementing intelligent monitoring system according to mechanical engeneering standards of technical diagnostics 

%TODO
% Future work
% - long term dataset from machinery until failure or propose way for factories to artifically train model for new machinery - make the system machine and placement independent
% - more diffenent feature set from one recoding - in windows
% - combine feature selection with incremntal learining
% - Different election methods and metrics
% - Find other ML model also suitable HoeffdingTreeClassifier - losing original data for labeling - mechnaism of transforming samples to model represntatin
% - Monitor more machines and combine their result with federated learning
% - Monitor faults of different equipment - moving wheels of the train or train tracks

% Main points of the thesis assigments namely effectivness representation of machinery faults in succint fashion and performance in machine learining diagnostics models has been compared extensively.