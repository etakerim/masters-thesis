\chapter{Conclusion} \label{section:conclusion}
The focus of this master's thesis is on the optimal choice of trend indicators from vibration signals in an inexpensive industrial condition monitoring solution. The goal is to enable timely fault detection of machinery parts with as little input data as possible. We answer \textbf{five research questions} for those purposes.

\textbf{The first question} is about finding numerical features that can represent machinery behaviour for accurate diagnosis. The technical standards in vibrodiagnostics and the descriptive statistics for time-varying phenomena are valuable in that regard. Vibrations have to be recorded at a high sampling rate. The researched formulas narrow down this massive data stream to groups of a few features. We define ten features that summarize the waveform in the time domain (\emph{TD}) and eleven features that describe its spectral density estimation (\emph{FD}). The features are either just taken from the axis of the jitter, or aggregated from all three spatial directions.

\textbf{The second question} answers the effort to achieve even more pronounced savings in transmission goodput. Groups of three predictors are chosen from each base feature set that together have the most similarity to the target variable. Feature selection scores of the correlation coefficient, F statistic, mutual information, and their rank product are computed in supervised learning scenarios. Lossy compression ratios attained are 2381:1 for all features and 25000:1 for six features from the MaFaulDa dataset. We managed to discard more than 99.995\% of the irrelevant data.

\textbf{The third question} asks about the accuracy of the machine learning model for machinery fault diagnostics. The major constraint put on the model is to provide inference from reduced sets of predictors. The k-nearest neighbours classifier is utilized because it works on a simple to-explain assumption that proximate data points mean the same type of fault. Additionally, the k-NN retains individual historical observations and can be applied in an incremental learning approach.

We implement the \textbf{data processing pipeline} that transforms the MaFaulDa dataset for multiclass classification with the k-NN algorithm and a tiny number of features. The fault labels with severity are composed of the file's path from the dataset structure. The features are extracted from vibration time series after digital filtering. Then, the features are arranged into multiple configurations: complete feature sets, enumerated feature subset combinations, and feature subsets chosen by filtering feature selection methods. The accuracy results are validated by five-fold cross-validation.

We found that for this particular dataset, the chosen time-domain features have reached higher accuracies than the chosen frequency-domain set because of fewer intercorrelations. An increase in k-neighbours lessens the model accuracy because class boundaries overlap and are relatively noisy. Predictors aggregated from tri-axis recording performed better in classification than those from just one axis.

The feature selection methods can pick a group of predictors with accuracy above the upper quartile of the statistical distribution of accuracy values for the same-sized sets. The rank product method found the best-performing attributes in the majority of 44\% of scenarios, closely followed by mutual information. The rank product for the predictor's triplets reaches the 97.50\% percentile from the accuracy distribution.

The number of features has a direct proportionality effect on accuracy. The absolute best predictor triplets for MaFaulDa are in TD (accuracy of 85.47\%): zero-crossing rate, peak-to-peak, skewness, and in FD (accuracy of 87.52\%): spectral centroid, roll-off frequency, and entropy.

\textbf{The fourth question} investigates machinery behavior in an industrial environment using signal processing techniques. The vibrations were collected a month apart by an accelerometer data logger according to the specification. The designated machines were air conditioning compressors and water pumps with motors in municipal pumping station. We implemented a data logger's firmware that reads samples after a button press from the accelerometer at 26.8 kHz and stores them onto a memory card.

Analysis of the collected dataset confirms the stationarity of the signal spectrum under constant load and assures that five-second bursts are indeed satisfactory for feature extraction. The dataset provides a practical exercise in labeling potential future faults from bearing defect frequency amplitudes. It also demonstrates the immense difficulty of obtaining observations of faulty machinery in the field without purposefully damaging it.

\textbf{The fifth question} is concerned with incremental machine learning because the considerable obstacle in the deployment of an autonomous fault detection system is the availability of annotations that can be assigned belatedly or even never. Incremental k-NN algorithm on an unbalanced dataset for three predictors achieves at best 77\% accuracy with labels in tumbling windows of hundred samples. The accuracy of 67\% is reached for the online model when just 10\% of observations stay associated with the label. The comparable model trained in batch reaches an accuracy of 88\%.

\textbf{We conclude} that an inexpensive industrial condition monitoring solution with low data rates is indeed possible. It uses the available accelerometer sensors and feature discovery techniques. According to the analysis, the system could diagnose the machine's health with high accuracy in a local model on the edge device.

\section{Future work}
Future work would focus on the deployment of our machine learning model on a large scale to sensor units and setting up the necessary infrastructure. The components for such an endeavour are described in the analysis for incremental learning. The way to annotate observations could be introduced, either automatically depending on the overall vibration level or manually during maintenance or after an unexpected failure.

The natural research extension of the thesis is to confirm that the results hold in other publicly available datasets of machinery faults, e.g. in CWRU. The other alteration is to use more and different starting base feature sets. Harmonic frequencies or wavelet coefficients can generate predictors that could be reduced further to the smallest possible subsets. 

Ensemble strategies inspired by classical election systems may be utilized in the task of feature selection. Different types of classifiers for online learning could be compared to k-nearest neighbours, e.g. Hoeffding tree. A combination of local models from multiple sensors is desirable, e.g. in a federative learning approach.