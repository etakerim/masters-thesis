\thispagestyle{empty}
\section*{Annotation}
\UniversityEN \\
\uppercase{\FacultyEN}
\vspace{-8pt}
{\setlength{\mathindent}{0cm}
\begin{align*}
&\text{Degree course:} && \text{\StudyProgrammeEN} \\
&\text{Author:} && \text{\AuthorEN} \\
&\text{\ThesisEN:} && \text{\TitleEN} \\
&\text{Thesis supervisor:} && \text{\SupervisorEN} \\
&\text{Departmental advisor:} && \text{\DepartmentalAdvisorEN} \\
&\text{Consultant:} && \text{\ConsultantEN} \\
&\text{\DateEN}
\end{align*}}

Master's thesis focuses on condition monitoring and predictive maintenance of rotating machinery by employing feature discovery and statistical classification. 

Trend indicators are derived from vibration signals to form summarizing numerical attributes. The similarity of features to a dependent variable that describes fault type is assessed by supervised feature selection metrics individually and in the ensemble. The accuracy of fault classification using the k-nearest neighbour algorithm is evaluated on the best feature subsets. The goal is to conserve the data rates of the monitoring solution. The incremental learning models are compared to their batch counterparts when true labels are delayed or missing. An extendible software toolkit of vibrodiagnostics is implemented as a Python package.

Classification metrics are evaluated on the MaFaulDa dataset, and lossy compression ratios are expressed. Our accelerometer data logger collects a novel dataset of machinery behavior for compressors and pumps in the industrial environment. The contribution of this work lies in identifying the viability of using a few quantities to describe not just the presence of the machine defect but also its cause.
\emptypage 

\thispagestyle{empty}
\section*{Anotácia}
\University \\
\uppercase{\Faculty}
\vspace{-8pt}
{\setlength{\mathindent}{0cm}
\begin{align*}
&\text{Študijný program:} && \text{\StudyProgramme} \\
&\text{Autor:} && \text{\Author} \\
&\text{\Thesis:} && \text{\Title} \\
&\text{Vedúci diplomovej práce:} && \text{\Supervisor} \\
&\text{Pedagogický vedúci:} && \text{\DepartmentalAdvisor} \\
&\text{Konzultant:} && \text{\Consultant} \\
&\text{\Date}
\end{align*}}

Diplomová práca sa zaoberá monitorovaním stavu rotačných strojov a ich prediktívnou údržbou prostredníctvom techník výberu atribútov a štatistickej klasifikácie.

Z vibračných signálov sú odvodené trendové indikátory, ktoré tvoria sumarizujúce číselné atribúty. Podobnosť atribútov so cieľovou premennou popisujúcou typ poruchy sa posudzuje individuálne a súborovo metrikami výberu atribútov v učení s učiteľom.  Presnosť klasifikácie porúch strojov algoritmom k-najbližších susedov sa vyhodnocuje na podmnožinách najlepších atribútov. Tým sa znižujú požiadavky na objem prenosu dát v monitorovacom riešení. Modely postupného učenia sú porovnané s učením v dávkach, v situáciach kedy je expertné označenie prevádzkového stavu oneskorené alebo chýbajúce. Rozšíriteľná sada funkcií pre vibrodiagnostiku je implementovaná ako balík v jazyku Python.

Metriky klasifikácie sú vyhodnotené v scenároch na dátovej sade MaFaulDa a tiež sú vyjadrené stratové kompresné pomery. Nami vytvorený data logger nazhromaždil novú dátovú sadu z kompresorov a čerpadiel v priemyselnom prostredí. Prínos tejto práce spočíva v overení možnosti použitia iba pár veličín na opis nielen prítomnosti poruchy, ale aj jej príčiny.
\emptypage

