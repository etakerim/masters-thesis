% Poďakovanie
\thispagestyle{empty}
\vspace*{\fill}

\begin{center}
\settowidth\longest{\itshape ---Wir müssen wissen.---}
\parbox{\longest}{
  \hrulefill\hspace{0.2cm} \decofourleft\decofourright \hspace{0.2cm} \hrulefill\par
  \raggedright{
  \itshape
  	Wir müssen wissen. \\ Wir werden wissen.\par
  }   
  \raggedleft{--- David Hilbert}\par
  \hrulefill\hspace{0.2cm} \decofourleft\decofourright\hspace{0.2cm} \hrulefill\par
}
\end{center}

\vspace*{\fill}
\section*{Poďakovanie}
{\linespread{1.0}\small Napriek tomu, že tento text je len kvapka v mori záverečných prác, pre mňa bol proces tvorby omnoho viac než zachytávajú slová na stránkach. Srdečná vďaka patrí školiteľovi Ing.~Marcelovi Balážovi,~PhD. a konzultantovi Ing.~Lukášovi Doubravskému z firmy R-DAS,~s.r.o. Obaja boli otvorení mojim všemožným nápadom a podporovali ma v ich realizácii aj cez mnohé ťažkosti. Za praktický expertný pohľad na vibrodiagnostiku vďačím prof. Ing.~Stanislavovi Žiaranovi,~CSc. a Ing.~Ondrejovi Chlebovi,~PhD. zo Strojníckej fakulty STU. Cením si ústretovosť Bratislavskej vodárenskej spoločnosti,~a.s. v sprístupnení čerpadiel na merania a súvisiacich podkladov, a to konkrétne Ing.~Peterovi Csókovi a Ing.~Petrovi Kmeťkovi. Rovnako ďakujem za ochotu firme VNET,~a.s., konkrétne Michalovi Országovi a Mgr.~Vladimírovi Kupčovi za odporúčania možných strojov na merania v dátových centrách a sprístupnenie kompresorov. Prácu by som rád venoval rodine a kolegom-kamarátom, ktorí stáli pri mne na ,,šialenej akademickej dráhe'' a podnetné diskusie s nimi prispeli aj k môjmu pohľadu na odborné problémy. Nuž a v konečnom dôsledku som mal niekedy len viac šťastia ako rozumu.}
\vspace{3cm}