\chapter{Introduction}
The industry experiences a shift in traditional asset operational status evaluation and utilization. The rise of Industry 4.0 means greater automation and robotization of the production halls to achieve optimal usage of available resources. The secondary aspect in the enterprises' endeavor, but not less important, is to keep track of the equipment's wear and tear. The corrective action, be it repair or replacement, should be done on time in response to the key indicators.

The goal is to preserve required safety and production efficiency while extending the useful life of rotating mechanical parts. In the factories and logistics where this sort of equipment is vital, there is a rising interest in the ability to watch in real time the machine's health status. Proactive fault diagnosis is imperative to initiate a repair without adding unnecessary costs.

Vibrations are a non-intrusive way to sense and record eventually fatal deficiencies right at the onset. The experts use it to distinguish faulty states and to identify the malfunction's root causes. In critical circumstances such as is the case of the large turbines in power plants, the precautions leading to regular machinery check-ups are already in place. The monitoring solution has to be sufficiently independent, reliable, and accurate to achieve wider acceptance and spread.

The main issue to consider in large-scale machinery monitoring with vibrations is that there are lots of uninformative streams of samples not directly useful for the production line operator. The dashboard must aggregate these flows into trend variables with severity levels categorized based on industrial standards. The majority of signals are viewed once at the maximum. Therefore, to store or even transmit them from the edge device in its entirety would be wasteful. The complex overview of the mechanical equipment status is attainable only when agent devices and sensors are cheap enough with a long lifespan powered out of the battery pack. Preferably these devices should also remain physically small to reduce the additional clutter in the factory.

Attempted machine learning and deep learning approaches have the crucial impediment that the construction of every single machine is unique to some extent because of tolerances and variable load during regular operation. The model has to be trained for the target environment to achieve the ideal performance. In addition, failures are relatively rare events that usually occur several months apart. In these circumstances, it is hard to quickly obtain a large enough sample of fault events. Novelty detection is a technique that can be applied in this case.

This progress report on the master's thesis \rom{2} is organized followingly. The problem analysis chapter \ref{chapter:problem-analysis} begins by exploring the mechanical maintenance approaches and industry standards on common fault identification. Vibration monitoring starts with a section on preprocessing that summarizes basic filters. The process of transforming raw samples into features and their meaningful selection is covered in automatic fault pattern recognition with machine learning models trained in offline and online contexts. An overview follows of the most common machinery fault databases. 

Chapter \ref{chapter:design} on solution design starts by posing research questions and establishing goals. Then, the MaFaulDa dataset statistical properties and labeling procedure are described on the original samples as well as on extracted attributes. Data rates with feature selection are derived based on the MaFaulDa. In the section about classification models, we compare the performance for various feature sets in the k-nearest neighbors algorithm. Finally, in preparation for the evaluation part, the sensor device is proprosed which is suitable for measurement of designated machines in the industrial environment.
