\chapter{Introduction}
The industry is experiencing a shift in traditional asset operational status evaluation and utilization. The rise of Industry 4.0 means greater automation and robotization of the production halls to achieve optimal usage of available resources. The secondary aspect in the enterprises' endeavour, but not less important, is to keep track of the equipment's wear and tear. The corrective action, be it repair or replacement, should be done on time in response to the key indicators.

The goal is to preserve required safety and production efficiency while extending the useful life of rotating mechanical parts. In the factories and logistics where this equipment is vital, there is a rising interest in the ability to watch the machine's health status in real time. Proactive fault diagnosis is imperative to initiate a repair without adding unnecessary costs.

Vibrations are a non-intrusive way to sense and record eventually fatal deficiencies right at the onset. The experts use it to distinguish faulty states and to identify the malfunction's root causes. The precautions leading to regular machinery check-ups are already in place in critical circumstances, as is the case for large turbines in power plants. The monitoring solution has to be sufficiently independent, reliable, and accurate to achieve wider acceptance and spread.

The main issue to consider in large-scale machinery monitoring with vibrations is the presence of many uninformative streams of samples not directly usable for the production line operator. The dashboard must aggregate these flows into trend variables with severity levels categorized based on industrial standards. The majority of signals are viewed once at the maximum. Therefore, to store or even transmit them from the edge device in its entirety would be wasteful. The complex overview of the mechanical equipment status is attainable only when agent devices and sensors are cheap enough with a long lifespan powered out of the battery pack. The devices should preferably also remain physically small to reduce the additional clutter in the factory.

Attempted machine learning and deep learning approaches have the crucial impediment that the construction of every single machine is unique to some extent because of tolerances and variable load during regular operation. The model has to be trained for the target environment to achieve the ideal performance. In addition, failures are relatively rare events that usually occur several months apart. In these circumstances, it is difficult to obtain a large enough sample of fault events. Novelty detection is a technique that can be applied in this case.

The master's thesis opens with a chapter on problem analysis~\ref{chapter:problem-analysis} where the explanation is given of mechanical maintenance approaches and industry standards on common fault identification. Vibration monitoring starts with a section on preprocessing that summarizes digital filters. The process of transforming raw samples into features and their meaningful selection is covered in automatic fault pattern recognition with machine learning models trained in offline and online contexts. An overview follows of the most common machinery fault databases.

Chapter~\ref{chapter:design} about solution design poses research questions and establishes goals. The statistical properties and labeling procedure of the MaFaulDa dataset are described on the original samples as well as on extracted attributes. The data collection procedure and accelerometer data logger for industrial water pumps and air compressor is outlined. Data reduction rates with feature selection are derived based on the MaFaulDa dataset. Chapter~\ref{chapter:implementation} describes software libraries and tools used in conducting model evaluation and data logger firmware implementation.

The chapter~\ref{chapter:evaluation} on evalution validates the classification accuracy of feature sets from MaFaulDa and strategies for their subset choice in the k-nearest neighbour model. The vibration signals in the realistic environment are also analysed and recommendations for practical application and further study are discussed.
