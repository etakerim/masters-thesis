\chapter{Design}

\section{Research questions}
\begin{enumerate}
\item \emph{Which time-frequency features can be extracted from vibrational signals to provide an accurate record of machinery faults?}
\item \emph{What are the savings in transmission bandwidth when chosen signal features are used in comparison to raw sampled measurement or lossless compression techniques?}
\item \emph{How can the machinery faults be continuously identified based on collected events?}
\end{enumerate}

\section{Infrastructure}
 \begin{itemize}
\item \textbf{Input:} Samples from three-axis MEMS accelerometers, RPM tachometer, Noise background
\item \textbf{Output is either:} machine overall status, type of fault,
\item \textbf{Output for domain expert}: Control chart of trend features
\end{itemize}

\begin{enumerate}
\item MEMS accelerometers are placed on at least two distinict measurement points in two perpendicular axis and one sensor in base for denoising. Rotational speed is captured at the same time too.
\item Sensors are triggered in regular intervals (every 15 minutes) to collect sample recording from the band saw.
\item \textbf{Features} are computed and compared to recent measurements. If there is an statistically significant change the whole summary is send, otherwise keepalive notification is send.
\item Set of features - at the explatory stage all are computed and stored locally. Use features in conjunction with transformation techniques and diagnostics model to reduce the set of features to minimum with the k-fold cross-validation.
\item Database stores history of measurements
\item \textbf{Diagnosis panel runs clustering} with introduction of annotations to notify the operator about observed fault and imminent failure of the machine.
\end{enumerate}


