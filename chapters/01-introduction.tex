\chapter{Introduction}
Manufacturing is experiencing a shift in the traditional practices of asset operational status evaluation and utilization. The rise of Industry 4.0 means greater automation and robotization of the production halls to achieve optimal usage of available resources. The secondary aspect in the enterprises' endeavor, however not less important, is to keep track of the equipment wear and tear. The corrective action be it repair or replacement should be taken on time in response to the key indicators. 

The goal is to preserve required safety and production efficiency when extending the useful life of machine moving parts. In the factories and logistics where this sort of equipment is vital, there is a rising interest in the ability to monitor in real-time the health of the machines and to proactively diagnose the fault to repair it without adding unnecessary costs. 

Vibrations are the most nonintrusive way with which such faults can be sensed. The experts use it to distinguish faulty states and to identify the malfunction's root cause. In critical circumstances such as in the case of the large turbines in the power plants, the precautions leading to regular machinery check-ups are already in place. To reach wider acceptance and spread, the monitoring solution has to be sufficiently independent, reliable, and as self-sufficient as the model design allows it to be.

The main issue to consider in large-scale machinery monitoring using vibrations are lots of uninformative streams of samples not directly useful for the production line operator. The dashboard must aggregate these flows into trend variables with severity levels categorized based on industrial standards. The majority of signals are viewed once at the maximum therefore to store or even transmit them from the edge device in its entirety would be wasteful. The complex overview of the mechanical equipment status is attainable only when agent devices and sensors are cheap enough with a long lifespan on battery power and preferably remain physically small to reduce the additional clutter.

Attempted machine and deep learning approaches have the crucial impediment that the construction of every single machine is unique to some extent because of tolerances and variable load. The model must be trained specifically for the target environment to achieve the ideal performance. In addition, the failures are relatively rare events occurring usually in the span of multiple months. In these circumstances, it is hard to obtain a large enough sample of fault events quickly. Novelty detection is a technique that can be applied in this case.

The thesis is organized in the following manner. In the first chapter of analysis in section 1 we explore the mechanical maintenance approaches and industry standards on common fault identification. Then section 2 is all about measuring vibrations and transforming them into features meaningful in automatic fault pattern recognition. In section 3 we delve into modes of diagnosis based on reduced relevant indicators. Section 4 deals with evaluation datasets used to determine computational requirements
on IIoT infrastructure. Chapter 2 defines data format and proposes processing steps to diagnose the imminent failure and different fault types. The approach taken is evaluated and validated in Chapter 6. 
  