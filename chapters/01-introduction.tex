\chapter{Introduction}
\textbf{Following is very coarse draft!}
Manufacturing is experiencing a shift in the traditional asset operational status evaluation and utilization. The goal is to promote safety and production efficiency when the useful life of machine moving parts is extended. In the factories and logistics where this sort of equipment is vital, there is a trend to be able to monitor the health of the machinery parts and above that to diagnose the fault in time to repair it without additional costs. Vibrations are the most nonintrusive way where such faults can be sensed and appear distinctly for an analyst to identify the root cause of the malfunction.

In critical circumstances, such measurements are already in place in some form, but in order to reach wider acceptance, and not remain just a quirk/trend system has to be sufficiently independent, reliable, and as self-sufficient as the model design allows it to be.

The thesis is structured in a following manner. In Chapter 1 we explore the theoretical (analytical) model view, mechanical maintenance approaches, and industry standards where common fault identification is described. Chapter 2 is all about taking vibration measurements (procuring) and transforming them into features meaningful in automatic fault pattern recognition. The methods for ranking are reviewed to obtain the most important and correlated features with machine health status. We delve into modes of diagnosis based on reduced relevant indicators in chapter 3. Chapter 4 takes a look into IoT communication infrastructure limiting the data throughput and devices that can be deployed (accommodate) in the factory environment. Chapter 5 defines measurement vectors and proposes processing steps to diagnose the reoccurring failure, RUL (remaining useful life), and fault types. The approach taken is evaluated and deployed in Chapter 6. 
  
