\section{Condition monitoring}
Why monitor with vibrations, 
Wear process curve, Fault - failure, Levels are dependant on load and lifecycle stage

\subsection{Maintenance strategies}
\paragraph{Reactive}
\paragraph{Preventive}
\paragraph{Predictive}
- value for the factory and assess risks asociated with potencial fault to assign them importance


% Automatic Anomaly Detection in Vibration Analysis Based on Machine Learning Algorithms
\cite{torres_automatic_2022}
% Practical Machinery Vibration Analysis and Predictive Maintenance
\cite{scheffer_practical_2004}
% Machinery Condition Monitoring: Principles and Practices
\cite{mohanty_machinery_2015}
% Techniques for Vibration Monitoring
\cite{davies_techniques_2012}
% Technická diagnostika
\cite{ziaran_technicka_2013}
% The experimental application of popular machine learning algorithms on predictive maintenance and the design of IIoT based condition monitoring system
\cite{cakir_experimental_2021}
% Vibration Guide
\cite{noauthor_vibration_2000}

% Bandsaws
% Vibration of bandsaws
\cite{lengoc_vibration_1990}
% Study on Online Detection and Fault Diagnosis of Band Saw Equipment
\cite{chen_study_2014}

\paragraph{Diagnosis indicators} - what we montitor? - when it fails and early signs of faulty component
- RUL (Remaining useful life) models - disadvantage lot of similiar machines (homogenous) or lots of runns until failure
\begin{itemize}
\item Similarity
\item Degradation - used by standards
\item Survival
\end{itemize}

\subsection{Vibration fault types}
- frequency ranges 1 - 300 Hz (shaft), 300 - 1000 Hz, 1000 - 10000 Hz (early bearings)
- Base analytical models: Jeffcott rotor - rotor dynamics, Bearnings model
- Resonance frequencies of each part - machine must run at speeds not aligned with resonance frequencies - Campbell diagram - task for mechanical engineers
- Faults - reasons and frequency content
\begin{itemize}
\item Synchrounous response - based on RPM
\item Mass unbalance
\item Misalignment
\item Eccentricity
\item Bent or bow shaft
\item Cracked shaft
\item Rotor rubs - friction
\item Looseness
\item Auxiliery mechanical systems: Gearbox, Bearings, Belt 
\end{itemize}

\subsection{Technical standards}
\paragraph{ISO 20816}
% ISO 20816-1:2016 - Mechanical vibration - Measurement and evaluation of machine vibration - Part 1: General guidelines
\cite{noauthor_iso_2016}
Part 1
\begin{itemize}
\item Measurement units - displacement, velocity, acceleration
\item RMS, and max. amplitude = severity
\item Measurement points for sensors (axial, radial) - image, and 45 degrees
\item Evaluation zones - A, B, C, D - Severity chart (Annex B) - Degradation model
\item Opeartional limits - Alarm, Trips
\end{itemize}

\paragraph{ISO 13373}
% ISO 13373-1:2002 - Condition monitoring and diagnostics of machines - Vibration condition monitoring - Part 1: General procedures
\cite{noauthor_iso_2016}
% ISO 13373-2:2016 - Condition monitoring and diagnostics of machines - Vibration condition monitoring - Part 2: Processing, analysis and presentation of vibration data
\cite{noauthor_iso_2016-1}

\cite{jack_d_frequency_nodate}
\begin{itemize}
\item Sensor mount type in relation to sensor resonance
\item Data presentation - standard display formats for analysis - trends, watefall plots ...
\item Potencial causes for faults (p. 45) - use in vibration fault types
\end{itemize}
 
