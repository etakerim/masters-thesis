\chapter{Problem analysis}


\section{Condition monitoring}
Why monitor with vibrations, Wear process curve, Fault - failure, Levels are dependant on load and lifecycle stage

\subsection{Maintenance strategies}
\paragraph{Reactive}
\paragraph{Preventive}
\paragraph{Predictive}
- value for the factory and assess risks asociated with potencial fault to assign them importance

\paragraph{Diagnosis indicators} - what we montitor? - when it fails and early signs of faulty component
- RUL (Remaining useful life) models - disadvantage lot of similiar machines (homogenous) or lots of runns until failure
\begin{itemize}
\item Similarity
\item Degradation - used by standards
\item Survival
\end{itemize}

\subsection{Vibration fault types}
- frequency ranges 1 - 300 Hz (shaft), 300 - 1000 Hz, 1000 - 10000 Hz (early bearings)
- Base analytical models: Jeffcott rotor - rotor dynamics, Bearnings model
- Resonance frequencies of each part - machine must run at speeds not aligned with resonance frequencies - Campbell diagram - task for mechanical engineers
- Faults - reasons and frequency content
\begin{itemize}
\item Synchrounous response - based on RPM
\item Mass unbalance
\item Misalignment
\item Eccentricity
\item Bent or bow shaft
\item Cracked shaft
\item Rotor rubs - friction
\item Looseness
\item Auxiliery mechanical systems: Gearbox, Bearings, Belt 
\end{itemize}

\subsection{Technical standards}
\paragraph{ISO 20816}
Part 1
\begin{itemize}
\item Measurement units - displacement, velocity, acceleration
\item RMS, and max. amplitude = severity
\item Measurement points for sensors (axial, radial) - image, and 45 degrees
\item Evaluation zones - A, B, C, D - Severity chart (Annex B) - Degradation model
\item Opeartional limits - Alarm, Trips
\end{itemize}

\paragraph{ISO 13373}
\begin{itemize}
\item Sensor mount type in relation to sensor resonance
\item Data presentation - standard display formats for analysis - trends, watefall plots ...
\item Potencial causes for faults (p. 45) - use in vibration fault types
\end{itemize}




\section{Feature engineering}
Large domain knowledge with compared to other areas of machine learning (mechanics - physics)
\subsection{Preprocessing}
\begin{itemize}
\item Detrending - DC removal filter
\item Denoising 
	\begin{itemize}
	\item Gaussian smoothing filter 
	\item Adaptive FIR filter with LMS algorithm
	\item Wavelet thresholding with coif2 or db2
	\end{itemize}
\item Time synchronous averagings
\end{itemize}

\subsection{Feature extraction}

\paragraph{Statistical measures}
\begin{itemize}
\item Standard Deviation
\item Max. amplitude
\item RMS amplitude
\item Skewness
\item Kurtosis \\
---
\item Spectral centroid
\item RMS frequency
\item Root variance frequency
\item Spectral kurtosis / Fast kurtogram
\item Harmonic spectral deviation \\
---
\item Energy
\item Spectral negentropy
\item TKEO - Teager-Kaiser energy operator
\end{itemize}

\paragraph{Signal decompositions - sparse approximations}
Matching pursuit algorithm optimalization problem
\begin{itemize}
\item FFT - Short Time Fourier Transform with Hamming window and Welch averaging
\item CWT-SST - Synchrosqueezing Wavelet Transform (vs. Transient-extracting transform)
\item EEMD - Ensemble Empirical Mode Decomposition - to IMF - mode mixing problem
\item WPD - Wavelet Packet Decomposition  - to approximation and detail coef. (Fejer-Korovkin wavelet)
\item EWT - Empirical Wavelet Transform - (Meyer wavelet) 
\end{itemize}


\subsection{Feature transformation}
\begin{itemize}
\item Principal Component Analysis (PCA)
\item Log transformation (Box-Cox Transform) to normal distribution
\item Normalization (min-max, standardize) 
\end{itemize}

\subsection{Feature selection}
Filter method - SelectKBest  in evaluation phase
\begin{itemize}
\item Variance Threshold
\item Pearson correlation
\item ANOVA F-value
\item Mutal information
\item Fisher score
\item Spectral feature selection algorithm (SPEC)
\end{itemize}

\section{Diagnostics techniques}
Idenification of faulty states in data streams in semi-supervised learning
\subsection{Novelty detection}
\begin{itemize}
\item DenStream (Density based clustering - DBSCAN)
\item Half-space Trees (Isolation forest)
\end{itemize}

\subsection{Classification}
\begin{itemize}
\item kNN + Metric Tree (M-Tree for neighbourhood queries) + Mahalanobis distance / Cosine similarity
\item Naive Bayes + Count-min sketch
\item Label propagation (Temporal label propagation, LabelRankT) - RBF similarity
\end{itemize}


\section{Evaluation Datasets}
\paragraph{MAFAULDA}

\paragraph{CWRU}

\section{Sensor and microcontroller}
